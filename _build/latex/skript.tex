%% Generated by Sphinx.
\def\sphinxdocclass{jupyterBook}
\documentclass[letterpaper,10pt,french]{jupyterBook}
\ifdefined\pdfpxdimen
   \let\sphinxpxdimen\pdfpxdimen\else\newdimen\sphinxpxdimen
\fi \sphinxpxdimen=.75bp\relax
%% turn off hyperref patch of \index as sphinx.xdy xindy module takes care of
%% suitable \hyperpage mark-up, working around hyperref-xindy incompatibility
\PassOptionsToPackage{hyperindex=false}{hyperref}
%% memoir class requires extra handling
\makeatletter\@ifclassloaded{memoir}
{\ifdefined\memhyperindexfalse\memhyperindexfalse\fi}{}\makeatother

\PassOptionsToPackage{warn}{textcomp}

\catcode`^^^^00a0\active\protected\def^^^^00a0{\leavevmode\nobreak\ }
\usepackage{cmap}
\usepackage{fontspec}
\defaultfontfeatures[\rmfamily,\sffamily,\ttfamily]{}
\usepackage{amsmath,amssymb,amstext}
\usepackage{babel}



\setmainfont{FreeSerif}[
  Extension      = .otf,
  UprightFont    = *,
  ItalicFont     = *Italic,
  BoldFont       = *Bold,
  BoldItalicFont = *BoldItalic
]
\setsansfont{FreeSans}[
  Extension      = .otf,
  UprightFont    = *,
  ItalicFont     = *Oblique,
  BoldFont       = *Bold,
  BoldItalicFont = *BoldOblique,
]
\setmonofont{FreeMono}[
  Extension      = .otf,
  UprightFont    = *,
  ItalicFont     = *Oblique,
  BoldFont       = *Bold,
  BoldItalicFont = *BoldOblique,
]


\usepackage[Sonny]{fncychap}
\ChNameVar{\Large\normalfont\sffamily}
\ChTitleVar{\Large\normalfont\sffamily}
\usepackage[,numfigreset=1,mathnumfig]{sphinx}

\fvset{fontsize=\small}
\usepackage{geometry}


% Include hyperref last.
\usepackage{hyperref}
% Fix anchor placement for figures with captions.
\usepackage{hypcap}% it must be loaded after hyperref.
% Set up styles of URL: it should be placed after hyperref.
\urlstyle{same}


\usepackage{sphinxmessages}



        % Start of preamble defined in sphinx-jupyterbook-latex %
         \usepackage[Latin,Greek]{ucharclasses}
        \usepackage{unicode-math}
        % fixing title of the toc
        \addto\captionsenglish{\renewcommand{\contentsname}{Contents}}
        \hypersetup{
            pdfencoding=auto,
            psdextra
        }
        % End of preamble defined in sphinx-jupyterbook-latex %
        

\title{Cours Analyse}
\date{déc. 21, 2022}
\release{}
\author{Iaousse M'barek}
\newcommand{\sphinxlogo}{\vbox{}}
\renewcommand{\releasename}{}
\makeindex
\begin{document}

\ifdefined\shorthandoff
  \ifnum\catcode`\=\string=\active\shorthandoff{=}\fi
  \ifnum\catcode`\"=\active\shorthandoff{"}\fi
\fi

\pagestyle{empty}
\sphinxmaketitle
\pagestyle{plain}
\sphinxtableofcontents
\pagestyle{normal}
\phantomsection\label{\detokenize{intro::doc}}


\sphinxAtStartPar
Après avoir manipuler les mathématiques au lycée. Vous allez apprendre dans votre cursus la construction des objets mathématiques. Le présent cours pose les bases et introduit les outils dont vous aurez besoin par la suite de votre parcours. Il contiendra les éléments suivants :
\begin{enumerate}
\sphinxsetlistlabels{\arabic}{enumi}{enumii}{}{.}%
\item {} 
\sphinxAtStartPar
L’ensemble des réels ;

\item {} 
\sphinxAtStartPar
Les suites des nombres réels ;

\item {} 
\sphinxAtStartPar
Limites et continuité d’une fonction numérique ;

\item {} 
\sphinxAtStartPar
Dérivabilité d’une fonction numérique ;

\item {} 
\sphinxAtStartPar
Fonctions usuelles ;

\item {} 
\sphinxAtStartPar
Développements limités ;

\item {} 
\sphinxAtStartPar
Calcul Intégral.

\end{enumerate}

\sphinxAtStartPar
Cet élément fait partie du module MATHEMATIQUES qui a pour but de permettre aux étudiants d’acquérir une formation en analyse mathématique en tenant compte des besoins spécifiques à l’informatique.


\chapter{Les nombres réels}
\label{\detokenize{R:les-nombres-reels}}\label{\detokenize{R::doc}}
\sphinxAtStartPar
Le présent Chapitre contiendra :
\begin{enumerate}
\sphinxsetlistlabels{\arabic}{enumi}{enumii}{}{.}%
\item {} 
\sphinxAtStartPar
L’ensemble des nombres rationnels \(\mathbb{Q}\)

\item {} 
\sphinxAtStartPar
Propriétés de \(\mathbb{R}\)

\item {} 
\sphinxAtStartPar
Densité de \(\mathbb{Q}\) dans \(\mathbb{R}\)

\item {} 
\sphinxAtStartPar
Borne supérieure

\item {} 
\sphinxAtStartPar
Exercices

\end{enumerate}


\section{Les nombres rationnels}
\label{\detokenize{rationnels:les-nombres-rationnels}}\label{\detokenize{rationnels::doc}}

\subsection{L’écriture décimale}
\label{\detokenize{rationnels:l-ecriture-decimale}}
\sphinxAtStartPar
Par définition, l’ensemble des nombres rationnels est
\begin{equation*}
\begin{split}
\mathbb{Q}:=\{\frac{p}{q}\mid\, p\in \mathbb{Z},\,q\in \mathbb{N}^{*}\}
\end{split}
\end{equation*}
\sphinxAtStartPar
Par exemple : \(\frac{2}{5},\,\frac{-7}{71},\frac{3}{6}=\frac{1}{2},\,etc...\)

\sphinxAtStartPar
Les nombres décimaux, c’est\sphinxhyphen{}à\sphinxhyphen{}dire les nombres de la forme \(\frac{a}{10^{n}},\)  avec \(a\in \mathbb{Z}\), \(n\in \mathbb{N}\) fournissent d’autres exemples:
\begin{equation*}
\begin{split}
1,234 = 1234\times 10^{-3} =\frac{1234}{1000} \mbox{ et }
0,00345 = 345\times 10^{-5}=\frac{345}{100000}
\end{split}
\end{equation*}
\begin{sphinxadmonition}{note}{Définition}

\sphinxAtStartPar
Un nombre est rationnel si et seulement s’il admet une écriture décimale périodique ou finie.
\end{sphinxadmonition}

\begin{sphinxadmonition}{note}{Exemple}

\sphinxAtStartPar
le nombre \(\frac{3}{5}\) est rationnel car:
\begin{equation*}
\begin{split}
\frac{3}{5}=0,6\quad \frac{1}{3}=0,333\ldots
\end{split}
\end{equation*}\end{sphinxadmonition}

\sphinxAtStartPar
Nous n’allons pas donner la démonstration mais le sens direct ( \(\Rightarrow\) ) repose sur la division euclidienne. Pour la réciproque (\(\Leftarrow\)) voyons comment cela marche sur un exemple : Montrons que \(x = 12,34 2021 2021\ldots\) est un rationnel.

\sphinxAtStartPar
L’idée est d’abord de faire apparaître la partie périodique juste après la virgule. Ici la période commence deux chiffres après la virgule, donc on multiplie par 100 :
\begin{equation}\label{equation:rationnels:100x}
\begin{split}
100x=1234,2021 2021 \ldots
\end{split}
\end{equation}
\sphinxAtStartPar
Maintenant on va décaler tout vers la gauche de la longueur d’une période, donc ici on multiplie encore par 10 000 pour décaler de 4 chiffres :

\sphinxAtStartPar
Maintenant on va décaler tout vers la gauche de la longueur d’une période, donc ici on multiplie encore par 10 000 pour décaler de 4 chiffres :
\begin{equation}\label{equation:rationnels:10000x}
\begin{split}
10 000\times 100x= 1234 2021, 2021 \ldots
\end{split}
\end{equation}
\sphinxAtStartPar
Les parties après la virgule des deux lignes \eqref{equation:rationnels:100x} et \eqref{equation:rationnels:10000x} sont les mêmes, donc si on les soustrait en faisant \eqref{equation:rationnels:10000x} \sphinxhyphen{} \eqref{equation:rationnels:100x} alors les parties décimales s’annulent :
\begin{equation*}
\begin{split}
10 000 \times 100x-100x = 12 342 021-1234
\end{split}
\end{equation*}
\sphinxAtStartPar
Donc \(999 900x = 12 340 787\) donc \(x=\frac{12 340 787}{999 900},\) \(x\) est bien un nombre rationnel.


\subsection{\protect\(\sqrt{2}\protect\) n’est pas un nombre rationnel}
\label{\detokenize{rationnels:sqrt-2-n-est-pas-un-nombre-rationnel}}
\sphinxAtStartPar
Il existe des nombres qui ne sont pas rationnels, les irrationnels. Les nombres irrationnels apparaissent naturellement dans les figures géométriques : par exemple la diagonale d’un carré de côté 1 est le nombre irrationnel \(\sqrt{2}\) ; la circonférence d’un cercle de rayon \(\frac{1}{2}\) est \(\pi\) qui est également un nombre irrationnel. Enfin \(e = exp(1)\) est aussi.

\sphinxAtStartPar
Nous allons prouver que \(\sqrt{2}\) n’est pas un nombre rationnel.

\begin{sphinxadmonition}{note}{Proposition}

\sphinxAtStartPar
\(\sqrt{2}\) est un nombre irrationnel.
\end{sphinxadmonition}

\sphinxAtStartPar
Par l’absurde supposons que \(\sqrt{2}\) soit un nombre rationnel. Alors il existe des entiers \(p\in\mathbb{Z}\) et \(q\in \mathbb{N}^{*}\) tels que \(\sqrt{2}=\frac{p}{q}\) de plus, ce sera important pour la suite– on suppose que \(p\) et \(q\) sont premiers entre eux (c’est\sphinxhyphen{}à\sphinxhyphen{}dire que la fraction \(\frac{p}{q}\) est sous une écriture irréductible).

\sphinxAtStartPar
En élevant au carré, l’égalité \(\sqrt{2}=\frac{p}{q}\) devient \(2q^2=p^2\). Cette dernière égalité est une égalité d’entiers. L’entier de gauche est pair, donc on en déduit que \(p^2\) est pair ; en termes de divisibilité 2 divise \(p^2\).
Mais si 2 divise \(p^2\)

\sphinxAtStartPar
Alors 2 divise \(p\) (cela se prouve par facilement l’absurde). Donc il existe un entier \(p'\in \mathbb{Z}\) tel que \(p=2p'\)

\begin{sphinxadmonition}{note}{Démonstration}

\sphinxAtStartPar
Repartons de l’égalité \(2q^2=p^2\) et remplaçons \(p\) par \(2p'.\) Cela donne \(2q^2=4p'^{2}\). Donc \(q^2=2p'^{2}.\)
Maintenant cela entraîne que 2 divise \(q^2\) et comme avant alors 2 divise \(q.\)
Nous avons prouvé que 2 divise à la fois \(p\) et \(q.\) Cela rentre en contradiction avec le fait que \(p\) et \(q\) sont premiers entre eux. Notre hypothèse de départ est donc fausse : \(\sqrt{2}\) n’est pas un nombre rationnel.
\end{sphinxadmonition}

\sphinxAtStartPar
Comme ce résultat est important en voici une deuxième démonstration, assez différente, mais toujours par l’absurde.
Autre démonstration. Par l’absurde, supposons \(\sqrt{2}=\frac{p}{q},\) donc \(q\sqrt{2}=p\in \mathbb{N}.\) Considérons l’ensemble
\begin{equation*}
\begin{split}
\mathcal{N}:=\{n\in \mathbb{N}^{*}\mid n\sqrt{2}\in \mathbb{N}\}
\end{split}
\end{equation*}
\sphinxAtStartPar
Cet ensemble n’est pas vide car on vient de voir que \(q\sqrt{2}=p\in \mathbb{N}\) donc \(q\in \mathcal{N}.\) Ainsi \(\mathcal{N}\) est une partie non vide de \(\mathbb{N},\)
elle admet donc un plus petit élément \(n_0 :=\min \mathcal{N}.\)

\sphinxAtStartPar
Posons
\begin{equation*}
\begin{split}
n_1=n_0 \sqrt{2}-n_0=n_0 (\sqrt{2}-1)
\end{split}
\end{equation*}
\sphinxAtStartPar
Il découle de cette dernière égalité et de \(1 <\sqrt{2} < 2\) que \(0 < n_1 < n_0.\)

\sphinxAtStartPar
De plus \(n_1 \sqrt{2}= (n_0 \sqrt{2}-n_0
)\sqrt{2}= 2n_0-n_0 \sqrt{2}\in \mathbb{N}.\) Donc \(n_1\in \mathcal{N}\) et \(n_1 < n_0\)
: on vient de trouver un élément \(n_1\) de
\(\mathcal{N}\) strictement plus petit que \(n_0\) qui était le minimum. C’est une contradiction.
Notre hypothèse de départ est fausse, donc \(\sqrt{2}\) est un nombre irrationnel.


\section{Propriétés de \protect\(\mathbb{R}\protect\)}
\label{\detokenize{proprties:proprietes-de-mathbb-r}}\label{\detokenize{proprties::doc}}

\subsection{Addition et multiplication}
\label{\detokenize{proprties:addition-et-multiplication}}
\sphinxAtStartPar
Ce sont les propriétés dont on a habitué. Pour \(a,\,b,\,c\in\mathbb{R},\) on a:
\begin{itemize}
\item {} 
\sphinxAtStartPar
\(a+b=b+a\)

\item {} 
\sphinxAtStartPar
\(a\times b=b\times a\)

\item {} 
\sphinxAtStartPar
\(a+0=a\)

\item {} 
\sphinxAtStartPar
\( a\times 1=a \mbox{ si } a\neq0\)

\item {} 
\sphinxAtStartPar
\(a+b=0\Leftrightarrow a=-b\)

\item {} 
\sphinxAtStartPar
\(ab=1\Leftrightarrow a=\frac{1}{b}\)

\item {} 
\sphinxAtStartPar
\((a+b)+c=a+(b+c)\)

\item {} 
\sphinxAtStartPar
\((a\times b)\times c=a\times(b\times c)\)

\item {} 
\sphinxAtStartPar
\(a\times(b+c)=a\times b+a\times c\)

\item {} 
\sphinxAtStartPar
\(a\times b=0\Leftrightarrow(a=0 \mbox{ ou } b=0)\)

\end{itemize}

\sphinxAtStartPar
On résume toutes ces propriétés en disant que :

\begin{sphinxadmonition}{note}{Proposition}

\sphinxAtStartPar
\((\mathbb{R},+,\times)\) est un corps commutatif.
\end{sphinxadmonition}


\subsection{Ordre sur \protect\(\mathbb{R}\protect\)}
\label{\detokenize{proprties:ordre-sur-mathbb-r}}
\sphinxAtStartPar
Nous allons voir que les réels sont ordonnés. La notion d’ordre est générale et nous allons définir cette notion sur un
ensemble quelconque. Cependant gardez à l’esprit que pour nous \(E=\mathbb{R}\) et \(\mathcal{R}=\leq\)

\begin{sphinxadmonition}{note}{Définition}

\sphinxAtStartPar
Soit \(E\) un ensemble.
\begin{enumerate}
\sphinxsetlistlabels{\arabic}{enumi}{enumii}{}{.}%
\item {} 
\sphinxAtStartPar
Une relation \(\mathcal{R}\) sur \(E\) est un sous\sphinxhyphen{}ensemble de l’ensemble produit \(E\times E.\) Pour \((x,y)\in E\times E,\) on dit que \(x\) est en relation avec \(y\) et on note \(x\mathcal{R}y\) pour dire que \((x,y)\in R.\)

\item {} 
\sphinxAtStartPar
Une relation \(\mathcal{R}\) est une relation d’ordre si
\begin{itemize}
\item {} 
\sphinxAtStartPar
\(\mathcal{R}\) est \sphinxstyleemphasis{réflexive}: Pour tout \(x\in E,\) \(x\mathcal{R}x\)

\item {} 
\sphinxAtStartPar
\(\mathcal{R}\) est \sphinxstyleemphasis{antisymétrique}: pour tout \(x,y\in E,\) \((x\mathcal{R}y\:\mbox{et} y\mathcal{R}x) \Rightarrow x=y.\)

\item {} 
\sphinxAtStartPar
\(\mathcal{R}\) est \sphinxstyleemphasis{transitive}: pour tout \(x,y,z\in E,\) \((x\mathcal{R}y\:\mbox{et} y\mathcal{R}z)\Rightarrow x\mathcal{R}z\)

\end{itemize}

\end{enumerate}
\end{sphinxadmonition}

\begin{sphinxadmonition}{note}{Définition}

\sphinxAtStartPar
Une relation d’ordre \(\mathcal{R}\) sur un ensemble \(E\) est totale si pour tout \(x, y\in E\) on a \(x\mathcal{R}y\) ou \(y\mathcal{R}x.\) On dit aussi que
\((E,\mathcal{R})\) est un ensemble totalement ordonné.
\end{sphinxadmonition}

\begin{sphinxadmonition}{note}{Proposition}

\sphinxAtStartPar
La relation \(\leq\) sur \(\mathbb{R}\) est une relation d’ordre, et de plus, elle est totale.
\end{sphinxadmonition}

\sphinxAtStartPar
Nous avons donc:
\begin{itemize}
\item {} 
\sphinxAtStartPar
pour tout \(x\in \mathbb{R}, x\leq x,\)

\item {} 
\sphinxAtStartPar
pour tout \(x, y \in \mathbb{R},\) si \(x\leq y\) et \(y\leq x\) alors \(x=y,\)

\item {} 
\sphinxAtStartPar
pour tout \(x, y, z\in\mathbb{R}\) si \(x \leq y\) et \(y \leq z\) alors \(x \leq z.\)

\end{itemize}

\begin{sphinxadmonition}{note}{Remarque}

\sphinxAtStartPar
Pour \((x, y)\in \mathbb{R}^{2}\) on a par définition:

\sphinxAtStartPar
\(x\leq y \Leftrightarrow y-x\in \mathbb{R}_{+}\)

\sphinxAtStartPar
\(x<y\Leftrightarrow x\leq y\quad\mbox{et}\quad x\neq y\)

\sphinxAtStartPar
Les opérations de \(\mathbb{R}\) sont compatibles avec la relation d’ordre \(\leq\) au sens suivant, pour des réels \(a, b,c, d:\)

\sphinxAtStartPar
\(a\leq b \quad\mbox{et}\quad c\leq d)\Rightarrow a+c\leq b+d\)

\sphinxAtStartPar
\(a\leq b \quad\mbox{et}\quad c\geq0)\Rightarrow a\times c\leq b\times c\)
\end{sphinxadmonition}

\begin{sphinxadmonition}{note}{Définition}

\sphinxAtStartPar
On définit le maximum de deux réels \(a\) et \(b\) par:
\begin{equation*}
\begin{split}
\max(a,b)=\left\{
\begin{array}{ll}
a\mbox{ si } a\geq b\\
b \mbox{ si } b>a
\end{array}
\right.
\end{split}
\end{equation*}\end{sphinxadmonition}


\subsection{Propriété d’Archimède}
\label{\detokenize{proprties:propriete-d-archimede}}
\begin{sphinxadmonition}{note}{Proposition (Propriété d’Archimède)}

\sphinxAtStartPar
\(\mathbb{R}\) est \sphinxstyleemphasis{archimédien}, c’est\sphinxhyphen{}à\sphinxhyphen{}dire:
\begin{equation*}
\begin{split}
\forall x\in \mathbb{R},\exists n\in \mathbb{N};\,n> x
\end{split}
\end{equation*}
\sphinxAtStartPar
Autrementdit, Pour tout réel \(x,\) il existe un entier naturel \(n\) strictement plus grand que \(x.\)
\end{sphinxadmonition}

\sphinxAtStartPar
Cette propriété peut sembler évidente, elle est pourtant essentielle puisque elle permet de définir la partie entière
d’un nombre réel:

\begin{sphinxadmonition}{note}{Proposition (partie entière)}

\sphinxAtStartPar
Soit \(x\in R,\) il existe un unique entier relatif, la partie entière notée \(E(x),\) tel que:
\begin{equation*}
\begin{split}
E(x)\leq x< E(x)+1
\end{split}
\end{equation*}\end{sphinxadmonition}

\begin{sphinxadmonition}{note}{Exemple}
\begin{itemize}
\item {} 
\sphinxAtStartPar
\(E(2, 853) = 2, E(\pi) = 3, E(-3,5) =-4.\)

\item {} 
\sphinxAtStartPar
\(E(x)=3\Leftrightarrow 3\leq x <4\)

\end{itemize}
\end{sphinxadmonition}

\sphinxAtStartPar
Pour la démonstration de la proposition de la \DUrole{xref,myst}{partie entière} il y a deux choses à établir: d’abord qu’un tel entier \(E(x)\) existe et ensuite
qu’il est unique:

\begin{sphinxadmonition}{note}{Preuve}
\begin{itemize}
\item {} 
\sphinxAtStartPar
\sphinxstylestrong{Existence}

\end{itemize}

\sphinxAtStartPar
Supposons \(x>0\).

\sphinxAtStartPar
Par la propriété d’Archimède il existe \(n\in N\) tel que \(n>x.\)

\sphinxAtStartPar
L’ensemble \(K:=\{k\in \mathbb{N}; k\leq x\}\) est donc fini (car pour tout \(k\) dans \(K,\) on a \(0\leq k \leq n\)).

\sphinxAtStartPar
Il admet donc un plus grand élément
\(k_{max} =\max K.\)

\sphinxAtStartPar
On a alors \(k_{max}\leq x\) car \(k_{max} \in K,\) et \(k_{max} + 1 > x\) car \(k_{max} + 1 \notin K.\) Donc \(k_{max}\leq x < k_{max} + 1\) et on prend donc \(E(x)= k_{max}.\)
\begin{itemize}
\item {} 
\sphinxAtStartPar
\sphinxstylestrong{Unicité:}

\end{itemize}

\sphinxAtStartPar
Si \(k\) et \(l\) sont deux entiers relatifs vérifiant \(k \leq x < k + 1\) et \(l \leq x < l + 1,\) on a donc \(k \leq x < l + 1\).

\sphinxAtStartPar
donc par transitivité \(k < l + 1.\)

\sphinxAtStartPar
En échangeant les rôles de \(l\) et \(k,\) on a aussi \(l< k + 1.\)

\sphinxAtStartPar
On en conclut que \(l-1 < k < l + 1,\) mais
il n’y a qu’un seul entier compris strictement entre \(l-1\) et \(l+1,\) c’est \(l.\)

\sphinxAtStartPar
Ainsi \(k = l.\)

\sphinxAtStartPar
Le cas \(x < 0\) est similaire.
\end{sphinxadmonition}


\subsection{Valeur absolue}
\label{\detokenize{proprties:valeur-absolue}}
\begin{sphinxadmonition}{note}{Définition}

\sphinxAtStartPar
Pour un nombre réel \(x,\) on définit la valeur absolue de \(x\) par:
\begin{equation*}
\begin{split}
|x|=\left\{
\begin{array}{ll}
x\quad\mbox{si}\quad x\geq0,\\
-x \quad\mbox{si}\quad x<0
\end{array}
\right.
\end{split}
\end{equation*}\end{sphinxadmonition}

\begin{sphinxadmonition}{note}{Proposition}
\begin{enumerate}
\sphinxsetlistlabels{\arabic}{enumi}{enumii}{}{.}%
\item {} 
\sphinxAtStartPar
\(|x|\geq0,\quad |x|=|-x|;\quad |x|>0\Leftrightarrow x\neq0\)

\item {} 
\sphinxAtStartPar
\(\sqrt{x^2}=|x|\)

\item {} 
\sphinxAtStartPar
\(|xy|=|x||y|\)

\item {} 
\sphinxAtStartPar
Inégalité triangulaire: \(|x+y|\leq |x|+|y|\)

\item {} 
\sphinxAtStartPar
\(||x|-|y||\leq |x-y|\)

\end{enumerate}
\end{sphinxadmonition}

\sphinxAtStartPar
Sur la droite numérique, \(|x-y|\) représente la distance entre les réels \(x\) et \(y\) ; en particulier \(|x|\) représente la distance
entre les réels \(x\) et 0. De plus on a \(|x-a|<r\Leftrightarrow a-r<x<a+r.\)


\section{Densité de \protect\(\mathbb{Q}\protect\) dans \protect\(\mathbb{R}\protect\)}
\label{\detokenize{proprties:densite-de-mathbb-q-dans-mathbb-r}}

\subsection{Intervalle}
\label{\detokenize{proprties:intervalle}}
\begin{sphinxadmonition}{note}{Définition}

\sphinxAtStartPar
Un intervalle de \(\mathbb{R}\) est un sous\sphinxhyphen{}ensemble \(I\) de \(\mathbb{R}\) vérifiant la propriété:
\begin{equation*}
\begin{split}
\forall a,\,b\in I,\;\forall x\in \mathbb{R},\;(a\leqslant x\leqslant b\Rightarrow x\in I)
\end{split}
\end{equation*}\end{sphinxadmonition}

\begin{sphinxadmonition}{note}{Remarque}

\sphinxAtStartPar
Par définition;
\begin{itemize}
\item {} 
\sphinxAtStartPar
\(I=\varnothing\) est un intervalle.

\item {} 
\sphinxAtStartPar
\(I=\mathbb{R}\) est aussi un intervalle.

\end{itemize}
\end{sphinxadmonition}

\begin{sphinxadmonition}{note}{Définition}

\sphinxAtStartPar
Un intervalle ouvert est un sous\sphinxhyphen{}ensemble de \(\mathbb{R}\) de la forme \(]a,b[= \{x\in \mathbb{R},\,a<x<b\}\), où \(a\) et \(b\) sont des éléments de \(\mathbb{R}.\)
\end{sphinxadmonition}

\sphinxAtStartPar
La notion de voisinage sera utile pour les limites.

\begin{sphinxadmonition}{note}{Définition}

\sphinxAtStartPar
Soit \(a\) un réel, \(V\subset R\) un sous\sphinxhyphen{}ensemble. On dit que \(V\) est un voisinage de \(a\) s’il existe un intervalle ouvert \(I\) tel que \(a \in I\) et \(I \subset V.\)
\end{sphinxadmonition}


\subsection{Densité}
\label{\detokenize{proprties:densite}}
\begin{sphinxadmonition}{note}{Théorème}
\begin{enumerate}
\sphinxsetlistlabels{\arabic}{enumi}{enumii}{}{.}%
\item {} 
\sphinxAtStartPar
\(\mathbb{Q}\) est dense dans \(\mathbb{R}\): tout intervalle ouvert (non vide) de \(\mathbb{R}\) contient une infinité de rationnels.

\item {} 
\sphinxAtStartPar
\(\mathbb{R}\setminus \mathbb{Q}\) est dense dans \(\mathbb{R}\) : tout intervalle ouvert (non vide) de \(\mathbb{R}\) contient une infinité d’irrationnels.

\end{enumerate}
\end{sphinxadmonition}


\section{Borne supérieure}
\label{\detokenize{proprties:borne-superieure}}

\subsection{Maximum, minimum}
\label{\detokenize{proprties:maximum-minimum}}
\begin{sphinxadmonition}{note}{Définition}

\sphinxAtStartPar
Soit \(A\) une partie non vide de \(\mathbb{R}.\) Un réel \(\alpha\) est un plus grand élément de \(A\) si :
\(\alpha\in A\) et \(\forall x\in A,\,x\leqslant \alpha.\)
S’il existe, le plus grand élément est unique, on le note alors \(\max A.\)
Le plus petit élément de \(A,\) noté \(\min A,\) s’il existe est le réel \(\alpha\) tel que \(\alpha\in A\) et \(\forall x\in A, \,x\geqslant \alpha.\)

\sphinxAtStartPar
Le plus grand élément s’appelle aussi le maximum et le plus petit élément, le minimum. Il faut garder à l’esprit que
le plus grand élément ou le plus petit élément n’existent pas toujours.
\end{sphinxadmonition}

\begin{sphinxadmonition}{note}{Exemple}
\begin{itemize}
\item {} 
\sphinxAtStartPar
3 est un majorant de \(]0, 2[ ;\)

\item {} 
\sphinxAtStartPar
−7,\(\pi,\) 0 sont des minorants de \(]0,+\infty[\) mais il n’y a pas de majorant.

\end{itemize}

\sphinxAtStartPar
Si un majorant (resp. un minorant) de \(A\) existe on dit que \(A\) est majorée (resp. minorée).
Comme pour le minimum et le maximum il n’existe pas toujours de majorant ni de minorant, en plus on n’a pas
l’unicité.

\sphinxAtStartPar
Soit \(A =[0,1[\)
\begin{enumerate}
\sphinxsetlistlabels{\arabic}{enumi}{enumii}{}{.}%
\item {} 
\sphinxAtStartPar
les majorants de \(A\) sont exactement les éléments de \([1,+\infty[,\)

\item {} 
\sphinxAtStartPar
les minorants de \(A\) sont exactement les éléments de \(]−\infty,0].\)

\end{enumerate}
\end{sphinxadmonition}


\subsection{Borne supérieure, borne inférieure}
\label{\detokenize{proprties:borne-superieure-borne-inferieure}}
\begin{sphinxadmonition}{note}{Définition}

\sphinxAtStartPar
Soit \(A\) une partie non vide de \(\mathbb{R}\) et \(\alpha\) un réel.
\begin{enumerate}
\sphinxsetlistlabels{\arabic}{enumi}{enumii}{}{.}%
\item {} 
\sphinxAtStartPar
\(\alpha\) est la borne supérieure de \(A\) si \(\alpha\) est un majorant de \(A\) et si c’est le plus petit des majorants. S’il existe on le
note \(\sup A.\)

\item {} 
\sphinxAtStartPar
\(\alpha\) est la borne inférieure de \(A\) si \(\alpha\) est un minorant de \(A\) et si c’est le plus grand des minorants. S’il existe on le
note \(\inf A.\)

\end{enumerate}
\end{sphinxadmonition}

\begin{sphinxadmonition}{note}{Exemple}

\sphinxAtStartPar
Soit \(A =]0,1].\)
\begin{enumerate}
\sphinxsetlistlabels{\arabic}{enumi}{enumii}{}{.}%
\item {} 
\sphinxAtStartPar
\(\sup A = 1\) : en effet les majorants de \(A\) sont les éléments de \([1,+\infty[.\) Donc le plus petit des majorants est 1.

\item {} 
\sphinxAtStartPar
\(\inf A = 0:\) les minorants sont les éléments de \(] −\infty,0]\) donc le plus grand des minorants est 0.

\end{enumerate}
\begin{itemize}
\item {} 
\sphinxAtStartPar
\(\sup[a, b] = b,\)

\item {} 
\sphinxAtStartPar
\(\inf[a, b]=a,\)

\item {} 
\sphinxAtStartPar
\(\sup]a, b[= b,\)

\item {} 
\sphinxAtStartPar
\(]0,+\infty[\) n’admet pas de borne supérieure,

\item {} 
\sphinxAtStartPar
\(\inf]0,+\infty[= 0.\)

\end{itemize}
\end{sphinxadmonition}

\begin{sphinxadmonition}{note}{Théorème}

\sphinxAtStartPar
Toute partie de \(\mathbb{R}\) non vide et majorée admet une borne supérieure.
\end{sphinxadmonition}

\sphinxAtStartPar
De la même façon : Toute partie de \(\mathbb{R}\) non vide et minorée admet une borne inférieure.

\begin{sphinxadmonition}{note}{Proposition (Caractérisation de la borne supérieure)}

\sphinxAtStartPar
Soit \(A\) une partie non vide et majorée de \(\mathbb{R}.\) La borne supérieure de \(A\) est l’unique réel \(\sup A\) tel que
\begin{enumerate}
\sphinxsetlistlabels{\arabic}{enumi}{enumii}{}{.}%
\item {} 
\sphinxAtStartPar
si \(x\in A,\) alors \(x \leqslant\sup A,\)

\item {} 
\sphinxAtStartPar
pour tout \(y < \sup A,\) il existe \(x\in A\) tel que \(y < x.\)

\end{enumerate}
\end{sphinxadmonition}


\section{Exercices}
\label{\detokenize{exo1:exercices}}\label{\detokenize{exo1::doc}}

\subsection{Exercice 1}
\label{\detokenize{exo1:exercice-1}}
\sphinxAtStartPar
Comment définir \(\max(a,b,c),\) \(\max(a_1,...,a_n)\)? Et \(\min(a,b)\)?


\subsection{Exercice 2}
\label{\detokenize{exo1:exercice-2}}\begin{enumerate}
\sphinxsetlistlabels{\arabic}{enumi}{enumii}{}{.}%
\item {} 
\sphinxAtStartPar
Écrire les nombres suivants sous forme d’une fraction : \(0,1212\) ;\(0,1212 ....\); \(78,33454545...\)

\item {} 
\sphinxAtStartPar
Sachant \(\sqrt{2}\notin\mathbb{Q},\) montrer que \(2-3\sqrt{2}\notin\mathbb{Q},\) \(1-\frac{1}{\sqrt{2}}\notin\mathbb{Q}.\)

\item {} 
\sphinxAtStartPar
Notons \(D\) l’ensemble des nombres de la forme \(\frac{a}{2^{n}}\) avec \(a\in\mathbb{Z}\) et \(n\in \mathbb{N}\). Montrer que \(\frac{1}{3}\notin D\). Trouver \(x\in D\) tel que \(1234<x<1234,001\)

\item {} 
\sphinxAtStartPar
Montrer que \(\frac{\sqrt{2}}{\sqrt{3}}\notin \mathbb{Q}.\)

\end{enumerate}


\subsection{Exercice 3}
\label{\detokenize{exo1:exercice-3}}\begin{enumerate}
\sphinxsetlistlabels{\arabic}{enumi}{enumii}{}{.}%
\item {} 
\sphinxAtStartPar
Démontrer que si \( r\in \mathbb{Q}\) et \(x \notin \mathbb{Q}\) alors \(r +x \notin \mathbb{Q}\) et si \(r \neq 0\) alors \(rx \notin \mathbb{Q}\).

\item {} 
\sphinxAtStartPar
Montrer que \(\sqrt{2}\notin \mathbb{Q}\).

\item {} 
\sphinxAtStartPar
En déduire que : entre deux nombres rationnels il y a toujours un nombre irrationnel.

\end{enumerate}


\subsection{Exercice 4}
\label{\detokenize{exo1:exercice-4}}
\sphinxAtStartPar
Montrer que \(\dfrac{\ln (3)}{ln (2)}\) est irrationnel.


\subsection{Exercice 5}
\label{\detokenize{exo1:exercice-5}}
\sphinxAtStartPar
Le maximum de deux nombres \(x\), \(y\) (c’est\sphinxhyphen{}à\sphinxhyphen{}dire le plus grand des deux) est noté \(\max(x, y)\). De même on notera
\(\min(x, y)\) le plus petit des deux nombres \(x\), \(y\). Démontrer que :
\(\max(x, y) = \dfrac{x+y+|x−y|}{2}\) et \(\min(x, y) = \dfrac{x+y−|x−y|}{2}\).

\sphinxAtStartPar
Trouver une formule pour \(\max(x, y,z)\).


\subsection{Exercice 6}
\label{\detokenize{exo1:exercice-6}}
\sphinxAtStartPar
Déterminer la borne supérieure et inférieure (si elles existent) de : \(A = \{u_n | n \in \mathbb{N}\}\) en posant \(u_n = 2^n\)
si \(n\) est pair et \(u_n = 2^{−n}\) sinon.


\subsection{Exercice 7}
\label{\detokenize{exo1:exercice-7}}
\sphinxAtStartPar
Déterminer (s’ils existent) : les majorants, les minorants, la borne supérieure, la borne inférieure, le plus grand
élément, le plus petit élément des ensembles suivants :
\begin{equation*}
\begin{split}
[0, 1]\cap \mathbb{Q}; ]0, 1[\cap \mathbb{Q}; \mathbb{N}; \left\{(-1)^n + \dfrac{1}{n^2} | n \in \mathbb{N}^\star \right\}
\end{split}
\end{equation*}

\subsection{Exercice 8}
\label{\detokenize{exo1:exercice-8}}
\sphinxAtStartPar
Soient \(A\) et \(B\) deux parties bornées de \(\mathbb{R}\). On note \(A+B = \{a+b | (a,b) \in  A\times B\}\).
\begin{enumerate}
\sphinxsetlistlabels{\arabic}{enumi}{enumii}{}{.}%
\item {} 
\sphinxAtStartPar
Montrer que \(\sup A+\sup B\) est un majorant de \(A+B\).

\item {} 
\sphinxAtStartPar
Montrer que \(\sup(A+B) = \sup A+\sup B\).

\end{enumerate}


\subsection{Exercice 9}
\label{\detokenize{exo1:exercice-9}}
\sphinxAtStartPar
Soit \(A\) et \(B\) deux parties bornées de \(\mathbb{R}\). Vrai ou faux?
\begin{enumerate}
\sphinxsetlistlabels{\arabic}{enumi}{enumii}{}{.}%
\item {} 
\sphinxAtStartPar
\(A \subset B \rightarrow \sup A \leq \sup B\),

\item {} 
\sphinxAtStartPar
\(A \subset B \rightarrow \inf A \leq \inf B\),

\item {} 
\sphinxAtStartPar
\(\sup(A\cup B) = \max(\sup A,\sup B)\),

\item {} 
\sphinxAtStartPar
\(\sup(A+B) < \sup A+\sup B\),

\item {} 
\sphinxAtStartPar
\(\sup(−A) = −\inf A\),

\item {} 
\sphinxAtStartPar
\(\sup A+\inf B \leq \sup(A+B)\).

\end{enumerate}


\subsection{Exercice 10}
\label{\detokenize{exo1:exercice-10}}\begin{enumerate}
\sphinxsetlistlabels{\arabic}{enumi}{enumii}{}{.}%
\item {} 
\sphinxAtStartPar
Soient \(x\) et \(y\) deux réels. Montrer que \(|x|\geq ||x+y|-|y||.\)

\item {} 
\sphinxAtStartPar
Soient \(x_1,\ldots,x_n\) des réels. Montrer que \(|x_1 +\ldots+ x_n|\leq|x1| + \ldots + |x_n|\). Dans quel cas a\sphinxhyphen{}t\sphinxhyphen{}on égalité?

\item {} 
\sphinxAtStartPar
Soient \(x, y > 0\) des réels. Comparer \(E(x + y)\) avec \(E(x)+E( y).\) Comparer \(E(xy)\) et \(E(x)E(y).\)

\end{enumerate}


\chapter{Les suites des nombres réels}
\label{\detokenize{suite:les-suites-des-nombres-reels}}\label{\detokenize{suite::doc}}
\sphinxAtStartPar
Le présent Chapitre contiendra :
\begin{enumerate}
\sphinxsetlistlabels{\arabic}{enumi}{enumii}{}{.}%
\item {} 
\sphinxAtStartPar
Suites réelles: Définitions générales

\item {} 
\sphinxAtStartPar
Convergence d’une suite réelle

\item {} 
\sphinxAtStartPar
Suites adjacentes

\item {} 
\sphinxAtStartPar
Suites récurrentes

\item {} 
\sphinxAtStartPar
Exercices

\end{enumerate}


\section{Suites réelles: Définitions générales}
\label{\detokenize{suites:suites-reelles-definitions-generales}}\label{\detokenize{suites::doc}}
\begin{sphinxadmonition}{note}{Définition}

\sphinxAtStartPar
Une suite est une application \(u:\mathbb{N}\rightarrow\mathbb{R}.\)

\sphinxAtStartPar
Pour \(n\in \mathbb{N},\) on note \(u(n)\) par \(u_n\) et on l’appelle \(n^{ième}\) terme ou terme général de la suite \(u.\)
\end{sphinxadmonition}

\sphinxAtStartPar
Une suite \(u:\mathbb{N}\rightarrow\mathbb{R}\) est plus souvent notée par \((u_n)_{n\in \mathbb{N}},\,(u_n)_{n\geq0}\) ou simplement par \((u_n).\)

\sphinxAtStartPar
Si une suite \(u\) est définie à partir d’un certain entier naturel \(n_0>0,\) alors dans ce cas on note cette suite par \((u_n)_{n\geq n_0}.\)

\begin{sphinxadmonition}{note}{Exemple}
\begin{enumerate}
\sphinxsetlistlabels{\arabic}{enumi}{enumii}{}{.}%
\item {} 
\sphinxAtStartPar
Soit \((u_n)\) la suite définie par \(u_n=\sqrt{n}-1.\) Les cinq premiers termes de cette suite sont \(-1,0,\sqrt{2}-1,\sqrt{3}-1,1.\)

\item {} 
\sphinxAtStartPar
Soit \((v_n)_{n\geq1}\) la suite définie par \(v_n=\dfrac{1}{n(n+1)}.\) Les trois premiers termes de cette suite sont \(\dfrac{1}{2},\dfrac{1}{6},\dfrac{1}{12}\)

\item {} 
\sphinxAtStartPar
\(((-1)^n)\) est une suite qui prends deux valeurs: 1 et \(-1.\)

\end{enumerate}
\end{sphinxadmonition}

\begin{sphinxadmonition}{note}{Définition}

\sphinxAtStartPar
Soit \((u_n)_{n\geq n_0}\) une suite.
\begin{enumerate}
\sphinxsetlistlabels{\arabic}{enumi}{enumii}{}{.}%
\item {} 
\sphinxAtStartPar
\((u_n)_{n\geq n_0}\) est dite une suite croissante (resp. strictement croissante) si \(u_{n+1}\geq u_n\) (resp. \(u_{n+1}>u_n\)), \(\forall n\geq n_0.\)

\item {} 
\sphinxAtStartPar
\((u_n)_{n\geq n_0}\) est dite une suite décroissante (resp. strictement décroissante) si \(u_{n+1}\leq u_n\) (resp. \(u_{n+1}<u_n\)), \(\forall n\geq n_0.\)

\end{enumerate}
\end{sphinxadmonition}

\begin{sphinxadmonition}{note}{Exemple}
\begin{enumerate}
\sphinxsetlistlabels{\arabic}{enumi}{enumii}{}{.}%
\item {} 
\sphinxAtStartPar
La suite \((u_n)_{n\geq1}\) définie par \(u_n=\dfrac{1}{n^2},\;\forall n\in \mathbb{N}^{*},\) est strictement décroissante. En effet, on a

\end{enumerate}
\begin{eqnarray*}
u_{n+1}-u_n
&=& \dfrac{1}{(n+1)^2}-\dfrac{1}{n^2}\\
&=& \dfrac{n^2-(n+1)^2}{n^2(n+1)^2}\\
&=& \dfrac{n^2-n^2-2n-1}{n^2(n+1)^2}\\
&=& -\dfrac{2n+1}{n^2(n+1)^2}\\
& & <0
\end{eqnarray*}\begin{enumerate}
\sphinxsetlistlabels{\arabic}{enumi}{enumii}{}{.}%
\item {} 
\sphinxAtStartPar
La suite \((v_n)\) définie par \(v_n=e^n,\;e^{n+1}-e^n=e^n e-e^n=e^n (e-1)>0.\)

\end{enumerate}
\end{sphinxadmonition}

\begin{sphinxadmonition}{note}{Remarque}

\sphinxAtStartPar
Si \((u_n)_{n\geq n_0}\) est une suite à termes strictement positifs, alors
\begin{enumerate}
\sphinxsetlistlabels{\arabic}{enumi}{enumii}{}{.}%
\item {} 
\sphinxAtStartPar
\((u_n)_{n\geq n_0}\) est croissante si, et seulement si \(\dfrac{u_{n+1}}{u_n}\geq1,\;\forall n\geq n_0.\)

\item {} 
\sphinxAtStartPar
\((u_n)_{n\geq n_0}\) est décroissante si, et seulement si \(\dfrac{u_{n+1}}{u_n}\leq 1,\;\forall n\geq n_0.\)

\end{enumerate}

\sphinxAtStartPar
Considérons à nouveau la suite \((v_n)\) telle que \(v_n=e^n,\;\forall n\in \mathbb{N}.\) On sait que \(v_n>0,\;\forall n\in \mathbb{N}.\) De plus, pour tout \(n\in \mathbb{N}\) on a
\begin{equation*}
\begin{split}
\dfrac{v_{n+1}}{v_n}=\dfrac{e^{n+1}}{e^n}=e^{(n+1)-n}=e
\end{split}
\end{equation*}
\sphinxAtStartPar
Donc \(\dfrac{v_{n+1}}{v_n}>1,\;\forall n\in \mathbb{N}.\) Ainsi la suite \((v_n)\) est strictement croissante.
\end{sphinxadmonition}

\begin{sphinxadmonition}{note}{Remarque}

\sphinxAtStartPar
Il se peut qu’une suite ne soit ni croissante ni décroissante. Voilà deux exemples:
\begin{enumerate}
\sphinxsetlistlabels{\arabic}{enumi}{enumii}{}{.}%
\item {} 
\sphinxAtStartPar
La suite \((u_n)\) telle que \(u_n=(-1)^n\) est une suite ni croissante ni décroissante.

\item {} 
\sphinxAtStartPar
Considérons la suite \((v_n)\) définie par \(v_n=\sin(n\dfrac{\pi}{2}),\;n\in\mathbb{N}.\) Ces premiers termes sont:

\end{enumerate}

\sphinxAtStartPar
\(v_0=\sin(0)=0,\;v_1=\sin(\dfrac{\pi}{2}),\;v_2=\sin(\pi)=0,\;v_3=\sin(3\dfrac{\pi}{2})=-1,\;v_4=\sin(2\pi)=0,...\)
\end{sphinxadmonition}

\begin{sphinxadmonition}{note}{Définition}

\sphinxAtStartPar
Une suite \((u_n)_{n\geq n_0}\) est dite monotone (resp. strictement monotone) si \((u_n)_{n\geq n_0}\) est croissante ou décroissante (resp. strictement croissante ou strictement décroissante).
\end{sphinxadmonition}

\begin{sphinxadmonition}{note}{Définition}

\sphinxAtStartPar
Soit \((u_n)_{n\geq n_0}\) une suite.
\begin{enumerate}
\sphinxsetlistlabels{\arabic}{enumi}{enumii}{}{.}%
\item {} 
\sphinxAtStartPar
\((u_n)_{n\geq n_0}\) est dite majorée si \(\exists M\in \mathbb{R}\) tel que \(u_n\leq M,\;\forall n\geq n_0.\)

\item {} 
\sphinxAtStartPar
\((u_n)_{n\geq n_0}\) est dite minorée si \(\exists m\in \mathbb{R}\) tel que \(u_n\geq M,\;\forall n\geq n_0.\)

\item {} 
\sphinxAtStartPar
\((u_n)_{n\geq n_0}\) est dite bornée si \((u_n)_{n\geq n_0}\) est minorée et majorée.

\end{enumerate}
\end{sphinxadmonition}

\begin{sphinxadmonition}{note}{Exemple}

\sphinxAtStartPar
On considère les trois suites \((u_n),\;(v_n)\) et \((w_n)\) définies par leurs termes généraux
\begin{equation*}
\begin{split}
u_n=1+n^2,\quad v_n=5-n(n+1),\quad w_n=2+\dfrac{1}{n+1},\quad n\in \mathbb{N}
\end{split}
\end{equation*}\begin{itemize}
\item {} 
\sphinxAtStartPar
Pour tout \(n\in \mathbb{N},\) on a \(u_n\geq1.\) Donc \((u_n)\) est minorée par 1.

\item {} 
\sphinxAtStartPar
Pour tout \(n\in \mathbb{N},\) on a \(v_n\leq1.\) Donc \((v_n)\) est majorée par 5.

\item {} 
\sphinxAtStartPar
Pour tout \(n\in \mathbb{N},\) on a \(2<w_n\leq 3.\) La suite est donc bornée.

\end{itemize}
\end{sphinxadmonition}

\begin{sphinxadmonition}{note}{Remarque}

\sphinxAtStartPar
Soit \((u_n)_{n\geq n_0}\) une suite réelle.
\begin{enumerate}
\sphinxsetlistlabels{\arabic}{enumi}{enumii}{}{.}%
\item {} 
\sphinxAtStartPar
Si \((u_n)_{n\geq n_0}\) est croissante, alors \((u_n)_{n\geq n_0}\) est minorée par son premier terme.

\item {} 
\sphinxAtStartPar
Si \((u_n)_{n\geq n_0}\) est décroissante, alors \((u_n)_{n\geq n_0}\) est majorée par son premier terme.

\end{enumerate}
\end{sphinxadmonition}

\begin{sphinxadmonition}{note}{Proposition}

\sphinxAtStartPar
Soit \((u_n)_{n\geq n_0}\) une suite réelle. Alors \((u_n)_{n\geq n_0}\) est bormée si et seulement si
\begin{equation*}
\begin{split}
\exists M\geq0\;\mbox{tel que}\;\forall n\geq n_0:\quad |u_n|\leq M
\end{split}
\end{equation*}\end{sphinxadmonition}


\section{Convergence d’une suite réelle}
\label{\detokenize{suites:convergence-d-une-suite-reelle}}
\begin{sphinxadmonition}{note}{Définition}

\sphinxAtStartPar
On dit qu’une suite réelle \((u_n)\) converge vers un réel \(\ell\) lorsque pour tout \(\varepsilon>0,\) il existe \(N\in \mathbb{N}\) tel que pour tout entier \(n>N,\) on a \(|u_n-\ell|<\varepsilon.\)

\sphinxAtStartPar
Dans ce cas, \(\ell\) est appelée limite de la suite \((u_n).\) On dit alors que \((u_n)\) a pour limite \(\ell\) ou \((u_n)\) tend vers \(\ell\) et on écrit
\begin{equation*}
\begin{split}
\lim_{n \rightarrow +\infty}u_n = \ell
\end{split}
\end{equation*}
\sphinxAtStartPar
ou \(u_n\rightarrow \ell,\) quand \(n\rightarrow+\infty\)
\end{sphinxadmonition}

\sphinxAtStartPar
Intuitivement, \(\lim_{n \rightarrow +\infty}u_n = \ell\) signifie que les termes de la suite \((u_n)\) se rapprochent de \(l\) toujours plus de \(\varepsilon\) lorsque l’indice \(n\) augmente indéfiniment.

\begin{sphinxadmonition}{note}{Définition}

\sphinxAtStartPar
Soit \(u_n\) une suite réelle.
\begin{enumerate}
\sphinxsetlistlabels{\arabic}{enumi}{enumii}{}{.}%
\item {} 
\sphinxAtStartPar
On dit que \((u_n)\) tend vers \(+\infty\) (et on écrit \(\lim_{n \rightarrow +\infty}u_n =+\infty\)) lorsque pour tout \(A>0,\) il existe \(N\in \mathbb{N}\) tel que pour tout entier \(n>N,\) on a

\end{enumerate}
\begin{equation*}
\begin{split}
u_n>A
\end{split}
\end{equation*}\begin{enumerate}
\sphinxsetlistlabels{\arabic}{enumi}{enumii}{}{.}%
\item {} 
\sphinxAtStartPar
On dit que \((u_n)\) tend vers \(-\infty\) (et on écrit \(\lim_{n \rightarrow +\infty}u_n =-\infty\)) lorsque pour tout \(A<0,\) il existe \(N\in \mathbb{N}\) tel que pour tout entier \(n>N,\) on a

\end{enumerate}
\begin{equation*}
\begin{split}
u_n<A
\end{split}
\end{equation*}\end{sphinxadmonition}

\begin{sphinxadmonition}{note}{Remarque}

\sphinxAtStartPar
Une suite \((u_n)\) est dite convergente si elle admet une limite finie. Dans le cas contraire, elle est divergente.
\end{sphinxadmonition}

\begin{sphinxadmonition}{note}{Proposition}

\sphinxAtStartPar
Si une suite est convergente alors sa limite est unique.
\end{sphinxadmonition}

\begin{sphinxadmonition}{note}{Preuve}

\sphinxAtStartPar
On procède par l’absurde. Soit \((u_n)_{n\in \mathbb{N}}\) une suite convergente ayant deux limites \(\ell\) et \(\ell',\) \(\ell\neq \ell'.\)  Choisissons \(\varepsilon>0\) tel que \(\varepsilon<\dfrac{|\ell-\ell'|}{2}\)

\sphinxAtStartPar
Comme \(\lim_{n \rightarrow +\infty}u_n=\ell,\) il existe \(N_1\) tel que \(n\geq N_1\) implique \(|u_n-\ell|<\varepsilon.\)

\sphinxAtStartPar
De même, \(\lim_{n \rightarrow +\infty}u_n=\ell',\) il existe \(N_2\) tel que \(n\geq N_2\) implique \(|u_n-\ell'|<\varepsilon.\)

\sphinxAtStartPar
Notons \(N=\max(N_1, N_2)\) on a alors pour ce \(N :\)
\begin{equation*}
\begin{split}
|u_N-\ell|<\varepsilon\qquad\mbox{et}\qquad|u_N-\ell'|<\varepsilon
\end{split}
\end{equation*}
\sphinxAtStartPar
Donc \(|\ell-\ell'|=|\ell-u_N+u_N-\ell'|\leqslant |\ell-u_N|+|u_N-\ell'|\)  d’après l’inégalité triangulaire. On en tire \(|\ell-\ell'|\leqslant\varepsilon+\varepsilon=2\varepsilon<|\ell-\ell'|,\) ce qui est impossible, d’où la contradiction.
\end{sphinxadmonition}

\begin{sphinxadmonition}{note}{Exemple}
\begin{enumerate}
\sphinxsetlistlabels{\arabic}{enumi}{enumii}{}{.}%
\item {} 
\sphinxAtStartPar
On a
\begin{equation*}
\begin{split}
    \lim_{n \rightarrow +\infty} \dfrac{1}{\sqrt{n}}=0
    \end{split}
\end{equation*}
\sphinxAtStartPar
La suite \((\dfrac{1}{\sqrt{n}})\) est alors une suite convergente et elle converge vers 0.

\item {} 
\sphinxAtStartPar
On a
\begin{equation*}
\begin{split}
    \lim_{n \rightarrow +\infty} 1-n^2=-\infty
    \end{split}
\end{equation*}
\sphinxAtStartPar
La suite \((1-n^2)\) est donc une suite divergente.

\end{enumerate}
\end{sphinxadmonition}

\begin{sphinxadmonition}{note}{Proposition}

\sphinxAtStartPar
Soit \(q\in \mathbb{R}^{*}.\) On considère la suite \(u_n=q^n,\:n\in \mathbb{N}.\) On a:
\begin{enumerate}
\sphinxsetlistlabels{\arabic}{enumi}{enumii}{}{.}%
\item {} 
\sphinxAtStartPar
Si \(q\leq-1:\) La suite \((u_n)\) est divergente et n’admet pas de limite, ni finie ni infinie.

\item {} 
\sphinxAtStartPar
Si \(-1<q<1:\) La suite \((u_n)_n\) est convergente et \(\lim_{n \rightarrow +\infty}u_n=0\)

\item {} 
\sphinxAtStartPar
Si \(q>1:\) La suite \((u_n)_n\) est divergente et \(\lim_{n \rightarrow +\infty}u_n=+\infty\)

\end{enumerate}
\end{sphinxadmonition}

\begin{sphinxadmonition}{note}{Proposition}

\sphinxAtStartPar
Soit \((u_n)\) une suite réelle et soit \(\ell\in \mathbb{R}.\) On a
\begin{equation*}
\begin{split}
\lim_{n \rightarrow +\infty}u_n=\ell\Leftrightarrow \lim_{n \rightarrow +\infty}|u_n-\ell|=0.
\end{split}
\end{equation*}\end{sphinxadmonition}

\begin{sphinxadmonition}{note}{Exemple}

\sphinxAtStartPar
Soit \((v_n)_{\geq1}\) la suite définie par \(v_n=\dfrac{2n-1}{n}.\) On a
\begin{equation*}
\begin{split}
\lim_{n \rightarrow +\infty}|v_n-2|=\lim_{n \rightarrow +\infty}|-\dfrac{1}{n}|=\lim_{n \rightarrow +\infty}\dfrac{1}{n}=0
\end{split}
\end{equation*}
\sphinxAtStartPar
En on déduit que \(\lim_{n \rightarrow +\infty}v_n=2.\)
\end{sphinxadmonition}

\begin{sphinxadmonition}{note}{Proposition (Opérations sur les limites des suites)}

\sphinxAtStartPar
Soient \((u_n)\) et \((v_n)\) deux suites et soient \(\alpha\) et \(\beta\) dans \(\mathbb{R},\)
\begin{enumerate}
\sphinxsetlistlabels{\arabic}{enumi}{enumii}{}{.}%
\item {} 
\sphinxAtStartPar
Si \(\lim_{n \rightarrow +\infty} u_n=\alpha\) et \(\lim_{n \rightarrow +\infty} v_n=\beta,\) alors
\begin{equation*}
\begin{split}
    \lim_{n \rightarrow +\infty}(u_n+v_n)=\alpha+\beta\quad\mbox{et}\quad \lim_{n \rightarrow +\infty}(u_n\times v_n)=\alpha\times \beta
    \end{split}
\end{equation*}
\item {} 
\sphinxAtStartPar
Si \(\lim_{n \rightarrow +\infty} u_n=\alpha\) alors pour tout \(\lambda\in \mathbb{R}\) on a
\begin{equation*}
\begin{split}
    \lim_{n \rightarrow +\infty} (\lambda u_n)=\lambda\alpha
    \end{split}
\end{equation*}
\item {} 
\sphinxAtStartPar
Si \(\lim_{n \rightarrow +\infty} u_n=\alpha\) et \(\alpha\neq0\) alors on a \(\lim_{n \rightarrow +\infty}\dfrac{1}{u_n}=\dfrac{1}{\alpha}.\)

\item {} 
\sphinxAtStartPar
Si \(\lim_{n \rightarrow +\infty} u_n=\pm\infty\) alors on a \(\lim_{n \rightarrow +\infty}\dfrac{1}{u_n}=0,\)

\item {} 
\sphinxAtStartPar
Si \(\lim_{n \rightarrow +\infty} u_n=\pm\infty\) et \(\lim_{n \rightarrow +\infty} v_n=\alpha,\) avec \(\alpha\in \mathbb{R}\) alors \(\lim_{n \rightarrow +\infty} (u_n+v_n)=\lim_{n \rightarrow +\infty} u_n.\)

\item {} 
\sphinxAtStartPar
Si \(\lim_{n \rightarrow +\infty} u_n=\pm\infty\) et \(\lim_{n \rightarrow +\infty} v_n=\alpha,\) avec \(\alpha\in \mathbb{R}^{+}\) alors \(\lim_{n \rightarrow +\infty} (u_n \times v_n)=\)\textbackslash{}lim\_\{n \textbackslash{}rightarrow +\textbackslash{}infty\} u\_n.\$

\item {} 
\sphinxAtStartPar
Si \(\lim_{n \rightarrow +\infty} u_n=+\infty\) et \(\lim_{n \rightarrow +\infty} v_n=+\infty,\) alors \(\lim_{n \rightarrow +\infty} (u_n+v_n)=+\infty\) et \(\lim_{n \rightarrow +\infty}(u_n\times v_n)=+\infty.\)

\item {} 
\sphinxAtStartPar
Si \(\lim_{n \rightarrow +\infty} u_n=-\infty\) et \(\lim_{n \rightarrow +\infty} v_n=-\infty,\) alors \(\lim_{n \rightarrow +\infty} (u_n+v_n)=-\infty\) et \(\lim_{n \rightarrow +\infty} (u_n\times v_n)=+\infty.\)

\item {} 
\sphinxAtStartPar
Si \(\lim_{n \rightarrow +\infty} u_n=+\infty\) et \(\lim_{n \rightarrow +\infty} v_n=-\infty,\) alors \(\lim_{n \rightarrow +\infty} (u_n\times v_n)=-\infty.\)

\end{enumerate}
\end{sphinxadmonition}

\begin{sphinxadmonition}{note}{Exemple}
\begin{enumerate}
\sphinxsetlistlabels{\arabic}{enumi}{enumii}{}{.}%
\item {} 
\sphinxAtStartPar
On a \(\lim_{n \rightarrow +\infty}(\dfrac{-2}{\sqrt{n}-3}+3n+1)=+\infty,\) puisque \(\lim_{n \rightarrow +\infty}(\dfrac{-2}{\sqrt{n}-3}) =0\) et \(\lim_{n \rightarrow +\infty}(3n+1)=+\infty.\)

\item {} 
\sphinxAtStartPar
On a \(\lim_{n \rightarrow +\infty}\dfrac{\dfrac{2}{3n}-5}{1+n+e^n}=0,\) puisque \(\lim_{n \rightarrow +\infty}\dfrac{2}{3n}-5=-5\) et \(\lim_{n \rightarrow +\infty}1+n+e^n=+\infty.\)

\end{enumerate}
\end{sphinxadmonition}

\sphinxAtStartPar
Parfois on tombe sur l’une des quatres « \sphinxstylestrong{formes indéterminées} » suivantes \(+\infty-\infty,\;0\times \pm\infty,\;\dfrac{\pm\infty}{\pm\infty},\;\dfrac{0}{0}.\)

\sphinxAtStartPar
\sphinxstylestrong{Limites usuelles utiles}
\begin{itemize}
\item {} 
\sphinxAtStartPar
\(\lim_{n \rightarrow +\infty} n\sin(\dfrac{1}{n})=\lim_{n \rightarrow +\infty}\dfrac{\sin(\dfrac{1}{n})}{\dfrac{1}{n}}=1,\)

\item {} 
\sphinxAtStartPar
\(\lim_{n \rightarrow +\infty}n^2\cos (1-\dfrac{1}{n})=\lim_{n \rightarrow +\infty}\dfrac{\cos(1-\dfrac{1}{n})}{(\dfrac{1}{n})^2}=\dfrac{1}{2},\)

\item {} 
\sphinxAtStartPar
\(\lim_{n \rightarrow +\infty} n\ln(1+\dfrac{1}{n})=\lim_{n \rightarrow +\infty}\dfrac{\ln(1+\dfrac{1}{n})}{\dfrac{1}{n}}=1,\)

\item {} 
\sphinxAtStartPar
\(\lim_{n \rightarrow +\infty} n(e^{\dfrac{1}{n}})=1,\)

\item {} 
\sphinxAtStartPar
\(\lim_{n \rightarrow +\infty} \dfrac{(\ln (n))^{\alpha}}{n}=0,\)

\item {} 
\sphinxAtStartPar
\(\lim_{n \rightarrow +\infty} n^{\alpha}e^{-n}=0.\)

\end{itemize}

\begin{sphinxadmonition}{note}{Proposition}
\begin{enumerate}
\sphinxsetlistlabels{\arabic}{enumi}{enumii}{}{.}%
\item {} 
\sphinxAtStartPar
Soient \((u_n)\) et \((v_n)\) deux suites convergentes telles que \(\forall n\in \mathbb{N}:\, u_n \leq v_n\) (ou \(u_n < v_n\)). Alors
\begin{equation*}
\begin{split}
    \lim_{n \rightarrow +\infty} u_n \leq \lim_{n \rightarrow +\infty} v_n
    \end{split}
\end{equation*}
\item {} 
\sphinxAtStartPar
Soient \((u_n)\) et \((v_n)\) deux suites telles que \(\forall n\in \mathbb{N}:\,u_n\leq v_n\) et \(\lim_{n \rightarrow +\infty} u_n=+\infty.\) Alors
\begin{equation*}
\begin{split}
    \lim_{n \rightarrow +\infty} v_n=+\infty
    \end{split}
\end{equation*}
\item {} 
\sphinxAtStartPar
Soient \((u_n)\) et \((v_n)\) deux suites telles que \(\forall n\in \mathbb{N}:\,u_n\leq v_n\) et \(\lim_{n \rightarrow +\infty} v_n=-\infty.\) Alors
\begin{equation*}
\begin{split}
    \lim_{n \rightarrow +\infty} u_n=-\infty
    \end{split}
\end{equation*}
\end{enumerate}
\end{sphinxadmonition}

\begin{sphinxadmonition}{note}{Théorème (Théorème des « gendarmes »)}

\sphinxAtStartPar
Si \((u_n),\) \((v_n)\) et \((w_n)\) sont trois suites telles que
\begin{equation*}
\begin{split}
\forall n\in \mathbb{N}:\; u_n\leq w_n\leq v_n
\end{split}
\end{equation*}
\sphinxAtStartPar
et si \(\lim_{n \rightarrow +\infty}u_n=\lim_{n \rightarrow +\infty}v_n=\ell,\) alors la suite \((w_n)\) est convergente. De plus on a
\begin{equation*}
\begin{split}
\lim_{n \rightarrow +\infty} w_n=\ell
\end{split}
\end{equation*}\end{sphinxadmonition}

\begin{sphinxadmonition}{note}{Exemple}

\sphinxAtStartPar
Calculer, à l’aide de Théorème des « gendarmes », la limite suivante
\begin{equation*}
\begin{split}
\lim_{n \rightarrow +\infty}(1+\dfrac{\sin (n)}{n})
\end{split}
\end{equation*}\end{sphinxadmonition}

\sphinxAtStartPar
Toute suite convergente est bornée. La réciproque est fausse: la suite \(((-1)^n)\) est bornée mais elle diverge (elle n’admet pas de limite).En revanche, on a le résultat suivant.

\begin{sphinxadmonition}{note}{Théorème}
\begin{enumerate}
\sphinxsetlistlabels{\arabic}{enumi}{enumii}{}{.}%
\item {} 
\sphinxAtStartPar
Toute suite croissante et majorée est convergente.

\item {} 
\sphinxAtStartPar
Toute suite décroissante et minorée est convergente.

\end{enumerate}
\end{sphinxadmonition}

\begin{sphinxadmonition}{note}{Preuve}

\sphinxAtStartPar
On montrera l’assertion « Toute suite convergente est bornée. » En effet, soit \((u_n)_{n\in \mathbb{N}}\) une suite convergeant vers le réel \(\ell.\) En appliquant la définition de limite avec \(\varepsilon=1,\)  on obtient qu’il existe un entier naturel \(N\) tel que pour \(n\geq N\) on ait, \(|u_n-\ell|\leq1,\) et donc pour \(n\geq N\) on a
\begin{equation*}
\begin{split}
|u_n|=|\ell+(u_n-\ell)|\leq |\ell|+|u_n-\ell|\leq |\ell|+1
\end{split}
\end{equation*}
\sphinxAtStartPar
Donc si on pose \(M=\max(|u_0|,|u_1|,...,|u_{N-1}|,|\ell|+1)\)

\sphinxAtStartPar
on a alors \(\forall n\in \mathbb{N},\,|u_n|\leq M.\)
\end{sphinxadmonition}


\section{Suites adjacentes}
\label{\detokenize{suites:suites-adjacentes}}
\begin{sphinxadmonition}{note}{Définition}

\sphinxAtStartPar
Deux suites \((u_n)\) et \((v_n)\) sont dites adjacentes si \((u_n)\) est celle qui est croissante, l’autre est décroissante et leur différente et leur différence tend vers 0.
\end{sphinxadmonition}

\begin{sphinxadmonition}{note}{Remarque}

\sphinxAtStartPar
Si \((u_n)\) et \((v_n)\) sont deux suites adjacentes et si \((u_n)\) est celle qui croissante (donc \((v_n)\) est décroissante) alors
\begin{equation*}
\begin{split}
\forall n\in \mathbb{N}:\; u_n\leq v_n
\end{split}
\end{equation*}\end{sphinxadmonition}

\sphinxAtStartPar
Pratiquement, pour montrer que deux suites \((u_n)\) et \((v_n)\) sont adjacentes, on commence par chercher celle qui est plus grande que l’autre. On montre alors que la plus grande est décroissante et que l’autre (la plus petite) est croissante puis on montre que la différence des deux suites converge vers 0.

\begin{sphinxadmonition}{note}{Exemple}

\sphinxAtStartPar
Les deux suites \((1+\dfrac{1}{n})_{n>0}\) et \((1-\dfrac{1}{n})_{n>0}\) sont adjacentes.
\end{sphinxadmonition}

\begin{sphinxadmonition}{note}{Théorème}

\sphinxAtStartPar
Deux suites adjacentes sont convergentes et convergent vers la même limite.
\end{sphinxadmonition}

\begin{sphinxadmonition}{note}{Exemple}

\sphinxAtStartPar
On pose \(u_n=\sum_{k=1}^{n}\dfrac{1}{k^2},\quad n\in \mathbb{N}^{*}.\)

\sphinxAtStartPar
L’objectif de cet exemple est de montrer que la suite \((u_n)_{n\geq1}\) est une suite convergente. Soit \((v_n)_{n\geq1}\) la suite définie par \(v_n=u_n+\dfrac{2}{n+1},\quad n\geq1.\)

\sphinxAtStartPar
Montrons que les deux suites \((u_n)_{n\geq1}\) \((v_n)_{n\geq1}\) sont adjacentes.
\begin{itemize}
\item {} 
\sphinxAtStartPar
Remarquons tout d’abord que \(u_n\leq v_n,\;n\geq1.\) Montrons alors que \((u_n)_{n\geq1}\) est croissante et que \((v_n)_{n\geq1}\) est décroissante.

\end{itemize}

\sphinxAtStartPar
On a \(\forall n\geq 1:\quad u_{n+1}-u_n=\dfrac{1}{(n+1)^2}.\)

\sphinxAtStartPar
Donc \((u_n)_{n\geq1}\) est croissante. Et pour tout \(n\geq1\) on a
\begin{eqnarray*}
v_{n+1}-v_n
&=& u_{n+1}+\dfrac{2}{n+2}-u_n-\dfrac{2}{n+1}\\
&=& \dfrac{1}{(n+1)^2}+\dfrac{2}{n+2}-\dfrac{2}{n+1}\\
&=& \dfrac{-n}{(n+1)^2 (n+2)}.
\end{eqnarray*}
\sphinxAtStartPar
Donc \(v_{n+1}-v_n\leq0,\;n\geq1.\) Ainsi, la suite \((v_n)_{n\geq1}\) est décroissante.
\begin{itemize}
\item {} 
\sphinxAtStartPar
On a \(\lim_{n \rightarrow +\infty}(v_n-u_n)=\lim_{n \rightarrow +\infty}\dfrac{2}{n+1}=0.\)

\end{itemize}

\sphinxAtStartPar
En on déduit les deux suites \((u_n)_{n\geq1}\) \((v_n)_{n\geq1}\) sont adjacentes. Donc \((u_n)_{n\geq1}\) et \((v_n)_{n\geq1}\) convergent vers la même limite. En particulier, la suite \((u_n)_{n\geq1}\) est convergente.
\end{sphinxadmonition}


\section{Suites récurrentes}
\label{\detokenize{suites:suites-recurrentes}}
\begin{sphinxadmonition}{note}{Définition}

\sphinxAtStartPar
Une suite \((u_n)_{n\geq n_0}\) est dite récurrente si elle est définie par son premier terme \(u_{n_0}\) et une relation de récurrence de forme
\begin{equation*}
\begin{split}
u_{n+1}=f(u_n),\quad\forall n\geq n_0
\end{split}
\end{equation*}
\sphinxAtStartPar
où \(f\) est une fonction définie sur un intervalle \(I\) de \(\mathbb{R}.\)
\end{sphinxadmonition}


\subsection{Exemples remarquables de suites récurrentes}
\label{\detokenize{suites:exemples-remarquables-de-suites-recurrentes}}
\sphinxAtStartPar
\sphinxstylestrong{Suites arithmétiques:}

\begin{sphinxadmonition}{note}{Définition}

\sphinxAtStartPar
Une suite \((u_n)_{n\geq n_0}\) est dite \sphinxstylestrong{arithmétique} si elle est définie par son premier terme \(u_{n_0}\) et
\begin{equation*}
\begin{split}
\exists r\in \mathbb{R},\;\forall n\geq n_0:\quad u_{n+1}=u_n +r
\end{split}
\end{equation*}
\sphinxAtStartPar
Dans ce cas, le réel \(r\) est appelé la raison de la suite \((u_n)_{n\geq n_0}.\)
\end{sphinxadmonition}

\begin{sphinxadmonition}{note}{Proposition}

\sphinxAtStartPar
Si \((u_n)_{n\geq n_0}\) est une suite arithmétique de raison \(r,\) alors \(u_n=u_{n_0}+(n-n_0)r,\, \forall n\geq n_0.\) En particulier, on a
\begin{itemize}
\item {} 
\sphinxAtStartPar
si \(r>0\):
\begin{equation*}
\begin{split}
    \lim_{n \rightarrow +\infty}u_n=+\infty
    \end{split}
\end{equation*}
\item {} 
\sphinxAtStartPar
si \(r<0\):
\begin{equation*}
\begin{split}
    \lim_{n \rightarrow +\infty}u_n=-\infty
    \end{split}
\end{equation*}
\item {} 
\sphinxAtStartPar
si \(r=0\):
\begin{equation*}
\begin{split}
    \lim_{n \rightarrow +\infty}u_n=u_0
    \end{split}
\end{equation*}
\end{itemize}
\end{sphinxadmonition}

\begin{sphinxadmonition}{note}{Proposition}

\sphinxAtStartPar
Soit \((u_n)_{n\geq n_0}\) une suite arithmétique de raison \(r\),(\(r\neq0\)) et de premier terme \(u_{n_0},\) la somme des termes successifs de la suite \((u_n)\) s’exprime par la formule suivante:
\begin{equation*}
\begin{split}
u_p+u_{p+1}+...+u_n=(n-p+1)\dfrac{u_p+u_n}{2}=(n-p+1)\dfrac{2u_p+(n-p)r}{2}\;n_0\leq p\leq n
\end{split}
\end{equation*}\end{sphinxadmonition}

\sphinxAtStartPar
\sphinxstylestrong{Suites géométriques:}

\begin{sphinxadmonition}{note}{Définition}

\sphinxAtStartPar
Une suite \((u_n)_{n\geq n_0}\) est dite \sphinxstylestrong{géométrique} si elle est définie par son premier terme \(u_{n_0}\) et
\begin{equation*}
\begin{split}
\exists q\in \mathbb{R},\;\forall n\geq n_0:\quad u_{n+1}=qu_n
\end{split}
\end{equation*}
\sphinxAtStartPar
Dans ce cas, le réel \(q\) est appelé la raison de la suite \((u_n)_{n\geq n_0}.\)
\end{sphinxadmonition}

\begin{sphinxadmonition}{note}{Proposition}

\sphinxAtStartPar
Si \((u_n)_{n\geq n_0}\) est une suite géométrique non nulle de raison \(q(q\neq0),\) alors \(u_n=u_{n_0}q^{n-n_0},\;\forall n\geq n_0.\) Et en particulier, on a
\begin{enumerate}
\sphinxsetlistlabels{\arabic}{enumi}{enumii}{}{.}%
\item {} 
\sphinxAtStartPar
Si \(q>1,\) alors \(\lim_{n \rightarrow +\infty}u_n=\pm\infty,\)

\item {} 
\sphinxAtStartPar
Si \(-1<q<1,\) alors \(\lim_{n \rightarrow +\infty}u_n=0,\)

\item {} 
\sphinxAtStartPar
Si \(q\leq-1,\) la suite \((u_n)_{n\geq n_0}\) diverge.

\end{enumerate}
\end{sphinxadmonition}

\begin{sphinxadmonition}{note}{Proposition(Série géométrique)}

\sphinxAtStartPar
Si \((u_n)_{n\geq n_0}\) est une suite géométrique de raison \(q(q\neq1),\) alors
\begin{equation*}
\begin{split}
\sum_{k=n_0}^{n}u_k= u_{u_0}\dfrac{1-q^{n-n_0+1}}{1-q},\quad\forall n\geq n_0
\end{split}
\end{equation*}\end{sphinxadmonition}

\begin{sphinxadmonition}{note}{Exemple}

\sphinxAtStartPar
Soit \((u_n)\) la suite définie par \(u_0=1\) et \(u_{n+1}=\dfrac{1}{2}u_n, \; n\in \mathbb{N}.\) Donc \((u_n)\) est une suite géométrique de raison \(q=\dfrac{1}{2}.\) Le terme général de cette suite est \(u_n=\dfrac{1}{2}^n,\,n\in \mathbb{N}.\) Ainsi puisque \(-1<\dfrac{1}{2}<1,\) alors \(\lim_{n \rightarrow +\infty}u_n=0.\)

\sphinxAtStartPar
Pour \(n\in \mathbb{N},\) on pose \(S_n=\sum_{k=0}^{n}u_k.\) On a alors
\begin{equation*}
\begin{split}
S_n=\dfrac{1-\dfrac{1}{2^{n+1}}}{1-\dfrac{1}{2}}=2(1-\dfrac{1}{2^{n+1}})
\end{split}
\end{equation*}
\sphinxAtStartPar
Autrement dit \(1+\dfrac{1}{2}+\dfrac{1}{2^2}+...+\dfrac{1}{2^n}=2(1-\dfrac{1}{2^{n+1}}).\)

\sphinxAtStartPar
D’autre part, on a
\begin{equation*}
\begin{split}
\lim_{n \rightarrow +\infty}u_n=2
\end{split}
\end{equation*}
\sphinxAtStartPar
On écrit alors \(\sum_{k=0}^{+\infty}u_k=2\)
\end{sphinxadmonition}


\subsection{Convergence d’une suite récurrente}
\label{\detokenize{suites:convergence-d-une-suite-recurrente}}
\sphinxAtStartPar
Dans toute la suite, soit \((u_n)_{n\geq n_0}\) une suite récurrente définie par son premier terme et par la relation \(u_{n+1}=f(u_n),\,n\geq n_0,\) où \(f\) est une fonction numérique donnée.

\begin{sphinxadmonition}{note}{Théorème}

\sphinxAtStartPar
Soit \(f\) une fonction continue sur un intervalle \(I\) avec \(f(I)\subset I.\) Si \(u_{n_0}\in I\) et si la suite récurrente \((u_n)\) est convergente, alors la limite \(\ell\) de cette suite est une solution de l’équation \(f(x)=x.\)
\end{sphinxadmonition}

\sphinxAtStartPar
Le théorème précédent impose sur la suite \((u_n)_{n\geq n_0}\) d’être convergente. Et si les autres conditions du théorème sont satisfaites, alors la limite de \((u_n)_{n\geq n_0}\) est parmi les solutions de l’équation \(f(x)=x.\) En ajoutant une condition supplémentaire sur la fonction \(f,\) la convergence de \((u_n)_{n\geq n_0}\) sera alors assurée:

\begin{sphinxadmonition}{note}{Théorème}

\sphinxAtStartPar
Supposons que \(f:[a,b]\rightarrow [a,b]\) une fonction continue et croissante et supposons que \(u_{n_0}\in [a,b].\) Alors la suite récurrente \((u_n)\) est monotone et converge vers \(\ell\in [a,b]\) vérifiant \(f(\ell)=\ell.\)
\end{sphinxadmonition}

\begin{sphinxadmonition}{note}{Remarque}

\sphinxAtStartPar
La monotonie de \((u_n)_{n\geq n_0}\) dans le théorème précédent s’obtienne en comparant seulement les deux premiers termes de la suite \((u_n)_{n\geq n_0}:\)
\begin{itemize}
\item {} 
\sphinxAtStartPar
Si \(u_{n_0}\leq u_{n_0+1}\) alors \((u_n)_{n\geq n_0}\) est croissante.

\item {} 
\sphinxAtStartPar
Si \(u_{n_0}\geq u_{n_0+1}\) alors \((u_n)_{n\geq n_0}\) est décroissante.

\end{itemize}
\end{sphinxadmonition}

\begin{sphinxadmonition}{note}{Exemple}

\sphinxAtStartPar
Soit \((u_n)_{n\in \mathbb{N}}\) la suite définie par: \(u_0=\dfrac{1}{3}\) et \(u_{n+1}=u_n-u_{n}^{2}.\)

\sphinxAtStartPar
On a \(u_{n+1}=f(u_n),\) avec \(f(x)=x-x^2.\) La fonction \(f\) est continue sur \([0,\dfrac{1}{2}]\) et on peut vérifier que \(f\) est croissante sur cet intervalle et que \(f([0,\dfrac{1}{2}])\subset [0,\dfrac{1}{2}].\)

\sphinxAtStartPar
On en déduit que \((u_n)_n\) est monotone: on a \(u_0=\dfrac{1}{3}\) et \(u_1=2/9,\) donc \(u_0> u_1.\) Ainsi \((u_n)_n\) est décroissante. De plus \((u_n)_n\) converge vers \(\ell\in [0,\dfrac{1}{2}]\) vérifiant \(f(\ell)=\ell.\) Or \(f(\ell)=\ell\Leftrightarrow \ell-\ell^2=\ell\Leftrightarrow -\ell^2=0\Leftrightarrow \ell=0.\)

\sphinxAtStartPar
Ainsi, on a \(\lim_{n \rightarrow +\infty}u_n=0.\)
\end{sphinxadmonition}


\section{Exercices}
\label{\detokenize{exo2:exercices}}\label{\detokenize{exo2::doc}}

\subsection{Exercice 1}
\label{\detokenize{exo2:exercice-1}}
\sphinxAtStartPar
Montrer que toute suite convergente est bornée.


\subsection{Exercice 2}
\label{\detokenize{exo2:exercice-2}}
\sphinxAtStartPar
Montrer qu’une suite d’entiers qui converge est constante à partir d’un certain rang.


\subsection{Exercice 3}
\label{\detokenize{exo2:exercice-3}}
\sphinxAtStartPar
Les suites suivantes sont\sphinxhyphen{}elles croissantes? décroissantes?
\begin{equation*}
\begin{split}
u_n = n^2 + 5n +4
\end{split}
\end{equation*}\begin{equation*}
\begin{split}
v_n =\dfrac{-2n+3}{n+1}
\end{split}
\end{equation*}\begin{equation*}
\begin{split}
w_n = \sqrt{2n+5}
\end{split}
\end{equation*}\begin{equation*}
\begin{split}
a_n = \dfrac{2^n}{n}
\end{split}
\end{equation*}

\subsection{Exercice 4}
\label{\detokenize{exo2:exercice-4}}
\sphinxAtStartPar
Montrer que la suite \((u_n)_{n\in\mathbb{N}}\) définie par
\begin{equation*}
\begin{split}
u_n = (-1)^n + \dfrac{1}{n}
\end{split}
\end{equation*}
\sphinxAtStartPar
n’est pas convergente.


\subsection{Exercice 5}
\label{\detokenize{exo2:exercice-5}}
\sphinxAtStartPar
Étudier la nature des suites suivantes, et déterminer leur limite éventuelle :
\begin{enumerate}
\sphinxsetlistlabels{\arabic}{enumi}{enumii}{}{.}%
\item {} 
\sphinxAtStartPar
la suite
\begin{equation*}
\begin{split}
    \dfrac{sin(n) + 3cos(n^2)}{\sqrt{n}}
    \end{split}
\end{equation*}
\item {} 
\sphinxAtStartPar
la suite
\begin{equation*}
\begin{split}
    \dfrac{2n + (-1)^n}{5n + (-1)^{n+1}}
    \end{split}
\end{equation*}
\item {} 
\sphinxAtStartPar
la suite
\begin{equation*}
\begin{split}
    \dfrac{n^3 + 5n}{4n^2 + sin(n) + ln(n)}
    \end{split}
\end{equation*}
\end{enumerate}


\subsection{Exercice 6}
\label{\detokenize{exo2:exercice-6}}
\sphinxAtStartPar
Étudier la nature des suites suivantes, et déterminer leur limite éventuelle :
\begin{enumerate}
\sphinxsetlistlabels{\arabic}{enumi}{enumii}{}{.}%
\item {} 
\sphinxAtStartPar
la suite
\begin{equation*}
\begin{split}
    3^n e^{-3n}
    \end{split}
\end{equation*}
\item {} 
\sphinxAtStartPar
la suite
\begin{equation*}
\begin{split}
    \sqrt{2n+1} - \sqrt{2n-1}
    \end{split}
\end{equation*}
\item {} 
\sphinxAtStartPar
la suite
\begin{equation*}
\begin{split}
    \dfrac{ln(n+e^n)}{n}
    \end{split}
\end{equation*}
\item {} 
\sphinxAtStartPar
la suite
\begin{equation*}
\begin{split}
    \dfrac{ln(1+\sqrt{n})}{1+n^2}
    \end{split}
\end{equation*}
\end{enumerate}


\subsection{Exercice 7}
\label{\detokenize{exo2:exercice-7}}\begin{enumerate}
\sphinxsetlistlabels{\arabic}{enumi}{enumii}{}{.}%
\item {} 
\sphinxAtStartPar
Déterminer deux réels \(a\) et \(b\) tels que:
\begin{equation*}
\begin{split}
    \dfrac{1}{k^2 - 1} = \dfrac{a}{k+1}+\dfrac{b}{k-1}
    \end{split}
\end{equation*}
\item {} 
\sphinxAtStartPar
En déduire la limite de la suite:
\begin{equation*}
\begin{split}
    \sum_{k=2}^{n}\dfrac{1}{k^2 - 1}
    \end{split}
\end{equation*}
\end{enumerate}


\subsection{Exercice 8}
\label{\detokenize{exo2:exercice-8}}
\sphinxAtStartPar
Montrer que, pour tout \(n\in \mathbb{N}^∗\), on a:
\begin{equation*}
\begin{split}
\sqrt{n+1}-\sqrt{n} \leq \dfrac{1}{2\sqrt{n}}
\end{split}
\end{equation*}
\sphinxAtStartPar
En déduire le comportement de la suite \((u_n)\) définie par:
\begin{equation*}
\begin{split}
u_n = 1 + \dfrac{1}{\sqrt{2}}+ \ldots + \dfrac{1}{\sqrt{n}}
\end{split}
\end{equation*}

\subsection{Exercice 9}
\label{\detokenize{exo2:exercice-9}}
\sphinxAtStartPar
Soit \((u_n)_{n\geq 1}\) la suite définie par:
\begin{equation*}
\begin{split}
u_n = \sum_{k=1}^{n} \dfrac{1}{k^2}
\end{split}
\end{equation*}\begin{itemize}
\item {} 
\sphinxAtStartPar
Démontrer que la suite \((u_n)\) est croissante.

\item {} 
\sphinxAtStartPar
Démontrer que, pour tout \(n \geq 1\)
\begin{equation*}
\begin{split}
    \dfrac{1}{(n+1)^2} \leq \dfrac{1}{n} - \dfrac{1}{n+1}
    \end{split}
\end{equation*}
\item {} 
\sphinxAtStartPar
Démontrer que, pour tout \(n \geq 1\), \(u_n \leq 2−\dfrac{1}{n}\).

\item {} 
\sphinxAtStartPar
En déduire que la suite \((u_n)\) est convergente.

\end{itemize}


\chapter{Fonctions numériques, limite et continuité}
\label{\detokenize{limitefct:fonctions-numeriques-limite-et-continuite}}\label{\detokenize{limitefct::doc}}
\sphinxAtStartPar
Le présent Chapitre contiendra :
\begin{enumerate}
\sphinxsetlistlabels{\arabic}{enumi}{enumii}{}{.}%
\item {} 
\sphinxAtStartPar
Fonctions numérique: Définitions générales

\item {} 
\sphinxAtStartPar
Limite d’une fonction numérique

\item {} 
\sphinxAtStartPar
Limite en un point en l’infini

\item {} 
\sphinxAtStartPar
Limite à gauche et à droite en un point

\item {} 
\sphinxAtStartPar
Propriétés des limites

\item {} 
\sphinxAtStartPar
Continuité d’une fonction numérique

\item {} 
\sphinxAtStartPar
Prolongement par continuité

\item {} 
\sphinxAtStartPar
Théorème des valeurs intermédiaires

\item {} 
\sphinxAtStartPar
Théorème de la bijection

\item {} 
\sphinxAtStartPar
Application: Fonctions Logarithme et exponentielle

\item {} 
\sphinxAtStartPar
Exercices

\end{enumerate}


\section{Fonctions numérique: Définitions générales}
\label{\detokenize{limitefcts:fonctions-numerique-definitions-generales}}\label{\detokenize{limitefcts::doc}}
\begin{sphinxadmonition}{note}{Définition}

\sphinxAtStartPar
Une fonction d’une variable réelle à valeurs réelles est une application \(f:D\rightarrow\mathbb{R},\) où \(D\) est une partie de \(\mathbb{R}.\) En général, \(D\) est un intervalle ou une réunion d’intervalles. On appelle \(D\) le domaine de définition de la fonction \(f\) et on le note souvent par \(D_f\) et on a
\begin{equation*}
\begin{split}
D_f:=\{x\in \mathbb{R};\;f(x)\in \mathbb{R}\}
\end{split}
\end{equation*}\end{sphinxadmonition}

\begin{sphinxadmonition}{note}{Exemple}

\sphinxAtStartPar
Le domaine de définition de la fonction \(f(x)=\dfrac{\sqrt{x}}{x-2}\) est \(D_f=[0,2[\cup]2,+\infty[.\)
\end{sphinxadmonition}

\sphinxAtStartPar
Soient \(f:D\rightarrow \mathbb{R}\) et \(g:D\rightarrow\mathbb{R}\) deux fonctions définies sur une même partie \(D\) de \(\mathbb{R}.\)
\begin{itemize}
\item {} 
\sphinxAtStartPar
La somme de \(f\) et \(g\) est la fonction \(f+g:D\rightarrow \mathbb{R}\) définie par \((f+g)(x)=f(x)+g(x)\) pour tout \(x\in D.\)

\item {} 
\sphinxAtStartPar
Le produit de \(f\) et \(g\) est la fonction \(f\times g:D\rightarrow \mathbb{R}\) définie par \((f\times g)(x)=f(x)\times g(x)\) pour tout \(x\in D.\)

\item {} 
\sphinxAtStartPar
La multiplication par un scalaire \(\lambda\in \mathbb{R}\) de \(f\) est la fonction \(\lambda.f:D\rightarrow \mathbb{R}\) définie par \((\lambda.f)(x)=\lambda.f(x)\) pour tout \(x\in D.\)

\end{itemize}

\begin{sphinxadmonition}{note}{Définition}

\sphinxAtStartPar
Soit \(f:D\rightarrow \mathbb{R}\) une fonction. On dit que:
\begin{itemize}
\item {} 
\sphinxAtStartPar
\(f\) est majorée sur \(D\) si \(\exists M\in \mathbb{R},\;\forall x\in D;\; f(x)\leq M;\)

\item {} 
\sphinxAtStartPar
\(f\) est minorée sur \(D\) si \(\exists m\in \mathbb{R},\,\forall x\in D,\, f(x)\geq m;\)

\item {} 
\sphinxAtStartPar
\(f\) est bornée sur \(D\) si \(f\) est à la fois majorée et minorée sur \(D,\) c’est\sphinxhyphen{}à\sphinxhyphen{}dire si \(\exists M\in \mathbb{R}_{+},\forall x\in D,\,|f(x)|\leq M.\)

\end{itemize}
\end{sphinxadmonition}

\begin{sphinxadmonition}{note}{Définition}

\sphinxAtStartPar
Soit \(f:D\rightarrow \mathbb{R}\) une fonction. On dit que:
\begin{itemize}
\item {} 
\sphinxAtStartPar
\(f\) est croissante sur \(D\) si: pour tout \(x\) et \(y\) dans \(D\) on a \(x\leq y\Rightarrow f(x)\leq f(y).\)

\item {} 
\sphinxAtStartPar
\(f\) est strictement croissante sur \(D\) si: pour tout \(x\) et \(y\) dans \(D\) on a \(x < y\Rightarrow f(x)< f(y)\)

\item {} 
\sphinxAtStartPar
\(f\) est décroissante sur \(D\) si: pour tout \(x\) et \(y\) dans \(D\) on a \(x\leq y\Rightarrow f(x)\geq f(y).\)

\item {} 
\sphinxAtStartPar
\(f\) est strictement décroissante sur \(D\) si: pour tout \(x\) et \(y\) dans \(D\) on a \(x < y\Rightarrow f(x)> f(y).\)

\item {} 
\sphinxAtStartPar
\(f\) est monotone (resp. strictement monotone) sur \(D\) si \(f\) est croissante ou décroissante (resp. strictement croissante ou strictement décroissante) sur \(D.\)

\end{itemize}
\end{sphinxadmonition}

\begin{sphinxadmonition}{note}{Définition}

\sphinxAtStartPar
On dit que:
\begin{itemize}
\item {} 
\sphinxAtStartPar
\(f\) est paire si \(\forall x\in D_f:\) \(-x\in D_f\) et \(\forall x\in D_f\) \(f(-x)=f(x),\)

\item {} 
\sphinxAtStartPar
\(f\) est impaire si \(\forall x\in D_f:\) \(-x\in D_f\) et \(\forall x\in D_f\) \(f(-x)=-f(x),\)

\end{itemize}
\end{sphinxadmonition}

\begin{sphinxadmonition}{note}{Définition}

\sphinxAtStartPar
Soit \(f:\mathbb{R}\rightarrow \mathbb{R}\) une fonction et \(T\) un nombre réel, \(T>0.\) La fonction \(f\) est dite périodique de période \(T\) si \(\forall x\in \mathbb{R}\) \(f(x+T)=f(x).\)
\end{sphinxadmonition}

\begin{sphinxadmonition}{note}{Exemple}

\sphinxAtStartPar
Les fonctions sinus et cosinus sont \(2\pi\)\sphinxhyphen{}périodique. La fonction tangente est \(\pi\)\sphinxhyphen{}périodique.
\end{sphinxadmonition}


\section{Limite d’une fonction numérique}
\label{\detokenize{limitefcts:limite-d-une-fonction-numerique}}

\subsection{Limite en un point en l’infini}
\label{\detokenize{limitefcts:limite-en-un-point-en-l-infini}}
\sphinxAtStartPar
Soit \(f\) une fonction définie sur un ensemble de la forme \(]a,x_0[\cup ]x_0,b[,\,x_0\in \mathbb{R}.\)

\begin{sphinxadmonition}{note}{Définition}

\sphinxAtStartPar
Soit \(l\in \mathbb{R}.\) On dit que \(f\) a pour limite en \(x_0\) si

\sphinxAtStartPar
\(\forall \varepsilon>0,\quad \exists \delta>0, \quad x\in D_f:\quad |x-x_0|<\delta\Rightarrow |f(x)-l|<\varepsilon.\)

\sphinxAtStartPar
On dit aussi que \(f(x)\) tend vers \(l\) lorsque \(x\) tend vers \(x_0.\) On note alors \(\lim_{x\rightarrow x_0}f(x)=l\) ou bien \(\lim_{x_0}f(x)=l.\)
\end{sphinxadmonition}

\begin{sphinxadmonition}{note}{Définition}

\sphinxAtStartPar
On dit que \(f\) a pour limite \(+\infty\) en \(x_0\) si

\sphinxAtStartPar
\(\forall A>0, \quad \exists \delta>0, \quad x\in I:\quad |x-x_0|<\delta\Rightarrow f(x)>A.\)

\sphinxAtStartPar
On note alors \(\lim_{x\rightarrow x_0}f(x)=+\infty.\)

\sphinxAtStartPar
On dit que \(f\) a pour limite \(-\infty\) en \(x_0\) si

\sphinxAtStartPar
\(\forall A>0, \quad \exists \delta>0, \quad x\in I:\quad |x-x_0|<\delta\Rightarrow f(x)<-A.\)

\sphinxAtStartPar
On note alors \(\lim_{x\rightarrow x_0}f(x)=-\infty.\)
\end{sphinxadmonition}

\sphinxAtStartPar
Soit \(f:I\rightarrow\mathbb{R}\) une fonction définie sur un intervalle de la forme \(I=]a,+\infty[.\)

\begin{sphinxadmonition}{note}{Définition}
\begin{itemize}
\item {} 
\sphinxAtStartPar
Soit \(l\in \mathbb{R}.\) On dit que \(f\) a pour limite \(l\) en \(+\infty\) si

\end{itemize}

\sphinxAtStartPar
\(\forall \varepsilon>0,\quad \exists B>0, \quad x\in I:\quad x> B\Rightarrow |f(x)-l|<\varepsilon.\)

\sphinxAtStartPar
On note alors \(\lim_{x\rightarrow +\infty}f(x)=l\) ou \(\lim_{+\infty}=l.\)
\begin{itemize}
\item {} 
\sphinxAtStartPar
On dit que \(f\) a pour limite \(+\infty\) en \(+\infty\) si

\end{itemize}

\sphinxAtStartPar
\(\forall A>0,\quad \exists B>0, \quad x\in I:\quad x> B\Rightarrow f(x)>A.\)

\sphinxAtStartPar
On note alors \(\lim_{x\rightarrow +\infty}f(x)=+\infty\)

\sphinxAtStartPar
On définirait de la même manière la limite en \(-\infty\) pour des fonctions définies sur les intervalles de type \(]-\infty,a[.\)
\end{sphinxadmonition}

\begin{sphinxadmonition}{note}{Exemple}

\sphinxAtStartPar
On a les limites classiques suivantes pour tout \(n\geq 1:\)
\begin{itemize}
\item {} 
\sphinxAtStartPar
\(\lim_{x\rightarrow +\infty}x^n=+\infty\) et \(\lim_{x\rightarrow-\infty}x^n=\left\{
\begin{array}{ll}
+\infty \quad\mbox{si} \;n\;\mbox{est pair}\\
-\infty \quad\mbox{si}\;n\;\mbox{est impair}
\end{array}
\right.\)

\item {} 
\sphinxAtStartPar
\(\lim_{x\rightarrow \pm\infty}(\dfrac{1}{x^n})=0\)

\item {} 
\sphinxAtStartPar
Soient \(P(x)=a_n x^n+a_{n-1}x^{n-1}+...+a_1 x+a_0\) et \(Q(x)=b_m x^m+b_{m-1}^{m-1}+...+b_1 x+b_0\) deux polynômes (\(a_n\neq0\) et \(b_m\neq 0\)). On a

\end{itemize}

\sphinxAtStartPar
\(\lim_{x\rightarrow\pm\infty}P(x)=\lim_{x\rightarrow\pm\infty}a_n x^n\) et \(\lim_{x\rightarrow\pm\infty}\dfrac{P(x)}{Q(x)}=\left\{
\begin{array}{ll}
\pm\infty \quad\mbox{ si } \; n>m\\
\dfrac{a_n}{b_m} \quad\mbox{si}\;n=m\\
0 \quad\mbox{si} n<m
\end{array}
\right.\)
\begin{itemize}
\item {} 
\sphinxAtStartPar
Les fonctions sin et cos n’admettent pas de limite ni en \(+\infty\) ni en \(-\infty.\)

\end{itemize}
\end{sphinxadmonition}


\subsection{Limite à gauche et à droite en un point}
\label{\detokenize{limitefcts:limite-a-gauche-et-a-droite-en-un-point}}
\begin{sphinxadmonition}{note}{Définition}

\sphinxAtStartPar
Soit \(f\) une fonction définie sur un intervalle \(]x_0,b[\) (resp. \(]a,x_0[\)). On dit que \(f\) admet une limite \(\ell\in \mathbb{R}\) à droite (resp. à gauche) en \(x_0\) si

\sphinxAtStartPar
\(\forall \varepsilon>0,\quad \exists\delta>0,\quad \forall x\in D_f:\quad x_0< x<x_0+\delta\) (resp. \(x_0-\delta< x< x_0)\Rightarrow |f(x)-l|<\varepsilon.\)

\sphinxAtStartPar
On écrit alors \(\lim\limits_{\substack{x \rightarrow x_0 \\ x>x_0}} f(x)=l\) (resp. \(\lim\limits_{\substack{x \rightarrow x_0 \\ x<x_0}} f(x))=l.\)

\sphinxAtStartPar
On note aussi \(\lim\limits_{\substack{x_{0}^{+}}}f\) pour la limite à droite et \(\lim\limits_{\substack{x_{0}^{-}}}f\) pour la limite à gauche.
\end{sphinxadmonition}

\begin{sphinxadmonition}{note}{Proposition}

\sphinxAtStartPar
Soit \(\ell\in \mathbb{R}.\) On a \(\lim\limits_{\substack{x \rightarrow x_0}}f(x)=\ell\Leftrightarrow \lim\limits_{\substack{x \rightarrow x_{0}^{+}}}f=\lim\limits_{\substack{x \rightarrow x_{0}^{-}}}f=\ell.\)
\end{sphinxadmonition}

\begin{sphinxadmonition}{note}{Exemple}

\sphinxAtStartPar
Considérons la fonction \(f(x)=\left\{
\begin{array}{ll}
x-2 \quad\mbox{si} \;x\geq 1,\\
x+1 \quad\mbox{si}\;x<1.
\end{array}
\right.\)

\sphinxAtStartPar
On a \(\lim\limits_{\substack{x \rightarrow 1^{-}}}f(x)=\lim\limits_{\substack{x \rightarrow 1^{-}}}(x+1)=2\) et \(\lim\limits_{\substack{x \rightarrow 1^{+}}}f(x)=\lim\limits_{\substack{x \rightarrow 1^{+}}}(x-2)=-1.\)

\sphinxAtStartPar
Comme \(\lim\limits_{\substack{x \rightarrow 1^{-}}}f(x)\neq\lim\limits_{\substack{x \rightarrow 1^{+}}}f(x),\) on en déduit que \(f\) n’a pas de limite en 1.
\end{sphinxadmonition}


\subsection{Propriétés}
\label{\detokenize{limitefcts:proprietes}}
\begin{sphinxadmonition}{note}{Proposition}

\sphinxAtStartPar
Si une fonction admet une limite, alors cette limite est unique.
\end{sphinxadmonition}

\sphinxAtStartPar
Soient deux fonctions \(f\) et \(g.\) On suppose que \(x_0\) est un réel, ou que \(x_0=\pm\infty.\)

\begin{sphinxadmonition}{note}{Proposition}

\sphinxAtStartPar
Si \(\lim\limits_{\substack{x_{0}}}f=\ell\in \mathbb{R}\) et \(\lim\limits_{\substack{x_{0}}}g=\ell'\in \mathbb{R},\) alors:
\begin{itemize}
\item {} 
\sphinxAtStartPar
\(\lim\limits_{\substack{x_{0}}}(\lambda. f)=\lambda.\ell\) pour tout \(\lambda\in \mathbb{R}\)

\item {} 
\sphinxAtStartPar
\(\lim\limits_{\substack{x_{0}}}(f+g)=\ell+\ell'\)

\item {} 
\sphinxAtStartPar
\(\lim\limits_{\substack{x_{0}}}(f\times g)=\ell\times \ell'\)

\item {} 
\sphinxAtStartPar
Si \(\ell\neq0,\) alors \(\lim\limits_{\substack{x_{0}}}\dfrac{1}{f}=\dfrac{1}{\ell}\)

\item {} 
\sphinxAtStartPar
Si \(\lim\limits_{\substack{x_{0}}}f=\pm\infty,\) alors \(\lim\limits_{\substack{x_{0}}}\dfrac{1}{f}=0.\)

\end{itemize}
\end{sphinxadmonition}

\sphinxAtStartPar
Voici une liste de formes indeterminées:
\begin{equation*}
\begin{split}
+\infty-\infty,\quad0\times\infty,\quad\dfrac{\infty}{\infty},\quad\dfrac{0}{0},\quad 1^{\infty},...
\end{split}
\end{equation*}
\begin{sphinxadmonition}{note}{Proposition}
\begin{itemize}
\item {} 
\sphinxAtStartPar
Si \(f\leq g\) et si \(\lim\limits_{\substack{x_{0}}}f=\ell\in \mathbb{R}\) et \(\lim\limits_{\substack{x_{0}}}g=\ell'\in \mathbb{R},\) alors: \(\ell\leqslant\ell'.\)

\item {} 
\sphinxAtStartPar
Si \(f\leq g\) et si \(\lim\limits_{\substack{x_{0}}}f=+\infty,\) alors \(\lim\limits_{\substack{x_{0}}}g=+\infty.\)

\item {} 
\sphinxAtStartPar
Si \(f\leq g\leq h\) et si \(\lim\limits_{\substack{x_{0}}}f=\lim\limits_{\substack{x_{0}}}h=\ell\in \mathbb{R},\) alors \(g\) a une limite en \(x_0\) et \(\lim\limits_{\substack{x_{0}}}g=\ell.\)

\end{itemize}
\end{sphinxadmonition}


\section{Continuité d’une fonction numérique}
\label{\detokenize{limitefcts:continuite-d-une-fonction-numerique}}
\begin{sphinxadmonition}{note}{Définition}
\begin{itemize}
\item {} 
\sphinxAtStartPar
Soit \(f\) une fonction définie en un voisinage de \(x_0.\) On dit que \(f\) est continue en \(x_0\) si \(\lim\limits_{\substack{x\rightarrow x_{0}}}f(x)=f(x_0).\)

\item {} 
\sphinxAtStartPar
Soit \(f\) une fonction définie en un intervalle de type \(]x_0-\varepsilon, x_0].\) On dit que \(f\) est continue à droite en \(x_0\) si \(\lim\limits_{\substack{x\rightarrow x_{0}^{+}}}f(x)=f(x_0).\)

\item {} 
\sphinxAtStartPar
Soit \(f\) une fonction définie en un intervalle de type \([x_0,x_0+\varepsilon[.\) On dit que \(f\) est continue à gauche en \(x_0\) si \(\lim\limits_{\substack{x\rightarrow x_{0}^{+}}}f(x)=f(x_0).\)

\end{itemize}
\end{sphinxadmonition}

\begin{sphinxadmonition}{note}{Définition}

\sphinxAtStartPar
Soit \(f\) une fonction définie en un voisinage de \(x_0.\) On a
\(f\) est continue en \(x_0\Leftrightarrow\lim\limits_{\substack{x\rightarrow x_{0}^{+}}}f(x)=\lim\limits_{\substack{x\rightarrow x_{0}^{-}}}f(x)=f(x_0)\)
\end{sphinxadmonition}

\begin{sphinxadmonition}{note}{Définition}
\begin{itemize}
\item {} 
\sphinxAtStartPar
Une fonction \(f\) est dite continue sur un intervalle ouvert \(I\subset D_f\) si elle est continue en tout point de \(I.\)

\item {} 
\sphinxAtStartPar
Une fonction \(f\) est dite continue sur un intervalle \([a,b]\subset D_f\) si elle est continue sur \(]a,b[\) et continue à droite en \(a\) et à gauche en \(b.\)

\item {} 
\sphinxAtStartPar
Une fonction \(f\) est dite continue sur un intervalle \([a,b[\subset D_f\) si elle est continue sur \(]a,b[\) et continue à droite en \(a.\)

\end{itemize}

\sphinxAtStartPar
De même, on définit la continuité d’une fonction sur un intervalle \(]a,b],\) sur \(]-\infty,a],\) sur \([a,+\infty[,\)…
\end{sphinxadmonition}

\begin{sphinxadmonition}{note}{Exemple}
\begin{itemize}
\item {} 
\sphinxAtStartPar
La fonction racine carrée \(x\mapsto \sqrt{x}\) est continue sur \([0,+\infty[.\)

\item {} 
\sphinxAtStartPar
Les fonctions sin et cos sont continues sur \(\mathbb{R}.\)

\item {} 
\sphinxAtStartPar
La fonction valeur absolue \(x\mapsto |x|\) est continue sur \(\mathbb{R}.\)

\end{itemize}
\end{sphinxadmonition}

\begin{sphinxadmonition}{note}{Proposition}

\sphinxAtStartPar
Soient \(f,\,g:I\rightarrow\mathbb{R}\) deux fonctions continues en un point \(x_0\in I.\) Alors
\begin{itemize}
\item {} 
\sphinxAtStartPar
\(\lambda.f\) est continue en \(x_0\) (pour tout \(\lambda\in \mathbb{R}\)),

\item {} 
\sphinxAtStartPar
\(f+g\) est continue en \(x_0.\)

\item {} 
\sphinxAtStartPar
\(f\times g\) est continue en \(x_0.\)

\item {} 
\sphinxAtStartPar
Si \(f(x_0)\neq0,\) alors \(\dfrac{1}{f}\) est continue en \(x_0.\)

\end{itemize}
\end{sphinxadmonition}

\begin{sphinxadmonition}{note}{Exemple}

\sphinxAtStartPar
Les polynômes sont continues sur \(\mathbb{R}.\)

\sphinxAtStartPar
Toute fraction rationnelle \(x\mapsto\dfrac{P(x)}{Q(x)}\) est continue sur son domaine de définition.
\end{sphinxadmonition}

\sphinxAtStartPar
La composition conserve la continuité (mais il faut faire attention en quels points les hypothèse s’appliquent)

\begin{sphinxadmonition}{note}{Proposition}

\sphinxAtStartPar
Soient \(f:I\rightarrow\mathbb{R}\) et \(g:J\rightarrow\mathbb{R}\) deux fonctions telles que \(f(I)\subset J.\) Si \(f\) est continue en un point \(x_0\in I\) et si \(g\) est continue en \(f(x_0),\) alors \(g\circ f\) est continue en \(x_0.\)
\end{sphinxadmonition}


\section{Prolongement par continuité}
\label{\detokenize{limitefcts:prolongement-par-continuite}}
\begin{sphinxadmonition}{note}{Définition}

\sphinxAtStartPar
Soit \(I\) un intervalle, \(x_0\) un point de \(I\) et \(f:I\setminus\{x_0\}\rightarrow\mathbb{R}\) une fonction.
\begin{itemize}
\item {} 
\sphinxAtStartPar
On dit que \(f\) est prolongeable par continuité en \(x_0\) si \(f\) admet une limite finie en \(x_0.\) Notons alors \(\ell=\lim\limits_{\substack{x_{0}}}f.\)

\item {} 
\sphinxAtStartPar
On définit alors la fonction \(\widetilde{f}:I\rightarrow\mathbb{R}\) en posant pour tout \(x\in I\)

\end{itemize}

\sphinxAtStartPar
\(\widetilde{f}(x)=\left\{
\begin{array}{ll}
f(x) \quad\mbox{si} \;x\neq x_0\\
\ell \quad\mbox{si}\;x=x_0.
\end{array}
\right.\)

\sphinxAtStartPar
Alors \(\widetilde{f}\) est continue en \(x_0\) et on l’appelle le prolongement par continuité de \(f\) en \(x_0.\)
\end{sphinxadmonition}

\begin{sphinxadmonition}{note}{Exemple}

\sphinxAtStartPar
Considérons la fonction \(f\) définie sur \(\mathbb{R}-\{1\}\) par \(f(x)=\dfrac{x^2+x-2}{x-1}.\) Etudions le prolongement par continuité de \(f\) en 1. On a
\begin{equation*}
\begin{split}
\lim\limits_{\substack{x\rightarrow1}}f(x)=\lim\limits_{\substack{x\rightarrow1}}\dfrac{(x-1)(x+2)}{x-1}=\lim\limits_{\substack{x\rightarrow1}} x+2=3
\end{split}
\end{equation*}
\sphinxAtStartPar
Donc \(f\) est prolongeable par continuité en 1. Le prolongement par continuité de \(f\) en 1 est donc
\(\widetilde{f}(x)=\left\{
\begin{array}{ll}
f(x) \quad\mbox{si} \;x\neq 1\\
3 \quad\mbox{si}\;x=1.
\end{array}
\right.\)
\end{sphinxadmonition}


\section{Théorème des valeurs intermédiaires}
\label{\detokenize{limitefcts:theoreme-des-valeurs-intermediaires}}
\begin{sphinxadmonition}{note}{Théorème (Théorème des valeurs intermédiaires)}

\sphinxAtStartPar
Soit \(f:[a,b]\rightarrow\mathbb{R}\) une fonction continue sur un segment. Pour tout réel \(y\) compris entre \(f(a)\) et \(f(b),\) il existe \(c\in [a,b]\) tel que \(f(c)=y.\)
\end{sphinxadmonition}

\sphinxAtStartPar
Voici la version la plus utilisée du théorème des valeurs intermédiaires.

\begin{sphinxadmonition}{note}{Corollaire}

\sphinxAtStartPar
Soit \(f:[a,b]\rightarrow\mathbb{R}\) une fonction continue sur un segment. Si \(f(a).f(b)<0,\) alors il existe \(c\in ]a,b[\) tel que \(f(c)=0.\)
\end{sphinxadmonition}

\begin{sphinxadmonition}{note}{Exemple}

\sphinxAtStartPar
Tout polynôme de degré impair possède au moins une racine réelle. En effet, un tel polynôme s’écrit \(P(x)=a_n x^n+...+a_1 x+a_0\) avec \(n\) un entier impair. On peut supposer que le coefficient \(a_n\) est strictement positif. Alors on a \(\lim\limits_{\substack{x\rightarrow-\infty}}P(x)=-\infty\) et \(\lim\limits_{\substack{x\rightarrow+\infty}}P(x)=+\infty.\) En particulier, il existe deux réel \(a\) et \(b\) tels que \(P(a)<0\) et \(P(b)>0\) et on conclut grâce au corollaire précédent qu’il existe au moins \(c\in \mathbb{R}\) tel que \(P(c)=0.\)
\end{sphinxadmonition}

\sphinxAtStartPar
Voici une formulation théorique du théorème des valeurs intermédiaires

\begin{sphinxadmonition}{note}{Corollaire}

\sphinxAtStartPar
Soit \(f:I\rightarrow\mathbb{R}\) une fonction continue sur un intervalle \(I.\) Alors \(f(I)\) est un intervalle.
\end{sphinxadmonition}

\sphinxAtStartPar
Attention! Il serait faux de croire que l’image par une fonction \(f\) de l’intervalle \([a,b]\) soit l’intervalle \([f(a), f(b)]\). Cependant, on a le théorème suivant

\begin{sphinxadmonition}{note}{Théorème}

\sphinxAtStartPar
Soit \(f:[a,b]\rightarrow\mathbb{R}\) une fonction continue sur un segment. Alors il existe deux réels \(m\) et \(M\) tels que \(f([a,b])=[m,M].\) Autrement dit, l’image d’un segment par une fonction continue est un segment.
\end{sphinxadmonition}

\sphinxAtStartPar
Comme on sait déjà par le théorème des valeurs intermédiaires que \(f([a,b])\) est un intervalle, le théorème précédent signifie que si \(f\) est continue sur \([a,b],\) alors \(f\) est bornée sur \([a,b],\) et elle atteint ses bornes: \(m\) est le minimum de la fonction sur l’intervalle \([a,b]\) alors que \(M\) est bon maximum sur \([a,b].\)


\section{Théorème de la bijection}
\label{\detokenize{limitefcts:theoreme-de-la-bijection}}
\begin{sphinxadmonition}{note}{Définition}

\sphinxAtStartPar
Soit \(f:E\rightarrow F\) une fonction, où \(E\) et \(F\) sont des parties de \(\mathbb{R}.\)
\begin{itemize}
\item {} 
\sphinxAtStartPar
\(f\) est injective \(\forall x,x'\in E,\; f(x)=f(x')\Rightarrow x=x';\)

\item {} 
\sphinxAtStartPar
\(f\) est surjective si \(\forall y\in F,\exists x\in E,\;y=f(x);\)

\item {} 
\sphinxAtStartPar
\(f\) est bijective si \(f\) est à la fois injective et surjective, c’est\sphinxhyphen{}à\sphinxhyphen{}dire si \(\forall y\in F,\,\exists! x\in E:\;y=f(x).\)

\end{itemize}
\end{sphinxadmonition}

\begin{sphinxadmonition}{note}{Proposition}

\sphinxAtStartPar
Si \(f:E\rightarrow F\) est une fonction bijective alors il existe une unique application \(g:F\rightarrow E\) telle que \(g\circ f=\mbox{Id}_E\) et \(f\circ g=\mbox{Id}_F.\) La fonction \(g\) est la bijection réciproque de \(f\) et se note \(f^{-1}.\)

\sphinxAtStartPar
\(\diamond\) On rappelle que l’identité, Id\(_E:E\rightarrow E\) est simplement définie par \(x\mapsto x.\)

\sphinxAtStartPar
\(\diamond\) \(g\circ f=\)Id\(_E\) se reformule ainsi: \(\forall x\in E,\quad g(f(x))=x.\)

\sphinxAtStartPar
\(\diamond\) Alors que \(f\circ g=\mbox{Id}_F\) s’écrit: \(\forall y\in F\quad f(g(y))=y.\)
\end{sphinxadmonition}

\sphinxAtStartPar
Le théorème suivant est un outil très utile dans la pratique pour montrer qu’une fonction est bijective.

\begin{sphinxadmonition}{note}{Théorème (Théorème de la bijection)}

\sphinxAtStartPar
Soit \(f:I\rightarrow\mathbb{R}\) une fonction définie sur un intervalle \(I\) de \(\mathbb{R}.\) Si \(f\) est continue et strictement monotone sur \(I,\) alors
\begin{enumerate}
\sphinxsetlistlabels{\arabic}{enumi}{enumii}{}{.}%
\item {} 
\sphinxAtStartPar
\(f\) établit une bijection de l’intervalle \(I\) dans l’intervalle \(J=f(I),\)

\item {} 
\sphinxAtStartPar
La fonction réciproque \(f^{-1}:J\rightarrow I\) est continue et strictement monotone sur \(J\) et elle a le même sens de variation que \(f.\)

\end{enumerate}
\end{sphinxadmonition}

\begin{sphinxadmonition}{note}{Exemple}

\sphinxAtStartPar
Considérons la fonction carrée définie sur \(\mathbb{R}\) par \(f(x)=x^2.\) La fonction \(f\) est continue et est strictement croissante sur \(I=[0,+\infty[,\) donc \(f\) établit une bijection de \(I\) dans \(J=f([0,+\infty[)=[0,+\infty[.\) Déterminons sa fonction réciproque: Soit \(x\) et \(y\) dans \(\mathbb{R}_{+},\) on a
\begin{equation*}
\begin{split}
f^{-1}(x)=y\Leftrightarrow x=f(y)\Leftrightarrow x=y^2 \Leftrightarrow y=\sqrt{x}
\end{split}
\end{equation*}
\sphinxAtStartPar
Donc \(f^{-1}(x)=\sqrt{x},\) pour tout \(x\in \mathbb{R}_{+}.\)
\end{sphinxadmonition}

\sphinxAtStartPar
Généralisons en partie l’exemple précédent;

\begin{sphinxadmonition}{note}{Exemple}

\sphinxAtStartPar
Soit \(n\geq 1.\) Soit \(f:[0,+\infty[\rightarrow [0,+\infty[\) définie par \(f(x)=x^n.\) On a \(f\) est continue et strictement croissante. Donc \(f\) admet sur \(I=[0,+\infty[\) une fonction réciproqie \(f^{-1}\) définie sur \(J=f([0,+\infty[)=[0,+\infty[.\)

\sphinxAtStartPar
\(f^{-1}\) est notée: \(x\mapsto x^{\frac{1}{n}}\) ou aussi \(x\mapsto \sqrt[n]{x};\) c’est la fonction racine \(n-\)ième. Elle est continue et strictement croissante sur \([0,+\infty[.\)
\end{sphinxadmonition}


\section{Application: Fonctions Logarithme et exponentielle}
\label{\detokenize{limitefcts:application-fonctions-logarithme-et-exponentielle}}
\begin{sphinxadmonition}{note}{Proposition}

\sphinxAtStartPar
Il existe une unique fonction, notée \(\ln:]0,+\infty[\rightarrow\mathbb{R}\) telle que:
\begin{equation*}
\begin{split}
\ln'(x)=\dfrac{1}{x}\quad\mbox{pour tout}\,x>0\qquad\mbox{et}\qquad \ln(1)=0
\end{split}
\end{equation*}
\sphinxAtStartPar
De plus, cette fonction vérifie (pour tout \(a,b>0\)):
\begin{enumerate}
\sphinxsetlistlabels{\arabic}{enumi}{enumii}{}{.}%
\item {} 
\sphinxAtStartPar
\(\ln(a\times b)=\ln(a)+\ln(b).\)

\item {} 
\sphinxAtStartPar
\(\ln(\dfrac{1}{a})=-\ln a,\)

\item {} 
\sphinxAtStartPar
\(\ln (a^n)=n\ln(a),\) (pour tout \(n\in \mathbb{N}\)).

\item {} 
\sphinxAtStartPar
\(\ln\) est une fonction continue, strictement croissante et définit une bijection de \(]0,+\infty[\) sur \(\mathbb{R}.\)

\item {} 
\sphinxAtStartPar
\(\lim\limits_{\substack{x\rightarrow0}}\dfrac{\ln(1+x)}{x}=1.\)

\end{enumerate}
\end{sphinxadmonition}

\begin{sphinxadmonition}{note}{Définition}

\sphinxAtStartPar
La bijection réciproque de \(\ln:]0,+\infty[\rightarrow\mathbb{R}\) s’appelle la fonction exponentielle, notée exp\(:\mathbb{R}\rightarrow]0,+\infty[.\)
\end{sphinxadmonition}

\begin{sphinxadmonition}{note}{Proposition}

\sphinxAtStartPar
La fonction exponentielle vérifie les propriétés suivantes:
\begin{enumerate}
\sphinxsetlistlabels{\arabic}{enumi}{enumii}{}{.}%
\item {} 
\sphinxAtStartPar
exp\((\ln (x))=x\) pour tout \(x>0\) et \(\ln(\mbox{exp}(x))=\) pour tout \(x\in \mathbb{R}\)

\item {} 
\sphinxAtStartPar
exp\((a+b)=\mbox{exp}(a)\times \mbox{exp}(b)\)

\item {} 
\sphinxAtStartPar
exp\((nx)=(\mbox{exp}(x))^n\)

\item {} 
\sphinxAtStartPar
exp\(:\mathbb{R}\rightarrow]0,+\infty[\) est une fonction continue, strictement croissante vérifiant \(\lim\limits_{\substack{x\rightarrow-\infty}}\mbox{exp}(x)=0\) et \(\lim\limits_{\substack{x\rightarrow+\infty}}\mbox{exp}(x)=+\infty\)

\item {} 
\sphinxAtStartPar
La fonction exponentielle est dérivable et \(\mbox{exp}'(x)=\mbox{exp}(x),\) pour tout \(x\in \mathbb{R}.\)

\end{enumerate}
\end{sphinxadmonition}

\begin{sphinxadmonition}{note}{Remarque}

\sphinxAtStartPar
La fonction exponentielle est l’unique fonction qui vérifie exp\('(x)=\)exp\((x)\) (pour tout \(x\in \mathbb{R}\)) et exp\((1)=e,\) où \(e\simeq2.718...\) est le nombre qui vérifie \(\ln (e)=1.\)

\sphinxAtStartPar
Par définition, pour \(a>0\) et \(b\in \mathbb{R},\,\) \(a^b=\mbox{exp}(b\ln (a))\)
\end{sphinxadmonition}

\begin{sphinxadmonition}{note}{Remarque}

\sphinxAtStartPar
\(\sqrt{a}=a^{\dfrac{1}{2}}=\mbox{exp}(\dfrac{1}{2}\ln (a))\)

\sphinxAtStartPar
\(\sqrt[n]{a}=a^{\dfrac{1}{n}}=\mbox{exp}(\dfrac{1}{n}\ln (a))\) ( la racine \(n\)\sphinxhyphen{}ième de \(a\))

\sphinxAtStartPar
On note aussi \(\mbox{exp}(x)\) par \(e^x\) ce qui se justifie par le calcul: \(e^x=\mbox{exp}(x\ln (e))=\mbox{exp}(x).\)
\end{sphinxadmonition}

\sphinxAtStartPar
Les fonctions \(x\mapsto a^x\) s’appellent aussi des fonctions exponentielles et se ramènent systématiquement à la fonction exponentielle classique par l’égalité \(a^x=\mbox{exp}(x\ln (a)).\) Il ne faut surtout pas les confondre avec les fonctions puissances \(x\mapsto x^a.\) On a les propriété suivantes;

\begin{sphinxadmonition}{note}{Proposition}

\sphinxAtStartPar
Soient \(x,y>0\) et \(a,\,b\in \mathbb{R}.\)
\begin{enumerate}
\sphinxsetlistlabels{\arabic}{enumi}{enumii}{}{.}%
\item {} 
\sphinxAtStartPar
\(x^{a+b}=x^a x^b\)

\item {} 
\sphinxAtStartPar
\(x^{-a}=\dfrac{1}{x^a}\)

\item {} 
\sphinxAtStartPar
\((xy)^a=x^a y^a\)

\item {} 
\sphinxAtStartPar
\((x^a)^b=x^{ab}\)

\item {} 
\sphinxAtStartPar
\(\ln(x^a)=a\ln (x)\)

\end{enumerate}
\end{sphinxadmonition}

\sphinxAtStartPar
Comparons les fonctions \(\ln (x),\,\mbox{exp}(x)\,\) avec \(x:\)

\begin{sphinxadmonition}{note}{Proposition}
\begin{equation*}
\begin{split}
\lim\limits_{\substack{x\rightarrow+\infty}}\dfrac{\ln (x)}{x}=0\qquad \mbox{et}\quad \lim\limits_{\substack{x\rightarrow+\infty}}\dfrac{\mbox{exp}x}{x}=+\infty
\end{split}
\end{equation*}\end{sphinxadmonition}


\section{Exercices}
\label{\detokenize{exo3:exercices}}\label{\detokenize{exo3::doc}}

\subsection{Exercice 1}
\label{\detokenize{exo3:exercice-1}}
\sphinxAtStartPar
Déterminer les limites suivantes :
\begin{equation*}
\begin{split}
\lim_{x\to 1} \dfrac{\sqrt{x^2+3}+1}{2x-1}
\end{split}
\end{equation*}\begin{equation*}
\begin{split}
\lim_{x\to +\infty} 2x^3+x^2-x+4
\end{split}
\end{equation*}\begin{equation*}
\begin{split}
\lim_{x\to +\infty} \dfrac{2x+5x^2-7x^4}{x-10x^2 + 14x^3}
\end{split}
\end{equation*}\begin{equation*}
\begin{split}
\lim_{x\to -\infty} \dfrac{3x+8x^2-2x^5}{x^2+2x^6}
\end{split}
\end{equation*}\begin{equation*}
\begin{split}
\lim_{x\to +\infty} \sqrt{x^2+x} -x
\end{split}
\end{equation*}\begin{equation*}
\begin{split}
\lim_{x\to \frac{\pi}{4}}\dfrac{\tan (x) - 1}{x- \frac{\pi}{4}}
\end{split}
\end{equation*}

\subsection{Exercice 2}
\label{\detokenize{exo3:exercice-2}}
\sphinxAtStartPar
Soit \(f\) la fonction definit par:
\begin{equation*}
\begin{split}
f(x) = \dfrac{(x+1)^2}{|x^2-1|}
\end{split}
\end{equation*}\begin{itemize}
\item {} 
\sphinxAtStartPar
Determiner le domaine de definition de \(f\).

\item {} 
\sphinxAtStartPar
Etudier la limite de \(f\) en \(x_0 = -1\).

\end{itemize}


\subsection{Exercice 3}
\label{\detokenize{exo3:exercice-3}}
\sphinxAtStartPar
Les fonctions suivantes sont\sphinxhyphen{}elles prolongeables par continuité sur \(\mathbb{R}\)?
\begin{equation*}
\begin{split}
f(x) = \sin(x)\sin(\dfrac{1}{x})
\end{split}
\end{equation*}\begin{equation*}
\begin{split}
g(x) = \dfrac{1}{x}\ln(\dfrac{e^x + e^{-x}}{2})
\end{split}
\end{equation*}\begin{equation*}
\begin{split}
h(x)= \dfrac{1}{1-x} - \dfrac{2}{1-x^2}
\end{split}
\end{equation*}

\subsection{Exercice 4}
\label{\detokenize{exo3:exercice-4}}
\sphinxAtStartPar
Soient \(I\) un intervalle de \(\mathbb{R}\) et \(f  : I\rightarrow \mathbb{R}\) continue, telle que pour chaque \(x \in I\), \(f(x)^2=1\). Montrer que \(f\) est constante et égale à \(1\) ou \(-1\).


\chapter{Dérivée d’une fonction numérique\sphinxhyphen{} fonctions usuelles}
\label{\detokenize{dirivf:derivee-d-une-fonction-numerique-fonctions-usuelles}}\label{\detokenize{dirivf::doc}}
\sphinxAtStartPar
Le présent Chapitre contiendra :
\begin{enumerate}
\sphinxsetlistlabels{\arabic}{enumi}{enumii}{}{.}%
\item {} 
\sphinxAtStartPar
Dérivabilité d’une fonction numérique

\item {} 
\sphinxAtStartPar
Opérations sur les fonctions dérivables

\item {} 
\sphinxAtStartPar
Quelques applications de la dérivabilité

\item {} 
\sphinxAtStartPar
Théorème de Rolle\sphinxhyphen{}Théorème des accroissements finis

\item {} 
\sphinxAtStartPar
Fonctions circulaires inverses

\item {} 
\sphinxAtStartPar
Fonctions hyperboliques et hyperboliques inverses

\item {} 
\sphinxAtStartPar
Tangente hyperbolique et son inverse

\item {} 
\sphinxAtStartPar
Exercices

\end{enumerate}


\section{Dérivabilité d’une fonction numérique}
\label{\detokenize{dirivfs:derivabilite-d-une-fonction-numerique}}\label{\detokenize{dirivfs::doc}}
\sphinxAtStartPar
Soit \(I\) un intervalle ouvert de \(\mathbb{R}\) et \(f:I\rightarrow\mathbb{R}\) une fonction. Soit \(x_0 \in I.\)

\begin{sphinxadmonition}{note}{Definition}

\sphinxAtStartPar
\(f\) est dérivable en \(x_0\) si le taux d’accroissement \(\frac{f(x)-f(x_0)}{x-x_0}\) a une limite finie lorsque \(x\) tend vers \(x_0.\) La limite s’appelle alors le nombre dérivé de \(f\) en \(x_0\) et est noté \(f'(x_0).\) Ainsi
\begin{equation*}
\begin{split}
f'(x_0)=\lim\limits_{\substack{x\rightarrow x_0}}\frac{f(x)-f(x_0)}{x-x_0}
\end{split}
\end{equation*}\end{sphinxadmonition}

\begin{sphinxadmonition}{note}{Définition}

\sphinxAtStartPar
\(f\) est dérivable sur \(I\) si \(f\) est dérivable en tout point \(x_0\in I.\) La fonction \(x\mapsto f'(x)\) est la fonction dérivée de \(f,\) elle se note \(f'\) ou \(\frac{df}{dx}.\)
\end{sphinxadmonition}

\begin{sphinxadmonition}{note}{Exemple}

\sphinxAtStartPar
La fonction définie par \(f(x)=x^2\) est dérivable en tout point \(x_0\in \mathbb{R}.\) En effet,
\begin{equation*}
\begin{split}
\frac{f(x)-f(x_0)}{x-x_0}=\frac{x^2-x_{0}^{2}}{x-x_0}=\frac{(x-x_0)(x+x_0}{x-x_0}=x+x_0\underset{x\to x_0}{\longrightarrow}2x_0
\end{split}
\end{equation*}
\sphinxAtStartPar
On a même montré que le nombre dérivé de \(f\) en \(x_0\) est \(2x_0.\) Autrement dit: \(f'(x)=2x,\) pour tout \(x\in \mathbb{R}.\)
\end{sphinxadmonition}

\sphinxAtStartPar
De même, on peut montrer que la fonction \(f(x)=x^3\) est dérivable sur \(\mathbb{R}\) et que \(f'(x)=3x^2,\,\forall x\in \mathbb{R}.\)

\begin{sphinxadmonition}{note}{Définition}
\begin{itemize}
\item {} 
\sphinxAtStartPar
Soit \(f\) une fonction définie sur un intervalle de type \([x_0,x_0+\varepsilon[.\) On dit que \(f\) est dérivable en \(x_0\) à droite si

\end{itemize}
\begin{equation*}
\begin{split}
\lim\limits_{\substack{x_{0}^{+}}}\frac{f(x)-f(x_0)}{x-x_0}=\ell\in \mathbb{R}
\end{split}
\end{equation*}
\sphinxAtStartPar
Dans ce cas le nombre \(\ell\) est appelé dérivé de \(f\) à droite en \(x_0\) et est noté par \(f'_d (x_0).\)
\begin{itemize}
\item {} 
\sphinxAtStartPar
Soit \(f\) une fonction définie sur un intervalle de type \(]x_0-\varepsilon,x_0].\) On dit que \(f\) est dérivable en \(x_0\) à gauche si

\end{itemize}
\begin{equation*}
\begin{split}
\lim\limits_{\substack{x_{0}^{-}}}\frac{f(x)-f(x_0)}{x-x_0}=\ell\in \mathbb{R}
\end{split}
\end{equation*}
\sphinxAtStartPar
Dans ce cas le nombre \(\ell\) est appelé dérivé de \(f\) à gauche en \(x_0\) et est noté par \(f'_g (x_0).\)
\end{sphinxadmonition}

\begin{sphinxadmonition}{note}{Exemple}

\sphinxAtStartPar
La fonction \(f(x)=\sqrt{x}\) n’est pas dérivable à droite en \(x_0=0.\) En effet, on a
\begin{equation*}
\begin{split}
\lim\limits_{\substack{x \rightarrow 0 \\ x>0}}\frac{f(x)-f(0)}{x-0}=\lim\limits_{\substack{x \rightarrow 0 \\ x>0}}\frac{\sqrt{x}}{x}=\lim\limits_{\substack{x \rightarrow 0 \\ x>0}}\frac{1}{\sqrt{x}}=+\infty
\end{split}
\end{equation*}\end{sphinxadmonition}

\begin{sphinxadmonition}{note}{Proposition}

\sphinxAtStartPar
\(f\) est dérivable en \(x_0\Leftrightarrow f'_d(x_0)=f'_g(x_0).\)
\end{sphinxadmonition}


\section{Opérations sur les fonctions dérivables}
\label{\detokenize{dirivfs:operations-sur-les-fonctions-derivables}}
\begin{sphinxadmonition}{note}{Proposition}

\sphinxAtStartPar
Soient \(f,g:I\rightarrow \mathbb{R}\) deux fonctions dérivables sur \(I.\) Alors pour tout \(x\in I:\)
\begin{itemize}
\item {} 
\sphinxAtStartPar
\((f+g)'(x)=f'(x)+g'(x),\)

\item {} 
\sphinxAtStartPar
\((\lambda f)'(x)=\lambda f'(x)\) où \(\lambda\) est un réel fixé,

\item {} 
\sphinxAtStartPar
\((f\times g)'(x)=f'(x)g(x)+f(x)g'(x),\)

\item {} 
\sphinxAtStartPar
\((\frac{1}{f})'(x)=-\frac{f'(x)}{(f(x))^2}\) (si \(f(x)\neq0\)),

\item {} 
\sphinxAtStartPar
\((\frac{f}{g})'(x)=\frac{f'(x)g(x)-f(x)g'(x)}{(g(x))^2}\) (si \(g(x)\neq 0\))

\end{itemize}
\end{sphinxadmonition}

\begin{sphinxadmonition}{note}{Exemple}
\begin{enumerate}
\sphinxsetlistlabels{\arabic}{enumi}{enumii}{}{.}%
\item {} 
\sphinxAtStartPar
Soit \(f:\mathbb{R}\rightarrow \mathbb{R}\) la fonction définie par \(f(x)=x+\mbox{exp}(x).\) \(f\) est dérivable sur \(\mathbb{R}\) car \(f\) est la somme de deux fonctions dérivables sur \(\mathbb{R}.\) De plus, on a \(f'(x)=1+\mbox{exp}(x),\) pour tout \(x\in \mathbb{R}.\)

\item {} 
\sphinxAtStartPar
Soit \(g\) la fonction définie sur \(\mathbb{R}-\{1/2\}\) par \(g(x)=\frac{x^2+3x-1}{2x-1}.\) La fonction \(g\) est dérivable sur son domaine de définition et, pour tout \(x\in \mathbb{R}-\{1/2\},\) on a

\end{enumerate}
\begin{equation*}
\begin{split}
g'(x)=\frac{(2x+3)(2x-1)-2(x^2+3x-1)}{(2x-1)^2}=\frac{2x^2-2x-1}{(2x-1)^2}
\end{split}
\end{equation*}\end{sphinxadmonition}

\begin{sphinxadmonition}{note}{Proposition}

\sphinxAtStartPar
Si \(f\) est dérivable en \(x\) et \(g\) est dérivable en \(f(x)\) alors \(g\circ f\) est dérivable en \(x\) de dérivée:
\begin{equation*}
\begin{split}
(g\circ f)'(x)=g'(f(x)).f'(x)
\end{split}
\end{equation*}\end{sphinxadmonition}

\begin{sphinxadmonition}{note}{Corollaire}

\sphinxAtStartPar
Soit \(I\) un intervalle ouvert. Soit \(f:I\rightarrow J\) dérivable et bijective dont on note \(f^{-1}:J\rightarrow I\) la bijection réciproque. Si \(f'\) ne s’annule pas sur \(I\) alors \(f^{-1}\) est dérivable et on a pour tout \(x\in J:\)
\begin{equation*}
\begin{split}
(f^{-1})'(x)=\frac{1}{f'(f^{-1}))}
\end{split}
\end{equation*}\end{sphinxadmonition}

\sphinxAtStartPar
Le tableau suivant résume les principales formules à connaître où \(x\) est une variable réelle;


\begin{savenotes}\sphinxattablestart
\centering
\begin{tabulary}{\linewidth}[t]{|T|T|}
\hline
\sphinxstyletheadfamily 
\sphinxAtStartPar
Fonction
&\sphinxstyletheadfamily 
\sphinxAtStartPar
Dérivée
\\
\hline
\sphinxAtStartPar
\(x^n\)
&
\sphinxAtStartPar
\(n x^{n-1}\)
\\
\hline
\sphinxAtStartPar
\(\frac{1}{x}\)
&
\sphinxAtStartPar
\(-\frac{1}{x^2}\)
\\
\hline
\sphinxAtStartPar
\(\sqrt{x}\)
&
\sphinxAtStartPar
\(\frac{1}{2}\frac{1}{\sqrt{x}}\)
\\
\hline
\sphinxAtStartPar
\(x^{\alpha}\)
&
\sphinxAtStartPar
\(\alpha x^{\alpha-1}\; (\alpha \in \mathbb{R})\)
\\
\hline
\sphinxAtStartPar
\(e^{x}\)
&
\sphinxAtStartPar
\(e^{x}\)
\\
\hline
\sphinxAtStartPar
\(\ln x\)
&
\sphinxAtStartPar
\(\frac{1}{x}\)
\\
\hline
\sphinxAtStartPar
\(\cos x\)
&
\sphinxAtStartPar
\(-\sin x\)
\\
\hline
\sphinxAtStartPar
\(\sin x\)
&
\sphinxAtStartPar
\(\cos x \)
\\
\hline
\sphinxAtStartPar
\(\tan x\)
&
\sphinxAtStartPar
\(1+\tan^2 x=\frac{1}{\cos^2 x}\)
\\
\hline
\end{tabulary}
\par
\sphinxattableend\end{savenotes}

\sphinxAtStartPar
Le tableau suivant résume les principales formules à connaître de la composée des fonctions dérivable où \(u\) est une fonction dérivable;


\begin{savenotes}\sphinxattablestart
\centering
\begin{tabulary}{\linewidth}[t]{|T|T|}
\hline
\sphinxstyletheadfamily 
\sphinxAtStartPar
Fonction
&\sphinxstyletheadfamily 
\sphinxAtStartPar
Dérivée
\\
\hline
\sphinxAtStartPar
\(u^n\)
&
\sphinxAtStartPar
\(n u' u^{n-1}\, n\in \mathbb{Z}\)
\\
\hline
\sphinxAtStartPar
\(\frac{1}{u}\)
&
\sphinxAtStartPar
\(-\frac{u'}{u^2}\)
\\
\hline
\sphinxAtStartPar
\(\sqrt{u}\)
&
\sphinxAtStartPar
\(\frac{1}{2}\frac{u'}{\sqrt{u}}\)
\\
\hline
\sphinxAtStartPar
\(u^{\alpha}\)
&
\sphinxAtStartPar
\(\alpha u' u^{\alpha-1}\; (\alpha \in \mathbb{R})\)
\\
\hline
\sphinxAtStartPar
\(e^{u}\)
&
\sphinxAtStartPar
\(u' e^{u}\)
\\
\hline
\sphinxAtStartPar
\(\ln u\)
&
\sphinxAtStartPar
\(\frac{u'}{u}\)
\\
\hline
\sphinxAtStartPar
\(\cos u\)
&
\sphinxAtStartPar
\(-u' \sin u\)
\\
\hline
\sphinxAtStartPar
\(\sin u\)
&
\sphinxAtStartPar
\(u' \cos u \)
\\
\hline
\sphinxAtStartPar
\(\tan u\)
&
\sphinxAtStartPar
\(u' (1+\tan^2 u) =\frac{u'}{\cos^2 u}\)
\\
\hline
\end{tabulary}
\par
\sphinxattableend\end{savenotes}

\begin{sphinxadmonition}{note}{Remarque}

\sphinxAtStartPar
Si vous voulez dériver une fonction avec un exposant dépendant de \(x\) il faut absolument repasser à la forme exponentielle. Par exemple si \(f(x)=2^x\) alors on réécrit d’abord \(f(x)=e^{x\ln 2}\) pour pouvoir calculer \(f'(x)=\ln 2. e^{x\ln 2}=\ln 2 . 2^x.\)
\end{sphinxadmonition}


\section{Quelques applications de la dérivabilité}
\label{\detokenize{dirivfs:quelques-applications-de-la-derivabilite}}

\subsection{Etude de sens de variation d’une fonction}
\label{\detokenize{dirivfs:etude-de-sens-de-variation-d-une-fonction}}
\begin{sphinxadmonition}{note}{Proposition (Dérivée et monotonie d’une fonction)}

\sphinxAtStartPar
Soit \(f:[a,b]\rightarrow\mathbb{R}\) une fonction continue sur \([a,b]\) et dérivable sur \(]a,b[.\)
\begin{enumerate}
\sphinxsetlistlabels{\arabic}{enumi}{enumii}{}{.}%
\item {} 
\sphinxAtStartPar
\(\forall x\in ]a,b[,\;f'(x)\geq 0 (\mbox{resp.}f'(x)>0)\Rightarrow f\) est croissante (resp. \(f\) est strictement croissante).

\item {} 
\sphinxAtStartPar
\(\forall x\in ]a,b[,\;f'(x)\leq 0 (\mbox{resp.}f'(x)<0)\Rightarrow f\) est décroissante (resp. \(f\) est strictement décroissante).

\item {} 
\sphinxAtStartPar
\(\forall x\in ]a,b[,\;f'(x)=0\Leftrightarrow f\) est constante.

\end{enumerate}
\end{sphinxadmonition}

\begin{sphinxadmonition}{note}{Exemple}

\sphinxAtStartPar
Soit \(g\) la fonction définie sur \(\mathbb{R}\) par \(g(x)=(1-x)e^x.\) On a
\begin{equation*}
\begin{split}
\forall x\in \mathbb{R}:\;g'(x)=-e^x+(1-x)e^x=(-1+1-x)e^x=-xe^x
\end{split}
\end{equation*}
\sphinxAtStartPar
Donc \(g'(x)=0\Leftrightarrow x=0\) et \(g'(x)>0\Leftrightarrow x<0.\) En déduit que \(g\) est strictement décroissante sur \(]0,+\infty[\) et est strictement croissante sur \(]-\infty,0[.\)
\end{sphinxadmonition}


\subsection{Etude d’extremums d’une fonction}
\label{\detokenize{dirivfs:etude-d-extremums-d-une-fonction}}
\begin{sphinxadmonition}{note}{Définition}

\sphinxAtStartPar
On dit que \(f\) admet un maximum (resp. un minimum) en un point \(x_0\in D_f\) si
\begin{equation*}
\begin{split}
\forall x\in D_f,\; f(x)\leq f(x_0)\quad (\mbox{resp.}\, f(x)\geq f(x_0))
\end{split}
\end{equation*}
\sphinxAtStartPar
On dit que \(f\) admet un maximum (resp. un minimum) local en un point \(x_0\) s’il existe un voisinage \(I\subset D_f\) de \(x_0\) tel que
\begin{equation*}
\begin{split}
\forall x\in I,\;f(x)\leq f(x_0)\quad (\mbox{resp.}\, f(x)\geq f(x_0))
\end{split}
\end{equation*}
\sphinxAtStartPar
On dit que \(f\) admet un extremum (resp. un extremum local) si \(f\) admet un maximum ou un minimum (resp. un maximum local ou un minimum local).
\end{sphinxadmonition}

\begin{sphinxadmonition}{note}{Proposition (Dérivée et extremums locaux d’une fonction)}

\sphinxAtStartPar
Soit \(f\) une fonction dérivable sur un intervalle \(I\) et soit \(x_0\) un point dans l’interieur de \(I\) tel que \(f'(x_0)=0.\)

\sphinxAtStartPar
\(\diamond\) S’il existe \(h>0\) tel que \(]x_0 -h,x_0 +h[\subset I\) avec \(f'>0\) sur \(]x_0-h,x_0[\) et \(f'<0\) sur \(]x_0,x_0+h[,\) alors \(f\) admet un maximum local en \(x_0.\)

\sphinxAtStartPar
\(\diamond\) S’il existe \(h>0\) tel que \(]x_0 -h,x_0 +h[\subset I\) avec \(f'<0\) sur \(]x_0-h,x_0[\) et \(f'>0\) sur \(]x_0,x_0+h[,\) alors \(f\) admet un minimum local en \(x_0.\)
\end{sphinxadmonition}

\begin{sphinxadmonition}{note}{Exemple}

\sphinxAtStartPar
Considérons la fonction \(g\) définie dans l’exemple précédent \(g(x)=(1-x)e^x.\) D’après le tableau de variation de \(g,\) elle admet un maximum global en \(x_0=0\) puisque
\begin{equation*}
\begin{split}
\forall x\in \mathbb{R};\; g(x)\leq g(0)
\end{split}
\end{equation*}
\sphinxAtStartPar
La fonction \(g\) n’admet pas de minimum (ni global ni local)
\end{sphinxadmonition}


\section{Théorème de Rolle\sphinxhyphen{}Théorème des accroissements finis}
\label{\detokenize{dirivfs:theoreme-de-rolle-theoreme-des-accroissements-finis}}
\begin{sphinxadmonition}{note}{Théorème(Théorème de Rolle)}

\sphinxAtStartPar
Soit \(f:[a,b]\rightarrow\mathbb{R}\) telle que

\sphinxAtStartPar
\(\diamond\) \(f\) est continue sur \([a,b],\)

\sphinxAtStartPar
\(\diamond\) \(f\) est dérivable sur \(]a,b[,\)

\sphinxAtStartPar
\(\diamond\) \(f(a)=f(b).\)

\sphinxAtStartPar
Alors il existe \(c\in ]a,b[\) tel que \(f'(c)=0.\)
\end{sphinxadmonition}

\begin{sphinxadmonition}{note}{Exemple}

\sphinxAtStartPar
Soit \(f(x)=x^3-x.\) On a \(f(-1)=f(1).\) De plus, \(f\) est continue et dérivable sur \(\mathbb{R}.\) En particulier, \(f\) est continue sur \([-1,1]\) et est dérivable sur \(]-1,1[.\) D’après le Théorème de Rolle, il existe \(c\in ]-1,1[\) tel que \(f'(c)=0.\)
\end{sphinxadmonition}

\begin{sphinxadmonition}{note}{Théorème (Théorème des accroissements finis)}

\sphinxAtStartPar
Soit \(f:[a,b]\rightarrow \mathbb{R}\) une fonction continue sur \([a,b]\) et dérivable sur \(]a,b[.\) Il existe \(c\in ]a,b[\) tel que
\begin{equation*}
\begin{split}
f(b)-f(a)=f'(c)(b-a)
\end{split}
\end{equation*}\end{sphinxadmonition}

\begin{sphinxadmonition}{note}{Corollaire (Inégalité des accroissements finis)}

\sphinxAtStartPar
Soit \(f:I\rightarrow\mathbb{R}\) une fonction dérivable sur un intervalle \(I\) ouvert. S’il existe une constante \(M\) telle que pour tout \(x\in I,\,|f'(x)|\leq M\) alors
\begin{equation*}
\begin{split}
|f(x)-f(y)|\leq M|x-y|,\;\forall x,y\in I
\end{split}
\end{equation*}\end{sphinxadmonition}

\begin{sphinxadmonition}{note}{Exemple}

\sphinxAtStartPar
En utilisant le théorème des accroissements finis, on va montrer que
\begin{equation*}
\begin{split}
x<e^x -1<xe^x,\;\forall x>0
\end{split}
\end{equation*}
\sphinxAtStartPar
Fixons donc \(x\) tel que \(x>0\) et considérons la fonction \(f(t)=e^t\) sur l’intervalle \([0,x].\) \(f\) est continue sur \([0,x]\) et est dérivable sur \(]0,x[.\) Donc, d’après le théorème des accroissements finis, il existe \(c\in ]0,x[\) tel que
\begin{equation*}
\begin{split}
f'x)-f(0)=f'(c)(x-0)
\end{split}
\end{equation*}
\sphinxAtStartPar
C\sphinxhyphen{}à\sphinxhyphen{}d: \(\exists c\in ]0,x[:\;e^x-1=xe^c,\) or, \(0<c<x\) implique \(1<e^c<e^x,\) donc puisque \(x>0,\) on a \(x<xe^c<xe^x.\) Ainsi \(x<e^x-1<xe^x.\)
\end{sphinxadmonition}

\begin{sphinxadmonition}{note}{Proposition (Règle de l’Hospital)}

\sphinxAtStartPar
Soient \(f,g:I\rightarrow \mathbb{R}\) deux fonctions dérivables et soit \(x_0\in I.\) On suppose que
\begin{itemize}
\item {} 
\sphinxAtStartPar
\(f(x_0)=g(x_0)=0,\)

\item {} 
\sphinxAtStartPar
\(\forall x\in I\setminus\{x_0\},\;g'(x)\neq0.\)

\end{itemize}

\sphinxAtStartPar
Si \(\lim\limits_{\substack{x\rightarrow x_0}}\frac{f'(x)}{g'(x)}=\ell \,(\ell\in \mathbb{R})\) alors \(\lim\limits_{\substack{x\rightarrow x_0}}\frac{f(x)}{g(x)}=\ell.\)
\end{sphinxadmonition}

\begin{sphinxadmonition}{note}{Exemple}

\sphinxAtStartPar
Calculer la limite en 1 de \(\frac{\ln(x^2+x-1)}{\ln(x)}.\) On vérifie que:
\begin{itemize}
\item {} 
\sphinxAtStartPar
\(f(x)=\ln(x^2+x-1),\, f(1)=0,\,f'(x)=\frac{2x+1}{x^2+x-1},\)

\item {} 
\sphinxAtStartPar
\(g(x)=\ln(x),\,g(1)=0,\,g'(x)=\frac{1}{x},\)

\item {} 
\sphinxAtStartPar
Prenons \(I=]0,1],\,x_0=1,\) alors \(g'\) ne s’annule pas sur \(I\setminus\{x_0\}.\)

\end{itemize}
\begin{equation*}
\begin{split}
\frac{f'(x)}{g'(x)}=\frac{2x+1}{x^2+x-1}\times x=\frac{2x^2+x}{x^2+x-1}\rightarrow 3,\,qd \,x\rightarrow 1
\end{split}
\end{equation*}
\sphinxAtStartPar
Donc, \(\frac{f'(x)}{g'(x)}\rightarrow 3,\,qd x\rightarrow 1.\)
\end{sphinxadmonition}


\section{Fonctions circulaires inverses}
\label{\detokenize{fcthycerinvs:fonctions-circulaires-inverses}}\label{\detokenize{fcthycerinvs::doc}}

\subsection{Arccosinus}
\label{\detokenize{fcthycerinvs:arccosinus}}
\sphinxAtStartPar
Considérons la fonction cosinus \(\cos : \mathbb{R}\rightarrow[−1,1], x \mapsto\cos x.\) Pour obtenir une bijection à partir de
cette fonction, il faut considérer la restriction de cosinus à l’intervalle \([0,\pi].\) Sur cet intervalle la
fonction cosinus est continue et strictement décroissante, donc la restriction
\begin{equation*}
\begin{split}
\cos:[0,\pi]\rightarrow[-1,1]
\end{split}
\end{equation*}
\sphinxAtStartPar
est une bijection. Sa bijection réciproque est la fonction \sphinxstylestrong{arccosinus:}
\begin{equation*}
\begin{split}
\arccos :[-1,1]\rightarrow[0,\pi]
\end{split}
\end{equation*}
\begin{figure}[htbp]
\centering
\capstart

\noindent\sphinxincludegraphics[height=150\sphinxpxdimen]{{sinconsinv}.png}
\caption{fonctions sin cos et leurs inverse}\label{\detokenize{fcthycerinvs:directive-fig}}\end{figure}

\sphinxAtStartPar
On a donc, par définition de la bijection réciproque:
\begin{equation*}
\begin{split}
\cos(\arccos(x)=x \quad\forall x \in [−1,1]
\end{split}
\end{equation*}\begin{equation*}
\begin{split}
\arccos(\cos(x)=x \quad x \in [0,\pi]
\end{split}
\end{equation*}
\sphinxAtStartPar
Autrement dit:
\begin{equation*}
\begin{split}
\mbox{Si}\quad x \in[0,\pi] \cos(x) =y \Leftrightarrow x = \arccos y
\end{split}
\end{equation*}
\sphinxAtStartPar
Notons finalement que la fonction arccosinus est dérivable sur l’intervalle \(]-1,1[\) et
\begin{equation*}
\begin{split}
\forall x\in ]-1,1[,\quad \arccos'(x)=\frac{-1}{\sqrt{1-x^2}}
\end{split}
\end{equation*}
\sphinxAtStartPar
\sphinxstylestrong{Démonstration:}

\sphinxAtStartPar
On a l’égalité \(\cos(\arccos x) = x\) que l’on dérive :
\begin{eqnarray*}
\cos(\arccos x)=x
&\Rightarrow& −\arccos'(x)\times\sin(\arccos x) = 1\\
&\Rightarrow&  \arccos'(x)=-\frac{1}{\sin(\arccos x)}\\
&\Rightarrow& \arccos'(x)=-\frac{1}{\sqrt{1-\cos^2(\arccos x)}}\\
&\Rightarrow& \arccos'(x)=-\frac{1}{\sqrt{1-x^2}}
\end{eqnarray*}

\subsection{Arcsinus}
\label{\detokenize{fcthycerinvs:arcsinus}}
\sphinxAtStartPar
La restriction
\begin{equation*}
\begin{split}
\sin:[-\frac{\pi}{2},\frac{\pi}{2}]\rightarrow [-1,1]
\end{split}
\end{equation*}
\sphinxAtStartPar
est une bijection. Sa bijection réciproque est la fonction \sphinxstylestrong{arcsinus} définie par
\begin{equation*}
\begin{split}
\sin(\arcsin(x))=x,\;\forall x\in [-1,1]
\end{split}
\end{equation*}\begin{equation*}
\begin{split}
\arcsin(\sin(x))=x,\;\forall x\in[-\frac{\pi}{2},\frac{\pi}{2}] 
\end{split}
\end{equation*}
\sphinxAtStartPar
On a alors: \(\forall x\in[-\frac{\pi}{2},\frac{\pi}{2}] , \sin (x)=y\Leftrightarrow x=\arcsin(y).\)

\sphinxAtStartPar
La fonction arcsinus est dérivable sur l’intervalle \(]-1,1[\) et
\begin{equation*}
\begin{split}
\forall x\in ]-1,1[,\quad \arcsin'(x)=\frac{1}{\sqrt{1-x^2}}
\end{split}
\end{equation*}

\subsection{Arctangente}
\label{\detokenize{fcthycerinvs:arctangente}}
\sphinxAtStartPar
La restriction \(\tan:]-\frac{\pi}{2},\frac{\pi}{2}[\rightarrow\mathbb{R}\) est une bijection. Sa bijection réciproque est la fonction arctangente:
\begin{equation*}
\begin{split}
\arctan:\mathbb{R}\rightarrow]-\frac{\pi}{2},\frac{\pi}{2}[
\end{split}
\end{equation*}
\sphinxAtStartPar
Ainsi
\begin{equation*}
\begin{split}
\tan(\arctan(x))=x,\forall x\in \mathbb{R}
\end{split}
\end{equation*}\begin{equation*}
\begin{split}
\arctan(\tan(x))=x, \forall x\in]-\frac{\pi}{2},\frac{\pi}{2}[
\end{split}
\end{equation*}
\sphinxAtStartPar
Et si \(x\in ]-\frac{\pi}{2},\frac{\pi}{2}[,\) alors \(\tan(x)=y\Leftrightarrow x=\arctan y.\)


\section{Fonctions hyperboliques et hyperboliques inverses}
\label{\detokenize{fcthycerinvs:fonctions-hyperboliques-et-hyperboliques-inverses}}

\subsection{Cosinus hyperbolique et son inverse}
\label{\detokenize{fcthycerinvs:cosinus-hyperbolique-et-son-inverse}}
\sphinxAtStartPar
Pour \(x\in \mathbb{R},\) le cosinus hyperbolique est :
\begin{equation*}
\begin{split}
ch(x)=\frac{e^x+e^{-x}}{2}
\end{split}
\end{equation*}
\sphinxAtStartPar
La restriction \(ch: [0,+\infty[\rightarrow[1,+\infty[\) est une bijection. Sa bijection réciproque est \(argch:[1,+\infty[\rightarrow[0,+\infty[\)


\subsection{Sinus hyperbolique et son inverse}
\label{\detokenize{fcthycerinvs:sinus-hyperbolique-et-son-inverse}}
\sphinxAtStartPar
Pour \(x\in \mathbb{R},\) le sinus hyperbolique est:
\begin{equation*}
\begin{split}
sh(x)=\frac{e^x-e^{-x}}{2}
\end{split}
\end{equation*}
\sphinxAtStartPar
\(sh:\mathbb{R}\rightarrow\mathbb{R}\) est une fonction continue, dérivable, strictement croissante vérifiant \(\lim\limits_{\substack{x\rightarrow-\infty}}shx=-\infty\) et \(\lim\limits_{\substack{x\rightarrow+\infty}}shx=+\infty,\)  c’est donc une bijection. Sa bijection réciproque est \(argsh:\mathbb{R}\rightarrow \mathbb{R}.\)

\begin{sphinxadmonition}{note}{Proposition}
\begin{itemize}
\item {} 
\sphinxAtStartPar
\(ch^2 x - sh^2 x=1.\)

\item {} 
\sphinxAtStartPar
\(ch'x = shx, \;sh'x = chx.\)

\item {} 
\sphinxAtStartPar
argsh est dérivable et \(argsh'x=\frac{1}{\sqrt{x^2+1}}\)

\item {} 
\sphinxAtStartPar
\(argshx = \ln(x+\sqrt{x^2 +1}).\)

\end{itemize}
\end{sphinxadmonition}

\begin{figure}[htbp]
\centering
\capstart

\noindent\sphinxincludegraphics[height=150\sphinxpxdimen]{{sh-ch}.png}
\caption{fonctions sh, ch, argch et argsh}\label{\detokenize{fcthycerinvs:id1}}\end{figure}


\section{Tangente hyperbolique et son inverse}
\label{\detokenize{fcthycerinvs:tangente-hyperbolique-et-son-inverse}}
\sphinxAtStartPar
Par définition la \sphinxstylestrong{tangente hyperbolique} est :
\begin{equation*}
\begin{split}
th x=\frac{sh x}{ch x}
\end{split}
\end{equation*}
\sphinxAtStartPar
La fonction \(th :\mathbb{R}\rightarrow]-1,1[\) est une bijection, on note \(argth :]-1,1[\rightarrow\mathbb{R}\) sa bijection réciproque.

\begin{figure}[htbp]
\centering
\capstart

\noindent\sphinxincludegraphics[height=150\sphinxpxdimen]{{th-argth}.png}
\caption{fonctions th et argth}\label{\detokenize{fcthycerinvs:id2}}\end{figure}


\subsection{Trigonométrie hyperbolique}
\label{\detokenize{fcthycerinvs:trigonometrie-hyperbolique}}\begin{equation*}
\begin{split}
ch^2 a-sh^2 a=1
\end{split}
\end{equation*}\begin{equation*}
\begin{split}
ch(a+b)=ch a. chb+sh a.shb
\end{split}
\end{equation*}\begin{equation*}
\begin{split}
ch(2a)= ch^2 a+sh^2 a=2 ch^2 a-1=1+2sh^2 a
\end{split}
\end{equation*}\begin{equation*}
\begin{split}
sh(a+b)=sh a. chb+sh b. ch a
\end{split}
\end{equation*}\begin{equation*}
\begin{split}
sh(2a)=2sha. cha
\end{split}
\end{equation*}\begin{equation*}
\begin{split}
th(a+b)=\frac{tha+thb}{1+tha.thb}
\end{split}
\end{equation*}
\sphinxAtStartPar
\sphinxstylestrong{Dérivées de fonctions hyperboliques et leurs inverses}
\begin{equation*}
\begin{split}
ch'x=shx
\end{split}
\end{equation*}\begin{equation*}
\begin{split}
sh'x=chx
\end{split}
\end{equation*}\begin{equation*}
\begin{split}
th'x=1-th^2 x=\frac{1}{ch^2 x}
\end{split}
\end{equation*}\begin{equation*}
\begin{split}
argch'x=\frac{1}{\sqrt{x^2-1}},\:|x|>1
\end{split}
\end{equation*}\begin{equation*}
\begin{split}
argsh'x=\frac{1}{\sqrt{x^2+1}}
\end{split}
\end{equation*}\begin{equation*}
\begin{split}
argth'x=\frac{1}{1-x^2},\:|x|<1
\end{split}
\end{equation*}\begin{equation*}
\begin{split}
argch(x)=\ln(x+\sqrt{x^2-1}),\:|x|\leq 1
\end{split}
\end{equation*}\begin{equation*}
\begin{split}
arhsh(x)=\ln(x+\sqrt{x^2+1}),\,(x\in\mathbb{R})
\end{split}
\end{equation*}\begin{equation*}
\begin{split}
argthx=\frac{1}{2}\ln(\frac{1+x}{1-x}),\,(-1<x<1)
\end{split}
\end{equation*}

\subsection{formules trigonometriques}
\label{\detokenize{fcthycerinvs:formules-trigonometriques}}
\sphinxAtStartPar
Rapel:
\begin{itemize}
\item {} 
\sphinxAtStartPar
les fonctions \(\sin\) et \(\cos\) sont definies sur \(\mathbb{R}\) et a valeurs dans \([-1, 1]\), \(2\pi\)\sphinxhyphen{}periodiques.

\item {} 
\sphinxAtStartPar
La fonction \(\tan\) est definie sur \(\mathbb{R}\setminus \{\frac{\pi}{2} +k\pi, k\in \mathbb{Z}\) a valeurs dans \(\mathbb{R}\), \(\pi\)\sphinxhyphen{}periodiques.

\item {} 
\sphinxAtStartPar
\(\cos^2(x) + \sin^2(x) = 1\)

\item {} 
\sphinxAtStartPar
\(\tan(x) = frac{\sin(x)}{\cos(x)}\)

\item {} 
\sphinxAtStartPar
\(\sin(-x) = -\sin(x)\)

\item {} 
\sphinxAtStartPar
\(\cos(-x) =\cos(x)\)

\item {} 
\sphinxAtStartPar
\(\tan(-x) = - \tan(x)\)

\item {} 
\sphinxAtStartPar
\( 1 + \tan^2(x) = \frac{1}{\cos^2(x)}\)

\item {} 
\sphinxAtStartPar
\(\sin(\pi - x) = \sin(x)\)

\item {} 
\sphinxAtStartPar
\(\cos(\pi - x) = - \cos(x)\)

\item {} 
\sphinxAtStartPar
\(\tan(\pi - x) = - \tan(x)\)

\item {} 
\sphinxAtStartPar
\(\sin(\frac{\pi}{2} - x)  = \cos(x)\)

\item {} 
\sphinxAtStartPar
\(\cos(\frac{\pi}{2} - x)  = \sin(x)\)

\item {} 
\sphinxAtStartPar
\(\tan(\frac{\pi}{2} - x)  = \frac{1}{\tan(x)}\)

\item {} 
\sphinxAtStartPar
\(\sin(2x) = 2\sin(x)\cos(x)\)

\item {} 
\sphinxAtStartPar
\(\cos(2x) = \cos^2(x) - \sin^2(x) = 2\cos^2(x) -1 = 1 - 2\sin^2(x)\)

\item {} 
\sphinxAtStartPar
\(\tan(2x) = \frac{2\tan(x)}{1-\tan^2(x)}\)

\item {} 
\sphinxAtStartPar
\(\sin(x) = \frac{2t}{1+t^2}, \mbox{ avec } t=\tan(\frac{x}{2})\)

\item {} 
\sphinxAtStartPar
\(\cos(x) = \frac{1-t^2}{1+t^2} , \mbox{ avec } t=\tan(\frac{x}{2})\)

\item {} 
\sphinxAtStartPar
\(\tan(x) = \frac{2t}{1-t^2} , \mbox{ avec } t=\tan(\frac{x}{2})\)

\end{itemize}


\section{Exercices}
\label{\detokenize{exo5:exercices}}\label{\detokenize{exo5::doc}}

\subsection{Exercice 1}
\label{\detokenize{exo5:exercice-1}}
\sphinxAtStartPar
Démontrer que, pour tous \(x,y \in \mathbb{R}\):
\begin{equation*}
\begin{split}
sh(x+y)=sh(x)ch(y)+ch(x)sh(y)
\end{split}
\end{equation*}\begin{equation*}
\begin{split}
ch(x+y)=ch(x)ch(y)+sh(x)sh(y).
\end{split}
\end{equation*}

\subsection{Exercice 2}
\label{\detokenize{exo5:exercice-2}}
\sphinxAtStartPar
Déterminer la valeur de \( \arcsin(-1/2), \arccos(-\sqrt 2/2), \arctan(\sqrt 3)\).


\subsection{Exercice 3}
\label{\detokenize{exo5:exercice-3}}
\sphinxAtStartPar
Calculer
\begin{equation*}
\begin{split}
\arccos \left(\cos\frac{2\pi}3\right),\quad \arccos\left(\cos\frac{-2\pi}{3}\right),\quad\arccos\left(\cos\frac{4\pi}{3}\right),\quad \arccos\left(\sin\frac{17\pi}5\right).
\end{split}
\end{equation*}

\subsection{Exercice 4}
\label{\detokenize{exo5:exercice-4}}
\sphinxAtStartPar
Soit \(a\neq 0\) un réel.
\begin{enumerate}
\sphinxsetlistlabels{\arabic}{enumi}{enumii}{}{.}%
\item {} 
\sphinxAtStartPar
Déterminer la dérivée de la fonction \(f\) définie sur \(\mathbb{R}\) par \(f(x)=\arctan(ax)\).

\item {} 
\sphinxAtStartPar
En déduire une primitive de \(\frac{1}{4+x^2}\).

\end{enumerate}


\subsection{Exercice 5}
\label{\detokenize{exo5:exercice-5}}
\sphinxAtStartPar
Simplifier les expressions suivantes :
\begin{equation*}
\begin{split}
\tan(\arcsin x),\quad \sin(\arccos x),\quad \cos(\arctan x).
\end{split}
\end{equation*}

\subsection{Exercice 6}
\label{\detokenize{exo5:exercice-6}}
\sphinxAtStartPar
Soit \(f\) la fonction définie par
\begin{equation*}
\begin{split}
f(x)=\arcsin\left(2x\sqrt{1-x^2}\right).
\end{split}
\end{equation*}\begin{enumerate}
\sphinxsetlistlabels{\arabic}{enumi}{enumii}{}{.}%
\item {} 
\sphinxAtStartPar
Quel est l’ensemble de définition de \(f\)?

\item {} 
\sphinxAtStartPar
En posant \(x=\sin t\), simplifier l’écriture de \(f\).

\end{enumerate}


\subsection{Exercice 7}
\label{\detokenize{exo5:exercice-7}}
\sphinxAtStartPar
Montrer que, pour tout \(x\in[-1,1], \arccos(x)+\arcsin(x)=\frac\pi2\)


\chapter{Complement sur les suites et les fonctions numériques}
\label{\detokenize{complement:complement-sur-les-suites-et-les-fonctions-numeriques}}\label{\detokenize{complement::doc}}
\sphinxAtStartPar
Le présent Chapitre contiendra : quelques notions qui completent les suites et les fonctions numeriques:
\begin{enumerate}
\sphinxsetlistlabels{\arabic}{enumi}{enumii}{}{.}%
\item {} 
\sphinxAtStartPar
Relations de comparaison des suites

\item {} 
\sphinxAtStartPar
Derivees succesives

\item {} 
\sphinxAtStartPar
Relations de comparaison des fonctions

\item {} 
\sphinxAtStartPar
Developements limites

\item {} 
\sphinxAtStartPar
Calcul integral

\item {} 
\sphinxAtStartPar
Exercices

\end{enumerate}


\section{Complément sur les suites et les fonctions numériques}
\label{\detokenize{complements:complement-sur-les-suites-et-les-fonctions-numeriques}}\label{\detokenize{complements::doc}}

\subsection{Suites et critère de Chauchy}
\label{\detokenize{complements:suites-et-critere-de-chauchy}}
\sphinxAtStartPar
Le très grand intéret du critère de Cauchy provient du fait qu’il caractérise dans \(\mathbb{R}\) les suites convergentes, sans que la limite apparaisse. D’où son utilisation dans l’étude des séries par exemple, ou encore pour montrer qu’une suite n’est pas convergente.

\sphinxAtStartPar
Le concept de suite de Cauchy correspond à la propriété que la distance entre deux termes de la suite devient arbitrairement petite (et non de plus en plus petite) quand ces termes sont de rang assez grand.

\begin{sphinxadmonition}{note}{Définition (suite de chauchy)}

\sphinxAtStartPar
Soit \((u_n)\) une suite réelle; on dit que \((u_n)\) est une suite de Cauchy ou vérifie le critère de Cauchy si :
\begin{equation*}
\begin{split}
\forall \epsilon>0, \exists N \in \mathbb{N},  \forall p, n \in \mathbb{N}, \left\{
\begin{array}{ll}
 p\geq N\\
n\geq N\\
\end{array}
\right. \Rightarrow |u_p-u_n|<\epsilon
\end{split}
\end{equation*}\end{sphinxadmonition}

\begin{sphinxadmonition}{note}{Théorème (Critère de Cauchy)}

\sphinxAtStartPar
Une suite de réels est convergente dans \(\mathbb{R}\) si, et seulement si, c’est une suite de Cauchy.
\end{sphinxadmonition}

\sphinxAtStartPar
Le critère de Cauchy est utilisé pour montrer qu’une suite \((u_n)\) est convergente (resp divergente) dans les cas où l’on peut obtenir facilement une majoration (resp minoration) de \(|u_p−u_n|\) pour \(n\) et \(p\) assez grands. C’est le cas en particulier pour certaines séries.


\subsection{Comparaison des suites}
\label{\detokenize{complements:comparaison-des-suites}}
\begin{sphinxadmonition}{note}{Définition}

\sphinxAtStartPar
Soient \((u_n)\), \((v_n)\) et \((\epsilon_n)\) trois suites telles qu’à partir d’un certain rang \(u_n = v_nε_n\). On dit que :
\begin{itemize}
\item {} 
\sphinxAtStartPar
\(u_n\) est dominée par \(v_n\) lorsque la suite \((\epsilon_n)\) est bornée. Notation : \(u_n = O(v_n)\).

\item {} 
\sphinxAtStartPar
\(u_n\) est négligeable devant \(v_n\) lorsque la suite \((\epsilon_n)\) tend vers 0 (\(\lim \epsilon_n =0\)). Notation : \(u_n = o(v_n)\).

\item {} 
\sphinxAtStartPar
\(u_n\) est équivalente à \(v_n\) lorsque la suite \((\epsilon_n)\) tend vers 1 (\(\lim \epsilon_n =1\)). Notation : \(u_n \sim v_n\).

\end{itemize}
\end{sphinxadmonition}

\begin{sphinxadmonition}{note}{Théorème (Caractérisations)}

\sphinxAtStartPar
Lorsque la suite \(v\) ne s’annule pas à partir d’un certain rang :
\begin{itemize}
\item {} 
\sphinxAtStartPar
\(u_n = O(v_n)\) si et seulement si la suite \(\frac{u}{v}\) est bornée.

\item {} 
\sphinxAtStartPar
\(u_n = o(v_n)\) si et seulement si \(\lim \frac{u_n}{v_n}= 0\).

\item {} 
\sphinxAtStartPar
\(u_n \sim v_n\) si et seulement si \(lim \frac{u_n}{v_n}= 1\).

\end{itemize}
\end{sphinxadmonition}

\begin{sphinxadmonition}{note}{Exemple}

\sphinxAtStartPar
On a:
\begin{itemize}
\item {} 
\sphinxAtStartPar
\(n =o(n^2)\) (on prend \(\epsilon = \frac{1}{n} \to 0\)).

\item {} 
\sphinxAtStartPar
\(ln(n) = o(n)\) car\(\frac{ln(n)}{n} \to 0\).

\item {} 
\sphinxAtStartPar
\(n\sin(n) = O(n)\) car \(\frac{n\sin(n)}{n}=sin(n)\) bornée.

\item {} 
\sphinxAtStartPar
\(\sin(n) = O(1)\) car \(\frac{\sin(n)}{1}=sin(n)\) bornée.

\item {} 
\sphinxAtStartPar
\(\frac{n}{n^2+1} \sim \frac{1}{n}\) car \(\dfrac{\frac{n}{n^2+1}}{\frac{1}{n}} = \frac{n^2}{n^2 +1} \to 1\).

\item {} 
\sphinxAtStartPar
\((1+\frac{1}{n})^n \sim e\) car \(\dfrac{(1+\frac{1}{n})^n}{e} \to 1\).

\end{itemize}
\end{sphinxadmonition}

\begin{sphinxadmonition}{note}{Remarque}
\begin{itemize}
\item {} 
\sphinxAtStartPar
\(u_n = O(1)\) signifie que la suite \((u_n)\) est bornée.

\item {} 
\sphinxAtStartPar
\(u_n = o(1)\) signifie que \(u_n\to 0\).

\item {} 
\sphinxAtStartPar
Si \(u_n = o(v_n)\) alors \(u_n = O(v_n)\).

\item {} 
\sphinxAtStartPar
Si \(u_n \sim v_n\) alors \(u_n = O(v_n)\).

\item {} 
\sphinxAtStartPar
Si \(u_n = o(v_n)\) et \(v_n = o(w_n)\), alors \(u_n = o(w_n)\) (transitivité).

\item {} 
\sphinxAtStartPar
Si \(u_n = O(v_n)\) et \(v_n = O(w_n)\), alors \(u_n = O(w_n)\) (transitivité).

\item {} 
\sphinxAtStartPar
\(u_n \sim v_n \Leftrightarrow u_n − v_n = o(v_n)\).

\item {} 
\sphinxAtStartPar
si \(l \in \mathbb{R}\) et si \(u_n \sim l\) alors \(u_n \to l\) (réciproque vraie lorsque \(l \neq 0\)).

\item {} 
\sphinxAtStartPar
Si \(u_n = o(v_n)\) alors \(\forall \lambda \in \mathbb{R}, \lambda u_n + v_n \sim v_n\).

\end{itemize}
\end{sphinxadmonition}

\begin{sphinxadmonition}{note}{Théorème (croissances comparées)}

\sphinxAtStartPar
Soient \(\alpha, \beta > 0\):
\begin{itemize}
\item {} 
\sphinxAtStartPar
si \(\alpha < \beta\) alors \(n^\alpha = o(n^\beta)\) et \( \frac{1}{n^\beta} = o(\frac{1}{n^\alpha})\).

\item {} 
\sphinxAtStartPar
\(ln(n)^\alpha = o(n^\beta)\).

\item {} 
\sphinxAtStartPar
\(n^\alpha =o(e^{n\beta})\) et \(n^\alpha =o(e^{n^\beta})\).

\item {} 
\sphinxAtStartPar
\(\forall a \in \mathbb{R}, a^n =o(n!)\) et \(n^\alpha = o(n!)\).

\item {} 
\sphinxAtStartPar
\(n! = o(n^n)\).

\end{itemize}
\end{sphinxadmonition}

\begin{sphinxadmonition}{note}{Théorème (les équivalents usuels)}
\begin{itemize}
\item {} 
\sphinxAtStartPar
Soit \((u_n)\) une suite de limite nulle (\(\lim u_n =0\)). On a:
\begin{itemize}
\item {} 
\sphinxAtStartPar
Si \(f : ]− a;a[\to \mathbb{R}\) (avec \(a > 0\)) est dérivable en \(0\), et si \(f'(0) \neq 0\), alors on a \(f(u_n)− f(0) \sim f'(0)u_n\).

\item {} 
\sphinxAtStartPar
\(\sin(u_n) \sim u_n\),

\item {} 
\sphinxAtStartPar
\(e^{u_n} - 1 \sim u_n\),

\item {} 
\sphinxAtStartPar
\(ln(1+u_n)\sim u_n\),

\item {} 
\sphinxAtStartPar
\(\tan(u_n) \sim u_n\),

\item {} 
\sphinxAtStartPar
\((1+u_n)^\alpha -1 \sim \alpha u_n\).

\end{itemize}

\item {} 
\sphinxAtStartPar
Soit \(P(x) = \sum_{k=0}^{p} a_kx^k\) une fonction polynomiale avec \(a_p \neq 0\), alors \(P(n) \sim a_pn^p\) (équivalence avec le
terme de plus haut degré).

\item {} 
\sphinxAtStartPar
Soit \(Q(x) =\frac{P(x)}{R(x)}\) une fraction rationnelle avec \(a_px^p\) le terme de plus haut degré de \(P\) (\(a_p \neq 0\)) et \(b_rx^r\) celui de \(R\) (\(b_r \neq 0\)), alors \(Q(n) \sim \frac{a_p}{b_r}n^{p−r}\) (équivalence avec le rapport des termes de plus haut degré).

\end{itemize}
\end{sphinxadmonition}

\begin{sphinxadmonition}{note}{Théorème}

\sphinxAtStartPar
Soient \(u\) et \(v\) deux suites,
\begin{itemize}
\item {} 
\sphinxAtStartPar
Si \(u_n \sim v_n\) alors les deux suites ont le meme singe à partir d’un certain rang.

\item {} 
\sphinxAtStartPar
Si \(u_n \sim v_n\) et si \(\lim v_n = l \in \mathbb{R}\cup \{+\infty, -\infty\}\), alors \(\lim u_n = l\).

\item {} 
\sphinxAtStartPar
Si \(u_n \sim v_n\) et si \(a_n \sim b_n\), alors \(u_na_n \sim v_nb_n\) (compatibilité avec la multiplication).

\item {} 
\sphinxAtStartPar
Si \(u_n \sim v_n\) et si \(v\) ne s’annule pas à partir d’un certain rang, alors \(\frac{1}{u_n} \sim \frac{1}{v_n}\) (compatibilité avec le
passage à l’inverse).

\item {} 
\sphinxAtStartPar
Si \(u_n \sim v_n\) et si \(v_n > 0\) ne s’annule pas à partir d’un certain rang, alors \(u_n^\alpha \sim v_n^\alpha\) pour tout réel \(\alpha\)
(compatibilité avec les puissances constantes).

\end{itemize}
\end{sphinxadmonition}

\begin{sphinxadmonition}{warning}{Avertissement:}\begin{itemize}
\item {} 
\sphinxAtStartPar
Si \(u_n = o(v_n)\) et \(w_n = o(v_n)\) ce ne veut pas dire que \(u_n=w_n\)!!!

\item {} 
\sphinxAtStartPar
Il n’y a pas compatibilité avec l’addition en général, par exemple : \(n+\frac{1}{n} \sim n\) et \(−n \sim 1−n\), mais \(\frac{1}{n}\)
n’est pas équivalent à \(1\).

\item {} 
\sphinxAtStartPar
Ces propriétés sont utiles pour les calculs de limites qui ne peuvent pas être faits directement : on essaie de se
ramener à un équivalent plus simple (s’il y en a …) dont on sait calculer la limite.

\end{itemize}
\end{sphinxadmonition}

\begin{sphinxadmonition}{note}{Note:}
\sphinxAtStartPar
L’écriture \(u_n = v_n + o(w_n)\) signifie que \(u_n-v_n = o(w_n)\).
\end{sphinxadmonition}

\begin{sphinxadmonition}{note}{Exemples}
\begin{itemize}
\item {} 
\sphinxAtStartPar
\(\dfrac{1}{n} = o(1)\)

\item {} 
\sphinxAtStartPar
\(1=o(n)\)

\item {} 
\sphinxAtStartPar
\((\dfrac{1}{2})^n = o(n)\)

\item {} 
\sphinxAtStartPar
\(n^2 = o(n^3)\)

\item {} 
\sphinxAtStartPar
\((ln(n))^2 = o(n)\)

\item {} 
\sphinxAtStartPar
\(3^n = o(n!)\)

\item {} 
\sphinxAtStartPar
\(e^n = o(3^n)\)

\item {} 
\sphinxAtStartPar
\(n^3 = o(e^n)\)

\item {} 
\sphinxAtStartPar
\(\dfrac{1}{n+1} = \dfrac{1}{n} + o(\dfrac{1}{n})\)

\item {} 
\sphinxAtStartPar
\(n^2 + 2ln(n) + 4 + \dfrac{1}{n} \sim n^2\)

\item {} 
\sphinxAtStartPar
\(\dfrac{2}{n} - \dfrac{4}{n^2} + \dfrac{2}{n^4} \sim \dfrac{2}{n}\)

\item {} 
\sphinxAtStartPar
\(2n^2 - n+n^3 \sim n^3\)

\item {} 
\sphinxAtStartPar
\(ln(n) + (ln(n))^2 \sim (ln(n))^2\)

\end{itemize}
\end{sphinxadmonition}

\begin{sphinxadmonition}{note}{Théorème (formule de stirling)}

\sphinxAtStartPar
\(n! \sim (\frac{n}{e})^n\sqrt{2\pi n}\)
\end{sphinxadmonition}


\subsection{Dérivées successives}
\label{\detokenize{complements:derivees-successives}}
\begin{sphinxadmonition}{note}{Définition}

\sphinxAtStartPar
Soit une fonction \(f\) dérivable sur \(D \subset \mathbb{R}\). Sa fonction dérivée \(f^{'}\) est appelé dérivée première (ou d’ordre 1) de la fonction \(f\) sur \(D\).
\(f^{"}\), est appelé dérivée seconde (ou d’ordre 2) de la fonction \(f\).
Par itération, pour tout naturel \(n > 2\), on définit la fonction dérivée n\sphinxhyphen{}ième (ou d’ordre \(n\)), notée \(f^{(n)}\) comme étant la fonction dérivée de la fonction dérivée d’ordre \((n − 1)\), soit : \(f^{(n)}= (f^{(n-1)})^{'}\).
\end{sphinxadmonition}

\begin{sphinxadmonition}{note}{Exemple}

\sphinxAtStartPar
Déterminer les dérivées d’ordre 1, 2 et 3 des fonctions suivantes :
\begin{itemize}
\item {} 
\sphinxAtStartPar
\(f(x) = x^4 - 2x^2 + x- 1\)

\item {} 
\sphinxAtStartPar
\(g(x) = cos(2x) + sin(2x)\)

\end{itemize}
\end{sphinxadmonition}

\begin{sphinxadmonition}{note}{Exercice}

\sphinxAtStartPar
Soit la fonction \(f\) définie sur \(] − 1 ; +\infty[\) par : \(f(x) = \ln(x + 1)\).
\begin{enumerate}
\sphinxsetlistlabels{\arabic}{enumi}{enumii}{}{.}%
\item {} 
\sphinxAtStartPar
Calculer les dérivées d’ordre 1, 2, 3 et 4.

\item {} 
\sphinxAtStartPar
En déduire et Démontrer par récurrence l’expression de la dérivée d’ordre \(n\).

\end{enumerate}
\end{sphinxadmonition}


\subsection{Fonctions de classe \protect\(\mathscr{C}^n\protect\)}
\label{\detokenize{complements:fonctions-de-classe-mathscr-c-n}}
\begin{sphinxadmonition}{note}{Définition}

\sphinxAtStartPar
On note :
\begin{itemize}
\item {} 
\sphinxAtStartPar
\(\mathscr{C}^0(I)\) est l’ensemble des fonctions continues sur \(I\).

\item {} 
\sphinxAtStartPar
\(\mathscr{C}^1(I)\) est l’ensemble des fonctions continûment dérivables sur \(I\), i.e. l’ensemble des fonctions qui sont dérivables sur \(I\) dont la fonction dérivée \(f^{'}\) est continue sur \(I\).

\item {} 
\sphinxAtStartPar
\(\mathscr{C}^n(I)\) est l’ensemble des fonctions \(n\) fois continûment dérivables sur \(I\), i.e. l’ensemble des fonctions n\sphinxhyphen{}fois dérivables sur \(I\) dont la fonction dérivée n\sphinxhyphen{}ième \(f^{(n)}\) est continue sur \(I\) ;

\item {} 
\sphinxAtStartPar
\(\mathscr{C}^\infty(I)\) est l’ensemble des fonctions indéfiniment dérivables sur \(I\).

\end{itemize}
\end{sphinxadmonition}


\subsection{Dérivée n\sphinxhyphen{}ième usuelles}
\label{\detokenize{complements:derivee-n-ieme-usuelles}}
\begin{sphinxadmonition}{note}{Proposition}

\sphinxAtStartPar
Soient \(a\in \mathbb{R}\), \(n\in \mathbb{N}\), \(f\) et \(g\) deux fonctions telles ques \(\forall x \in \mathbb{R}, f(x) = x^n\) et \(g(x) = (x-a)^n\). Alors \(f\) et \(g\) sont de classe \(\mathscr{C}^\infty\) sur \(\mathbb{R}\) et \(\forall k \in \mathbb{N}, \forall x \in \mathbb{R},\)
\begin{equation*}
\begin{split} 
f^{(k)}(x) =\left\{
\begin{array}{ll}
n(n-1)\times ...\times (n-k+1)x^{n-k} \quad\mbox{si} \;k \leq n\\
0 \quad\mbox{si}\;k>l
\end{array}
\right.
\end{split}
\end{equation*}\begin{equation*}
\begin{split}
g^{(k)}(x) =\left\{
\begin{array}{ll}
n(n-1)\times ...\times (n-k+1)(x-a)^{n-k} \quad\mbox{si} \;k \leq n\\
0 \quad\mbox{si}\;k>l
\end{array}
\right.
\end{split}
\end{equation*}\end{sphinxadmonition}

\begin{sphinxadmonition}{note}{Proposition}

\sphinxAtStartPar
Soient \(a\in \mathbb{R}\), \(f\) et \(g\) deux fonctions telles ques \(\forall x \in \mathbb{R^{*}}, f(x) = \dfrac{1}{x}\) et \(\forall x \in \mathbb{R}\setminus \{a\}, g(x) = \dfrac{1}{x-a}\). Alors \(f\) et \(g\) sont de classe \(\mathscr{C}^\infty\) sur \(\mathbb{R^{*}}\) et \(\mathbb{R}\setminus \{a\}\) respectivement et on a:
\begin{itemize}
\item {} 
\sphinxAtStartPar
\(\forall n \in \mathbb{N}, \forall x \in \mathbb{R^{*}}, f^{(n)}(x) =\dfrac{(-1)^nn!}{x^{n+1}}\)

\item {} 
\sphinxAtStartPar
\(\forall n \in \mathbb{N}, \forall x \in \mathbb{R^{*}}, g^{(n)}(x) =\dfrac{(-1)^nn!}{(x-a)^{n+1}}\)

\end{itemize}
\end{sphinxadmonition}

\begin{sphinxadmonition}{note}{Proposition}

\sphinxAtStartPar
Les fonctions \(\exp, \ln, \sin, \cos\) sont de classe \(\mathscr{C}^\infty\) sur leur domaine de définition et on pour tout \(n\in \mathbb{N}\):
\begin{itemize}
\item {} 
\sphinxAtStartPar
\(\exp^{(n)}(x) = \exp(x)\)

\item {} 
\sphinxAtStartPar
\(\ln^{(n)}(x) = \dfrac{(-1)^{n-1}(n-1)!}{x^{n}}\)

\item {} 
\sphinxAtStartPar
\(\sin^{(n)}(x) = \sin(x+n\frac{\pi}{2})\)

\item {} 
\sphinxAtStartPar
\(\cos^{(n)}(x) = \cos(x+ n\frac{\pi}{2})\)

\end{itemize}
\end{sphinxadmonition}

\begin{sphinxadmonition}{note}{Proposition}

\sphinxAtStartPar
Si \(f\) et \(g\) sont deux fonctions de classe \(\mathscr{C}^n\) sur un intervalle \(I\), alors pour tout \(\lambda\in \mathbb{R}, (\lambda f + g)\) et \(fg\) sont encore des fonctions de classe \(\mathscr{C}^n\) et de plus :
\begin{itemize}
\item {} 
\sphinxAtStartPar
\((αf + g)^{(n)} = αf^{(n)} + g^{(n)}\)

\item {} 
\sphinxAtStartPar
\((fg)^{(n)} = \sum_{i=0}^{n} \binom{n}{k}f^{(k)}g^{(n-k})\) ( avec \(\binom{n}{k} =\dfrac{n!}{k!(n-k)!})\)

\end{itemize}
\end{sphinxadmonition}

\begin{sphinxadmonition}{note}{Proposition}

\sphinxAtStartPar
Soient \(I\) et \(J\) deux intervalles et soit \(f : I \to \mathbb{R}\) et \(g : J \to \mathbb{R}\) avec \(f(I) \subset J\).
Si \(f\) est de classe \(\mathscr{C}^n\) sur \(I\), et si \(g\) est de classe \(\mathscr{C}^n\) sur \(J\), alors \(g\circ f\) est de classe \(\mathscr{C}^n\) sur \(I\).
\end{sphinxadmonition}

\begin{sphinxadmonition}{note}{Remarques}
\begin{itemize}
\item {} 
\sphinxAtStartPar
\(\forall \alpha \in \mathbb{R}, x \mapsto x^\alpha=\exp(\alpha\ln(x))\) est de classe \(\mathscr{C}^\infty\) sur \(]0, +\infty[\).

\item {} 
\sphinxAtStartPar
\(\forall a >0, x \mapsto a^x=\exp(x\ln(a))\) est de classe \(\mathscr{C}^\infty\) sur \(\mathbb{R}\).

\item {} 
\sphinxAtStartPar
Si \(f\) et \(g\) sont deux fonctions de classe \(\mathscr{C}^n\) sur un intervalle \(I\) et si \(g\) ne s’annule pas sur l’intervalle \(I\), alors la fonction \( x \mapsto\dfrac{f(x)}{g(x)}\) est de classe \(\mathscr{C}^n\) sur l’intervalle \(I\).

\end{itemize}
\end{sphinxadmonition}


\subsection{Comparaison des fonctions}
\label{\detokenize{complements:comparaison-des-fonctions}}
\sphinxAtStartPar
Soit I un intervalle de \(\mathbb{R}\) et \(a\in \bar{I}\). Nous supposerons ici que \(f\) et \(g\) sont deux fonctions qui ne s’annulent pas sur un voisinage de \(a\) privé de \(a\).
Il s’agit ici de comparer les 2 fonctions au voisinage de \(a\).  Pour cela, formons leur rapport \(\dfrac{f(x)}{g(x)}\) et regardons ce qui se passe lorsque \(x \to a\).

\begin{sphinxadmonition}{note}{Définition}
\begin{enumerate}
\sphinxsetlistlabels{\arabic}{enumi}{enumii}{}{.}%
\item {} 
\sphinxAtStartPar
Si \(\dfrac{f(x)}{g(x)}\) est bornée au voisinage de \(a\), on dira que \(f\) est dominée par \(g\) au voisinage de \(a\) et on écrit:

\end{enumerate}
\begin{equation*}
\begin{split}
f  \underset{x\to a}{O}(g) \mbox{ ou } f  \underset{a}{O}(g) \mbox{ ou encore } f = O(g) \mbox{ au voisinage de } a
\end{split}
\end{equation*}\begin{enumerate}
\sphinxsetlistlabels{\arabic}{enumi}{enumii}{}{.}%
\item {} 
\sphinxAtStartPar
Si \(\dfrac{f(x)}{g(x)}\) tend vers \(0\) lorsque \(x\) tend vers \(a\) (\(\underset{x\to a}{\lim} \dfrac{f(x)}{g(x)}= 0\)), on dira que \(f\) est négligeable devant \(g\) au voisinage de \(a\) et on écrit:

\end{enumerate}
\begin{equation*}
\begin{split}
f  \underset{x\to a}{o}(g) \mbox{ ou } f  \underset{a}{o}(g) \mbox{ ou encore } f = o(g) \mbox{ au voisinage de } a
\end{split}
\end{equation*}\begin{enumerate}
\sphinxsetlistlabels{\arabic}{enumi}{enumii}{}{.}%
\item {} 
\sphinxAtStartPar
Si \(\dfrac{f(x)}{g(x)}\) tend vers \(1\) lorsque \(x\) tend vers \(a\) (\(\underset{x\to a}{\lim} \dfrac{f(x)}{g(x)}= 1\)), on dira que \(f\) et \(g\) sont équivalentes au voisinage de \(a\) et on écrit:

\end{enumerate}
\begin{equation*}
\begin{split}
f  \underset{x\to a}{\sim}g \mbox{ ou } f  \underset{a}{\sim}g \mbox{ ou encore } f \sim g \mbox{ au voisinage de } a
\end{split}
\end{equation*}\end{sphinxadmonition}

\begin{sphinxadmonition}{note}{Remarque}
\begin{itemize}
\item {} 
\sphinxAtStartPar
Notation : \(f(x) = O(g(x))\), \(f(x) = o(g(x))\), \(f(x) \sim g(x)\) au voisinage de \(a\).

\item {} 
\sphinxAtStartPar
Par abus de langage, on notera \(O(g)\) (respectivement \(o(g)\)) toute fonction étant un grand \(O\)  (repectivement petit \(o\)) de \(g\) au voisinage de \(a\).

\item {} 
\sphinxAtStartPar
Lorsque \(f(x) = O(g(x))\), on pourra dans un calcul remplacer \(f(x)\) par \(O(g(x))\) mais \sphinxstylestrong{ATTENTION! IL EST FAUX de remplacer \(O(g(x))\) par \(f(x)\)}.

\item {} 
\sphinxAtStartPar
Lorsque \(f(x) = o(g(x))\), on pourra dans un calcul remplacer \(f(x)\) par \(o(g(x))\) mais \sphinxstylestrong{ATTENTION! IL EST FAUX de remplacer \(o(g(x))\) par \(f(x)\)}.

\item {} 
\sphinxAtStartPar
\sphinxstylestrong{Ne JAMAIS écrire que \(f(x) \underset{a}{\sim}0\)  puisque la fonction nulle ne vérifie pas les conditions d’application de la définition!!}

\item {} 
\sphinxAtStartPar
Il faut toujours specifier le voisinage du point dont on compare deux fonctions!

\item {} 
\sphinxAtStartPar
\(f = O(1)\) au voisinage de \(a\) signifie que \(f\) est bornée au voisinage de \(a\).

\item {} 
\sphinxAtStartPar
\(f = o(1)\) au voisinage de \(a\) signifie que \(f\) tend vers \(0\) lorsque \(x\to a\) (\(\underset{x\to a}{\lim} f(x) = 0\)).

\item {} 
\sphinxAtStartPar
Il n’y a aucune relation entre \(f(x) \underset{a}{\sim} g(x)\) et \(\underset{x\to a}{\lim} f(x) - g(x) = 0\).

\end{itemize}
\end{sphinxadmonition}

\begin{sphinxadmonition}{note}{Exemples}
\begin{enumerate}
\sphinxsetlistlabels{\arabic}{enumi}{enumii}{}{.}%
\item {} 
\sphinxAtStartPar
Soit \(f(x) = 3x^5 − x^4 + 2x\). Alors:
\begin{itemize}
\item {} 
\sphinxAtStartPar
\(f(x) = O(x)\) au voisinage de \(0\).

\item {} 
\sphinxAtStartPar
\(f(x) = O(x^5)\) au voisinge de \(+\infty\).

\item {} 
\sphinxAtStartPar
\(f(x) = o(x)\) au voisinge de \(0\).

\item {} 
\sphinxAtStartPar
\(f(x) = o(x^6)\) au voisinage de \(+\infty\).

\end{itemize}

\item {} 
\sphinxAtStartPar
Si \(P\) est une fonction polynomiale non nulle :
\begin{itemize}
\item {} 
\sphinxAtStartPar
\(P\) est équivalente à son monôme de plus haut degré au voisinage de \(+\infty\)

\item {} 
\sphinxAtStartPar
\(P\) est équivalente à son monôme de plus bas degré au voisinage de \(0\)

\end{itemize}

\item {} 
\sphinxAtStartPar
Au voisinage de \(+\infty\) : \(ch(x) \sim \dfrac{e^x}{2}\) et \(sh(x) \sim \dfrac{e^x}{2}\)

\end{enumerate}
\end{sphinxadmonition}

\begin{sphinxadmonition}{note}{Remarque}

\sphinxAtStartPar
Une fonction donnée admet une infinité d’équivalents au voisinage d’un point a. Seulement l’intérêt
d’un équivalent est de remplacer une fonction par une autre fonction plus simple. On choisira donc toujours l’équivalent le
plus simple.

\sphinxAtStartPar
Par exemple, au voisinage de \(+\infty\) on a:
\begin{itemize}
\item {} 
\sphinxAtStartPar
\(x^3 - 2x^2 + 3x-5 \sim x^3\)

\item {} 
\sphinxAtStartPar
\(x^3 - 2x^2 + 3x-5 \sim x^3-5x+8\)

\item {} 
\sphinxAtStartPar
\(x^3 - 2x^2 + 3x-5 \sim x^3+4x^2-6\)

\item {} 
\sphinxAtStartPar
\(x^3 - 2x^2 + 3x-5 \sim x^3 - 2x^2\)

\end{itemize}

\sphinxAtStartPar
Seul le premier équivalent a un intérêt!!
\end{sphinxadmonition}

\begin{sphinxadmonition}{note}{Proprietes}
\begin{itemize}
\item {} 
\sphinxAtStartPar
Soit \(p, q \in \mathbb{N}\) alors \(x^p = o(x^q)\) au voisinage de \(0\) \(\Leftrightarrow p<p\).

\item {} 
\sphinxAtStartPar
Si au voisinage d’un point \(a\) on a \(f(x) = o(g(x))\) alors \(f(x) = O(g(x))\).

\item {} 
\sphinxAtStartPar
Soient \(\alpha, \beta, \gamma >0\) trois reels strictment positifs:
\begin{itemize}
\item {} 
\sphinxAtStartPar
Comparaison \(\ln\) et puissances:
\begin{itemize}
\item {} 
\sphinxAtStartPar
au voisinage de \(+\infty\): \((\ln(x))^\alpha = o(x^\beta)\).

\item {} 
\sphinxAtStartPar
au voisinage de 0+: \(|\ln(x)|^\alpha = o(\frac{1}{x^\beta})\).

\end{itemize}

\item {} 
\sphinxAtStartPar
Comparaison puissance et exponentielle :
\begin{itemize}
\item {} 
\sphinxAtStartPar
au voisinage de \(+\infty\): \(x^\beta = o(e^{\gamma x})\)

\item {} 
\sphinxAtStartPar
au voisinage de \(+\infty\): \(x^\beta = o(a^{x})\) si \(a>1\).

\item {} 
\sphinxAtStartPar
au voisinage de \(-\infty\): \(e^{\gamma x} = o(\frac{1}{x^\beta})\) si \(\beta \in \mathbb{N}\).

\end{itemize}

\end{itemize}

\end{itemize}
\end{sphinxadmonition}

\begin{sphinxadmonition}{note}{Proposition}
\begin{itemize}
\item {} 
\sphinxAtStartPar
Opérations sur les relations de comparaisons
\begin{enumerate}
\sphinxsetlistlabels{\arabic}{enumi}{enumii}{}{.}%
\item {} 
\sphinxAtStartPar
\(f = o(g), g = o(h) \Leftarrow f = o(h)\) cad (transitivité) idem avec \(O\)

\item {} 
\sphinxAtStartPar
\(f_1 = o(g), f_2 = o(g) \Leftarrow f_1 + f_2 = o(g)\) cad \(o(g) + o(g) = o(g)\) idem avec \(O\)

\item {} 
\sphinxAtStartPar
\(f_1 = o(g_1), f_2 = o(g_2) \Leftarrow f_1f_2 = o(g_1g_2)\) cad \(o(g_1)o(g_2) = o(g_1g_2)\) idem avec \(O\)

\item {} 
\sphinxAtStartPar
\(f = o(g) \Leftarrow hf = o(hg)\) cad \(ho(g) = o(hg)\) idem avec \(O\)

\item {} 
\sphinxAtStartPar
\(f = o(\lambda g) (\lambda \in \mathbb{R}^{*}) \Leftarrow f = o(g)\) cad \( o(\lambda g) = o(g)\) idem avec \(O\)

\end{enumerate}

\item {} 
\sphinxAtStartPar
Caractérisation de l’équivalence de deux fonctions
On a au voisinage d’un point \(a\) :
\begin{equation*}
\begin{split}
    f(x) \sim g(x) \Leftrightarrow f(x) = g(x) + o(g(x))
    \end{split}
\end{equation*}
\sphinxAtStartPar
Cela sera particulieremet utile lorsqu’on souhaitera remplacer une expression par un équivalent dans une égalité

\item {} 
\sphinxAtStartPar
Lien entre les relations de comparaison

\end{itemize}

\sphinxAtStartPar
On se place au voisinage d’un point \(a\).
1. Si \(f(x) \sim g(x)\) alors \(f(x) = O(g(x))\).
2. Si \(f(x) \sim g(x)\) et \(f(x) = o(h(x))\) alors \(g(x) = o(h(x))\).
3. Si \(f(x) \sim g(x)\) et    \(h(x) = o(f(x))\) alors \(h(x) = o(g(x))\)
\begin{itemize}
\item {} 
\sphinxAtStartPar
Calculs avec des équivalents
\begin{enumerate}
\sphinxsetlistlabels{\arabic}{enumi}{enumii}{}{.}%
\item {} 
\sphinxAtStartPar
si \(f(x) \underset{x\to a}{\to} l\) et \(l\neq 0\) alors \(f \underset{a}{\sim} l\)

\item {} 
\sphinxAtStartPar
si \(f_1 \underset{a}{\sim} g_1\) et \(f_2 \underset{a}{\sim} g_2\) alors \(f_1f_2 \underset{a}{\sim} g_1g_2\) et \(\dfrac{f_1}{f_2} \underset{a}{\sim} \dfrac{g_1}{g_2}\)

\item {} 
\sphinxAtStartPar
si \(f \underset{a}{\sim} g\) et \(f\) et \(g\) sont positive alors \(f^\alpha \underset{a}{\sim} g^\alpha\) pour tout \(\alpha \in \mathbb{R}\).

\end{enumerate}

\end{itemize}
\end{sphinxadmonition}

\begin{sphinxadmonition}{note}{Théorème (Les équivalents de références)}

\sphinxAtStartPar
Les limites usuelles en 0, nous donnent les équivalents suivants au voisinage de \(0\) :
\begin{itemize}
\item {} 
\sphinxAtStartPar
\(\sin x \sim x\)

\item {} 
\sphinxAtStartPar
\(\arcsin x \sim x\)

\item {} 
\sphinxAtStartPar
\(sh x \sim x\)

\item {} 
\sphinxAtStartPar
\(\tan x \sim x\)

\item {} 
\sphinxAtStartPar
\(\arctan x \sim x\)

\item {} 
\sphinxAtStartPar
\(th x \sim x\)

\item {} 
\sphinxAtStartPar
\(1 − \cos x \sim \frac{x^2}{2}\)

\item {} 
\sphinxAtStartPar
\(1 − ch x \sim \frac{x^2}{2}\)

\item {} 
\sphinxAtStartPar
\(\ln(1 + x) \sim x\)

\item {} 
\sphinxAtStartPar
\(e^x − 1 \sim x\)

\item {} 
\sphinxAtStartPar
\((1 + x)^α − 1 \sim αx\)

\end{itemize}

\sphinxAtStartPar
Plus généralement, au voisinage de a lorsque \(\underset{x\to a}{f(x) \to 0}\), on a :
\begin{itemize}
\item {} 
\sphinxAtStartPar
\(\sin f(x) \sim f(x)\)

\item {} 
\sphinxAtStartPar
\(\arcsin f(x) \sim f(x)\)

\item {} 
\sphinxAtStartPar
\(sh f(x) \sim f(x)\)

\item {} 
\sphinxAtStartPar
\(\tan f(x) \sim f(x)\)

\item {} 
\sphinxAtStartPar
\(\arctan f(x) \sim f(x)\)

\item {} 
\sphinxAtStartPar
\(th x \sim f(x)\)

\item {} 
\sphinxAtStartPar
\(1 − \cos f(x) \sim \frac{f(x)^2}{2}\)

\item {} 
\sphinxAtStartPar
\(1 − ch f(x) \sim \frac{f(x)^2}{2}\)

\item {} 
\sphinxAtStartPar
\(\ln(1 + f(x)) \sim f(x)\)

\item {} 
\sphinxAtStartPar
\(e^{f(x)} − 1 \sim f(x)\)

\item {} 
\sphinxAtStartPar
\((1 + f(x))^α − 1 \sim αf(x)\)

\end{itemize}
\end{sphinxadmonition}

\begin{sphinxadmonition}{note}{Théorème (calcul avec les equivalence)}
\begin{enumerate}
\sphinxsetlistlabels{\arabic}{enumi}{enumii}{}{.}%
\item {} 
\sphinxAtStartPar
Si \(\underset{x\to a}{f(x) \to l}\) et \(l\neq 0\) alors \(f\underset{a}{\sim}l\).

\item {} 
\sphinxAtStartPar
Si \(f_1 \underset{a}{\sim}g_1\) et  \(f_2 \underset{a}{\sim}g_2\) alors  \(f_1f_2 \underset{a}{\sim}g_1g_2\) et  \(\frac{f_1}{f_2} \underset{a}{\sim}\frac{g_1}{g_2}\).

\item {} 
\sphinxAtStartPar
Soit \(\alpha \in \mathbb{R}\). Si \(f \underset{a}{\sim}g\) et \(f\) et \(g\) sont positives alors \(f^\alpha \underset{a}{\sim}g^\alpha\) .

\end{enumerate}
\end{sphinxadmonition}

\begin{sphinxadmonition}{note}{Proposition (Un équivalent donne localement le signe de la fonction)}

\sphinxAtStartPar
Soient deux fonctions \(f, g : I \to \mathbb{R}\) et un point \(a \in \bar{I}\).
Si au voisinage du point a, \(f \sim g\) alors, il existe un voisinage \(V\) de \(a\) sur lequel \(f\) et \(g\) ont même signe.
\end{sphinxadmonition}

\begin{sphinxadmonition}{note}{Proposition (Un équivalent donne la limite)}

\sphinxAtStartPar
Soient deux fonctions \(f, g : I \to \mathbb{R}\) et un point \(a \in \bar{I}\).

\sphinxAtStartPar
Si \(f\underset{a}{\sim} g\) et \(\underset{x\to a}{\lim} g(x) = l\) alors \(\underset{x\to a}{\lim} f(x) = l\).
\end{sphinxadmonition}


\section{Exercices}
\label{\detokenize{exo6:exercices}}\label{\detokenize{exo6::doc}}

\subsection{Exercice 1}
\label{\detokenize{exo6:exercice-1}}
\sphinxAtStartPar
Quels sont les équivalents corrects parmi les propositions suivantes?
\begin{equation*}
\begin{split}\begin{array}{lllll}
 \mathbf 1.\ n\sim_{+\infty}n+1&\quad&\mathbf 2.\ n^2\sim_{+\infty}n^2+n&\quad&\mathbf 3.\ \ln(n)\sim_{+\infty}\ln(10^6 n)\\
 \mathbf 4.\ \exp(n)\sim_{+\infty}\exp\left(n+10^{-6}\right)&\quad&\mathbf 5.\ \exp(n)\sim_{+\infty}\exp(2n)&\quad&\mathbf 6.\ \ln(n)\sim_{+\infty}\ln(n+1).
\end{array}
\end{split}
\end{equation*}

\subsection{Exercice 2}
\label{\detokenize{exo6:exercice-2}}
\sphinxAtStartPar
Trouver un équivalent le plus simple possible aux suites suivantes :
\begin{equation*}
\begin{split}
\begin{array}{lll}
\mathbf 1.\ u_n=\frac{1}{n-1}-\frac{1}{n+1}&\quad&\mathbf 2.\ v_n=\sqrt{n+1}-\sqrt{n-1}\\
\mathbf 3.\ w_n=\frac{n^3-\sqrt{1+n^2}}{\ln n-2n^2}&\quad&\mathbf 4.\ z_n=\sin\left(\frac1{\sqrt{n+1}}\right).
\end{array}
\end{split}
\end{equation*}

\subsection{Exercice 3}
\label{\detokenize{exo6:exercice-3}}
\sphinxAtStartPar
Déterminer un équivalent le plus simple possible des fonctions suivantes :
\begin{equation*}
\begin{split}
\begin{array}{lcl}
\mathbf 1.\ x+1+\ln x\textrm{ en 0 et en }+\infty&\quad\quad&\displaystyle \mathbf 2.\ \cos(\sin x)\textrm{ en 0}\\
\displaystyle \mathbf 3.\ \cosh(\sqrt x)\textrm{ en }+\infty
&\quad\quad&\displaystyle \mathbf 4.\ \frac{\sin x\ln(1+x^2)}{x\tan x}\textrm{ en 0}\\
\displaystyle \mathbf 5.\ \ln(\sin x)\textrm{ en }0
&\quad\quad&\displaystyle \mathbf 6.\ \ln(\cos x)\textrm{ en 0}
\end{array}\end{split}
\end{equation*}

\subsection{Exercice 4}
\label{\detokenize{exo6:exercice-4}}
\sphinxAtStartPar
Classer les suites suivantes par ordre de « négligeabilité » :
\begin{equation*}
\begin{split}
\begin{array}{llll}
a_n=\frac 1n&b_n=\frac1{n^2}&c_n=\frac{\ln n}n&d_n=\frac{e^n}{n^3}\\
e_n=n&f_n=1&g_n=\sqrt{ne^n}.
\end{array}\end{split}
\end{equation*}

\subsection{Exercice 5}
\label{\detokenize{exo6:exercice-5}}
\sphinxAtStartPar
Classer les fonctions suivantes par ordre de négligeabilité en \(+\infty\)
\begin{equation*}
\begin{split}
f_1(x)=x,\ f_2(x)=\exp(x),\ f_3(x)=\frac 1x,\ f_4(x)=2,\ f_5(x)=\ln(x),\ f_6(x)=\sqrt x\ln x,\ f_7(x)=\frac{e^x}{\sqrt x}.
\end{split}
\end{equation*}

\subsection{Exercice 6}
\label{\detokenize{exo6:exercice-6}}\begin{enumerate}
\sphinxsetlistlabels{\arabic}{enumi}{enumii}{}{.}%
\item {} 
\sphinxAtStartPar
Démontrer que
\begin{equation*}
\begin{split}
    \ln(1+x)+x^2\sim_0 x\textrm{ et }x^2+x^3\sim_0 x^2.
    \end{split}
\end{equation*}
\sphinxAtStartPar
En déduire \(\displaystyle \lim_{x\to 0^+}\frac{\ln(1+x)+x^2}{x^2+x^3}\)

\item {} 
\sphinxAtStartPar
Démontrer que
\begin{equation*}
\begin{split}
    \sin(2x)\sim_0 2x\textrm{ et }\tan(3x)\sim_0 3x.
    \end{split}
\end{equation*}
\sphinxAtStartPar
En déduire \(\displaystyle \lim_{x\to 0^+}\frac{\sin(2x)}{\tan(3x)}\)

\end{enumerate}


\subsection{Exercice 7}
\label{\detokenize{exo6:exercice-7}}
\sphinxAtStartPar
En utilisant (éventuellement) des équivalents, déterminer les limites suivantes :
\begin{equation*}
\begin{split}
\begin{array}{lcl}
\displaystyle \mathbf 1.\ \lim_{x\to 0}\frac{(1-\cos x)(1+2x)}{x^2-x^4}&&\displaystyle \mathbf 2.\ \lim_{x\to 0}x(3+x)\frac{\sqrt{x+3}}{\sqrt x\sin(\sqrt x)}\\
\displaystyle \mathbf 3.\ \lim_{x\to 0}\frac{\ln (1+\sin x)}{\tan(6x)}&&
\displaystyle \mathbf 4.\ \lim_{x\to\pi/2}\frac{\ln(\sin^2 x)}{\left(\frac{\pi}{2}-x\right)^2}\\
\displaystyle \mathbf 5.\ \lim_{x\to 0}\frac{\ln(\cos x)}{1-\cos 2x}&&\displaystyle \mathbf 6.\ \lim_{x\to+\infty}\sqrt{4x+1}\ln\left(1-\frac{\sqrt{x+1}}{x+2}\right)\\
\displaystyle \mathbf 7.\ \lim_{x\to+\infty}\exp\left(\frac1{x^2}\right)-
\exp\left(\frac{1}{(x+1)^2}\right)
&&\displaystyle \mathbf 8.\ \lim_{x\to 0}\left(\frac{x}{\sin x}\right)^{\frac{\sin x}{x-\sin x}}\\\displaystyle \mathbf 9.\ \lim_{x\to 0}\frac{(1-\cos x)\arctan x}{x\tan x}
\end{array}\end{split}
\end{equation*}

\subsection{Exercice 8}
\label{\detokenize{exo6:exercice-8}}
\sphinxAtStartPar
Comparer les fonctions suivantes :
\begin{enumerate}
\sphinxsetlistlabels{\arabic}{enumi}{enumii}{}{.}%
\item {} 
\sphinxAtStartPar
\(x\ln x\) et \(\ln(1+2x)\) au voisinage de \(0\).

\item {} 
\sphinxAtStartPar
\(x\ln x\) et \(\sqrt{x^2+3x}\ln(x^2)\sin x\) au voisinage de \(+\infty\).

\end{enumerate}

\sphinxAtStartPar
Étudier si les fonctions suivantes sont dérivables et \(C^1\) sur \(\mathbb{R}\):
\begin{equation*}
\begin{split}
f(x)=\left\{\begin{array}{ll}
x^2\sin\left(\frac 1x\right)&x\neq 0\\
0&x=0
\end{array}\right.\quad\quad\quad
g(x)=\left\{\begin{array}{ll}
x^3\sin\left(\frac 1x\right)&x\neq 0\\
0&x=0.
\end{array}\right.
\end{split}
\end{equation*}

\subsection{Exercice 9}
\label{\detokenize{exo6:exercice-9}}
\sphinxAtStartPar
Calculer les développements limités suivants :
\begin{equation*}
\begin{split}
\begin{array}{lcl}
\displaystyle \mathbf 1.\ \frac{1}{1-x}-e^x\textrm{ à l'ordre 3 en 0}&&\displaystyle \mathbf 2.\ \sqrt{1-x}+\sqrt{1+x}\textrm{ à l'ordre 4 en 0}\\
\displaystyle \mathbf 3.\ \sin x\cos(2x)\textrm{ à l'ordre 6 en 0}&&\displaystyle \mathbf 4.\ \cos(x)\ln(1+x)\textrm{ à l'ordre 4 en 0}\\
\displaystyle \mathbf 5.\ (x^3+1)\sqrt{1-x}\textrm{ à l'ordre 3 en 0}&&\displaystyle \mathbf 6.\ \big(\ln(1+x)\big)^2\textrm{ à l'ordre 4 en 0}
\end{array}
\end{split}
\end{equation*}

\subsection{Exercice 10}
\label{\detokenize{exo6:exercice-10}}
\sphinxAtStartPar
Déterminer les développements limités des fonctions suivantes :
\begin{equation*}
\begin{split}
\begin{array}{lcl}
\displaystyle \mathbf 1.\ \frac{1}{1+x+x^2}\textrm{ à l'ordre 4 en 0}&&\displaystyle \mathbf 2.\ \tan(x)\textrm{ à l'ordre 5 en 0}\\
\displaystyle \mathbf 3.\ \frac{\sin x-1}{\cos x+1}\textrm{ à l'ordre 2 en 0}&&\displaystyle \mathbf 4.\ \frac{\ln(1+x)}{\sin x}\textrm{ à l'ordre 3 en 0}.
\end{array}
\end{split}
\end{equation*}

\subsection{Exercice 11}
\label{\detokenize{exo6:exercice-11}}\begin{equation*}
\begin{split}
\begin{array}{lcl}
\displaystyle \mathbf 1.\ \ln\left(\frac{\sin x}{x}\right)\textrm{ à l'ordre 4 en 0}&&
\displaystyle \mathbf 2.\ \exp(\sin x)\textrm{ à l'ordre 4 en 0}\\
\displaystyle \mathbf 3.\ (\cos x)^{\sin x}\textrm{ à l'ordre 5 en 0}&&
\displaystyle \mathbf 4.\ x\big(\cosh x\big)^{\frac 1x}\textrm{ à l'ordre 4 en 0}.
\end{array}
\end{split}
\end{equation*}

\section{Dévelopements limités}
\label{\detokenize{dl:developements-limites}}\label{\detokenize{dl::doc}}

\subsection{Formule de Taylor}
\label{\detokenize{dl:formule-de-taylor}}
\begin{sphinxadmonition}{note}{Théorème}

\sphinxAtStartPar
Soit \(f : I\rightarrow\mathbb{R}\) une fonction de classe \(\mathscr{C}^{n+1},\,(n\in\mathbb{N},\) et soit \(a, x ∈\in I.\) Il existe un réel \(c\) entre \(a\) et \(x\) tel que:
\begin{equation*}
\begin{split}
f(x)=f(a)+f'(a)(x-a)+\frac{f''(a)}{2!}(x-a)^2+...+\frac{f^{(n)}(a)}{n!}(x-a)^n+\frac{f^{(n+1)}(c)}{(n+1)!}(x-a)^{n+1}
\end{split}
\end{equation*}\end{sphinxadmonition}

\sphinxAtStartPar
On notera par \(T_n\) la partie définie par
\begin{equation*}
\begin{split}
T_n(x):=f(a)+f'(a)(x-a)+\frac{f''(a)}{2!}(x-a)^2+...+\frac{f^{(n)}(a)}{n!}(x-a)^n
\end{split}
\end{equation*}
\begin{sphinxadmonition}{note}{Exercice}

\sphinxAtStartPar
Soient \(a, x \in\mathbb{R},\,(a<x).\) Pour tout entier \(n > 0\) il existe \(c\in]a,x[\)  tel que
\begin{equation*}
\begin{split}
\exp x=\exp a+\exp (a).(x-a)+\frac{\exp (a)}{2!}.(x-a)^2+\ldots+\frac{\exp (a)}{n!}.(x-a)^n+\frac{\exp (c)}{(n+1)!}.(c-a)^{n+1}
\end{split}
\end{equation*}
\sphinxAtStartPar
Dans la pratique, le nombre \(c\) ne s’éstime qua approximativement (via un encadrement), la proposition suivante en donnera un.
\end{sphinxadmonition}

\begin{sphinxadmonition}{note}{Proposition}

\sphinxAtStartPar
Si en plus la fonction \(|f^{(n+1)}|\) est majorée sur \(I\) par un réel positif \(M,\) alors pour tout \(a,\,x \in I,\) on a:
\begin{equation*}
\begin{split}
|f(x)-T_n(x)|\leq M\frac{|x-a|^{n+1}}{(n+1)!}
\end{split}
\end{equation*}\end{sphinxadmonition}

\begin{sphinxadmonition}{note}{Exercice}

\sphinxAtStartPar
Approximation de \(\sin(0,01)\)

\sphinxAtStartPar
Soit \(f(x)=\sin(x),\) alors \(f'(x)=\cos x,\,f''(x)=-\sin x,\,f^{(3)}(x)=-\cos x,\\
\,f^{(4)}(x)=\sin x.\) On obtient donc \(f(0)=0,\,f'(0)=1,\,f''(0)=0,\,f^{(3)}(x)=-1.\) La formule de Taylor ci\sphinxhyphen{}dessus en \(a=0\) à l’ordre 3 devient \(f(x)=0+1.x+0.\frac{x^2}{2!}-1.\frac{x^3}{3!}+f^{(4)}(c)\frac{x^4}{4!}\)

\sphinxAtStartPar
Pour \(x=0,01,\) et en négligeant le reste (qui est assez petit), on obtient:
\begin{equation*}
\begin{split}
\sin(0,01)\approx 0,01-\frac{(0,01)^3}{6}=0,00999983333...
\end{split}
\end{equation*}\end{sphinxadmonition}

\begin{sphinxadmonition}{note}{Remarque}
\begin{itemize}
\item {} 
\sphinxAtStartPar
Dans ce théorème l’hypothèse \(f\) de classe \(\mathscr{C}^{n+1}\)  peut\sphinxhyphen{}être affaiblie \(f\) est « \(n+1\) fois dérivable sur \(I\) » sans avoir besoin que \(f^{(n+1)}\) soit continue.

\item {} 
\sphinxAtStartPar
Pour \(n=0\) c’est exactement l’énoncé du théorème des accroissements finis : il existe \(c\in ]a,b[\) tel que \(f(b)=f(a)+f'(c)(b-a).\)

\item {} 
\sphinxAtStartPar
Si \(I\) est un intervalle fermé borné  et \(f\) de classe \(\mathscr{C}^{n+1},\) alors \(f\) est continue sur \(I\) donc il existe un \(M\) tel que \(|f^{(n+1)}(x)|\leq M\) pour tout \(x\in I.\) . Ce qui permet toujours d’appliquer la proposition.

\end{itemize}
\end{sphinxadmonition}

\begin{sphinxadmonition}{note}{Théorème(Théorème de Taylor\sphinxhyphen{}Young)}

\sphinxAtStartPar
Soit \(f : I\rightarrow\mathbb{R}\) une fonction de classe \(\mathscr{C}^{n},\,(n\in\mathbb{N},\) et soit \(a\in I.\) Alors pour tout \(x\in I\) on a:
\begin{equation*}
\begin{split}
f(x)=f(a)+f'(a)(x-a)+\frac{f''(a)}{2!}(x-a)^2+...+\frac{f^{(n)}(a)}{n!}(x-a)^n+(x-a)^n \varepsilon(x)
\end{split}
\end{equation*}
\sphinxAtStartPar
avec \(\varepsilon(x)\rightarrow0\) quand \(x\rightarrow a.\)
\end{sphinxadmonition}

\begin{sphinxadmonition}{note}{Exercice}

\sphinxAtStartPar
Soit \(f :]-1,+\infty[\rightarrow \mathbb{R},\, x \mapsto\ln(1+x);\) \(f\) est infiniment dérivable. Appliquons la formule de Taylor\sphinxhyphen{}Young à la fonction \(f\) au voisinage de \(0,\) soit \(T_n\) la fonction polynômiale associée à chaque \(n,\) pour cela on calcule d’abord \(f^{(k)}(0)\) pour \(k=0,1,...,n.\)

\sphinxAtStartPar
\(f(0)=0,\,f'(0)=1,\,f''(0)=-1,\,f^{(3)}(0)=2,...\) Montrer par récurrence que \(f^{(n)}(x)=(-1)^{n-1}(n-1)!\frac{1}{(1+x)^n}\) et alors \(f^{(n)}(0)=(-1)^{n-1}(n-1)!\)

\sphinxAtStartPar
Voici donc les premiers polynômes de Taylor:
\begin{equation*}
\begin{split}
T_0(x)=0,\quad T_1(x)=x
\end{split}
\end{equation*}\begin{equation*}
\begin{split}
T_2(x)=x-\frac{x^2}{n},\quad T_3(x)=x-\frac{x^2}{2}+\frac{x^3}{3}
\end{split}
\end{equation*}
\sphinxAtStartPar
Les formules de Taylor nous disent que les restes sont de plus en plus petits lorsque \(n\) croît. Sur le dessins les graphes
des polynômes \(T_0,\,T_1,\,T_2,\,T_3\) s’approchent de plus en plus du graphe de \(f.\) Attention ceci n’est vrai qu’autour de \(0.\)
\end{sphinxadmonition}

\begin{figure}[htbp]
\centering
\capstart

\noindent\sphinxincludegraphics[height=150\sphinxpxdimen]{{Approximation2}.png}
\caption{Approximation}\label{\detokenize{dl:directive-fig}}\end{figure}

\sphinxAtStartPar
\sphinxstylestrong{Cas particulier : Formule de Taylor\sphinxhyphen{}Young au voisinage de 0.} On se ramène souvent au cas particulier où \(a=0\)
\begin{equation*}
\begin{split}
f(x)=f(0)+f'(0)x+\frac{f''(0)}{2!}x^2+...+\frac{f^{(n)}(0)}{n!}x^n+x^n \varepsilon(x)
\end{split}
\end{equation*}
\sphinxAtStartPar
avec \(\varepsilon(x)\rightarrow0\) quand \(x\rightarrow .\)

\sphinxAtStartPar
On écrit souvent cette formule avec la notation dite « petit \(o\) »
\begin{equation*}
\begin{split}
f(x)=f(0)+f'(0)x+\frac{f''(0)}{2!}x^2+...+\frac{f^{(n)}(0)}{n!}x^n+o(x^n)
\end{split}
\end{equation*}

\subsection{Développements limités au voisinage d’un point}
\label{\detokenize{dl:developpements-limites-au-voisinage-d-un-point}}

\subsubsection{Définition et existence}
\label{\detokenize{dl:definition-et-existence}}
\sphinxAtStartPar
Soit \(I\) un intervalle ouvert et \(f :I\rightarrow\mathbb{R}\) une fonction quelconque.

\begin{sphinxadmonition}{note}{Definition}

\sphinxAtStartPar
Pour \(a\in  I\) et \(n\in\mathbb{N},\) on dit que \(f\) admet un \sphinxstylestrong{développement limité (DL)} au point \(a\) et à l’ordre \(n,\) s’il existe des
réels \(c_0,c_1,...,c_n\) et une fonction \(\varepsilon:I\rightarrow\mathbb{R}\) telle que \(\lim\limits_{\substack{x \rightarrow a}}\varepsilon(x)=0\) de sorte que pour tout \(x\in I:\)
\begin{equation*}
\begin{split}
f(x)=c_0+c_1(x-a)+c_2(x-a)^2+...+c_n(x-a)^n+(x-a)^n \varepsilon(x)
\end{split}
\end{equation*}\begin{itemize}
\item {} 
\sphinxAtStartPar
L’égalité précédente s’appelle un DL de \(f\) au voisinage de \(a\) à l’ordre \(n.\)

\item {} 
\sphinxAtStartPar
Le terme \(c_0+c_1(x-a)+c_2(x-a)^2+...+c_n(x-a)^n\) est appelé \sphinxstylestrong{la partie polynomiale du DL}.

\item {} 
\sphinxAtStartPar
Le terme  \((x-a)^n \varepsilon(x)\) est appelé \sphinxstylestrong{le reste du DL.}

\end{itemize}
\end{sphinxadmonition}

\sphinxAtStartPar
La formule de Taylor\sphinxhyphen{}Young permet d’obtenir immédiatement des développements limités en posant \(c_k=\frac{f^{(k)}}{k!}.\)

\begin{sphinxadmonition}{note}{Remarque}
\begin{enumerate}
\sphinxsetlistlabels{\arabic}{enumi}{enumii}{}{.}%
\item {} 
\sphinxAtStartPar
Si \(f\) est de classe \(\mathscr{C}^{n}\) au voisinage d’un point \(0,\) un DL en \(0\) à l’ordre \(n\) est l’expression:
\begin{equation*}
\begin{split}
    f(x)=f(0)+f'(0)x+\frac{f''(0)}{2!}x^2+...+\frac{f^{(n)}(0)}{n!}x^n+x^n \varepsilon(x)
    \end{split}
\end{equation*}
\item {} 
\sphinxAtStartPar
Si \(f\) admet un DL en un point \(a\) à l’ordre \(n\) alors elle en possède un pour tout \(k\leq n.\) En effet

\sphinxAtStartPar
\(f(x)=f(a)+f'(a)(x-a)+\frac{f''(a)}{2!}(x-a)^2+...+\frac{f^{(k)}(a)}{k!}(x-a)^k\\
 +\frac{f^{(k+1)}(a)}{(k+1)!}(x-a)^{k+1}+...+\frac{f^{(n)}(a)}{n!}(x-a)^n+(x-a)^n \varepsilon(x)\)

\end{enumerate}
\end{sphinxadmonition}


\subsubsection{Unicité}
\label{\detokenize{dl:unicite}}
\begin{sphinxadmonition}{note}{Proposition}

\sphinxAtStartPar
Si \(f\) admet un \(DL\) alors ce \(DL\) est unique.
\end{sphinxadmonition}

\sphinxAtStartPar
\sphinxstylestrong{Preuve:} En exercice (supposer l’existence de deux DL, puis utiliser la propriété d’un polynôme nul.)

\begin{sphinxadmonition}{note}{Corollaire}

\sphinxAtStartPar
Si \(f\) est paire (resp. impaire) alors la partie polynomiale de son \(DL\) en \(0\) ne contient que des monômes de degrés pairs
(resp. impairs).
\end{sphinxadmonition}

\begin{sphinxadmonition}{note}{Remarque}

\sphinxAtStartPar
Si \(f\) admet un \(DL\) en un point \(a\) à l’ordre \(n\geq1,\) alors \(f\) est dérivable en \(a\) et on a \(c_0 = f(a)\) et \(c_1 = f'(a).\) Par
conséquent \(y = c_0 + c_1(x-a)\) est l’équation de la tangente au graphe de \(f\) au point d’abscisse \(a.\)
\end{sphinxadmonition}


\subsubsection{DL des fonctions usuelles à l’origine}
\label{\detokenize{dl:dl-des-fonctions-usuelles-a-l-origine}}
\sphinxAtStartPar
Les DL suivants en \(0\) proviennent de la formule de Taylor\sphinxhyphen{}Young.

\sphinxAtStartPar
\(\exp x=1+\frac{x}{1!}+\frac{x^2}{2!}+...+\frac{x^n}{n!}+x^n \varepsilon(x)\)

\sphinxAtStartPar
\(chx=1+\frac{x^2}{2!}+\frac{x^4}{4!}...+\frac{x^{2n}}{(2n)!}+x^{2n+1} \varepsilon(x)\)

\sphinxAtStartPar
\(shx=1+\frac{x}{1!}\frac{x^3}{3!}+\frac{x^5}{5!}...+\frac{x^{2n+1}}{(2n+1)!}+x^{2n+2} \varepsilon(x)\)

\sphinxAtStartPar
\(\cos x=1-\frac{x^2}{2!}+\frac{x^4}{4!}...+(-1)^n\frac{x^{2n}}{(2n)!}+x^{2n+1} \varepsilon(x)\)

\sphinxAtStartPar
\(shx=1+\frac{x}{1!}\frac{x^3}{3!}+\frac{x^5}{5!}...+(-1)^n\frac{x^{2n+1}}{(2n+1)!}+x^{2n+2} \varepsilon(x)\)

\sphinxAtStartPar
\(\frac{1}{1-x}=1+x+x^2+...+x^n+x^n \varepsilon(x)\)

\sphinxAtStartPar
\(\frac{1}{1+x}=1-x+x^2-x^3+...+(-1)^n x^n+x^n \varepsilon(x)\)

\sphinxAtStartPar
\(\ln(1+x)=x-\frac{x^2}{2}+\frac{x^3}{3}-...+(-1)^{n-1} \frac{x^n}{n}+x^n \varepsilon(x)\)

\sphinxAtStartPar
Généralement, pour \(\alpha\neq -1,\) on a:

\sphinxAtStartPar
\((1+x)^{\alpha}=1+\alpha x+\frac{\alpha(\alpha-1)}{2}x^2+...+\frac{\alpha(\alpha-1)...(\alpha-n+1)}{n!}x^n+x^{n} \varepsilon(x)\)

\sphinxAtStartPar
En particulier, pour \(\alpha=\frac{1}{2},\) on a:

\sphinxAtStartPar
\(\sqrt{1+x}=1+\frac{x}{2}-\frac{1}{8}x^2+...+(-1)^n\frac{1.3.5....(2n-3)}{2^n n!}x^n+x^n \varepsilon(x)\)


\subsubsection{\protect\(DL\protect\) des fonctions en un point quelconque}
\label{\detokenize{dl:dl-des-fonctions-en-un-point-quelconque}}
\sphinxAtStartPar
La fonction \(f\) admet un \(DL\) au voisinage d’un point \(a\) si et seulement si la fonction \(x\mapsto f(x+a)\) admet un \(DL\) au
voisinage de \(0.\) Souvent on ramène donc le problème en \(0\) en faisant le changement de variables \(h=x-a.\)

\begin{sphinxadmonition}{note}{Exercice}

\sphinxAtStartPar
\(DL\) de \(f(x)=exp x\) en \(1.\)

\sphinxAtStartPar
On pose \(h=x-1.\) Si \(x\) est proche de \(1\) alors \(h\) est proche de \(0.\) Nous allons nous ramener à un \(DL\) de \(exp h\) en
\(h = 0.\) On note \(e = exp 1.\)
\begin{equation*}
\begin{split}
\begin{eqnarray*}
\exp 
&=&\mbox{exp(1+(x-1))}=\exp (1)\exp (x-1)=e\exp h\\
&=& e(1+h+\frac{h^2}{2}+...+\frac{h^n}{n!}+h^n \varepsilon(h))\\
&=& e(1+(x-1)+\frac{(x-1)^2}{2}+...+\frac{(x-1)^n}{n!}+(x-1)^n \varepsilon(x-1)))\\
\end{eqnarray*}
\end{split}
\end{equation*}
\sphinxAtStartPar
où \(\lim\limits_{\substack{x \rightarrow 1}}\varepsilon(x-1)=0\)
\end{sphinxadmonition}


\subsection{Opérations sur les développements limités}
\label{\detokenize{dl:operations-sur-les-developpements-limites}}

\subsubsection{Somme et produit}
\label{\detokenize{dl:somme-et-produit}}
\sphinxAtStartPar
On suppose que \(f\) et \(g\) sont deux fonctions qui admettent des \(DL\) en \(0\) à l’ordre \(n:\)
\begin{equation*}
\begin{split}
f(x)=c_0+c_1 x+...+c_n x^n +x^n \varepsilon_1 (x)\quad\mbox{et}\quad g(x)=d_0+d_1 x+...+d_n x^n +x^n \varepsilon_2 (x)
\end{split}
\end{equation*}
\begin{sphinxadmonition}{note}{Proposition}
\begin{itemize}
\item {} 
\sphinxAtStartPar
\(f+g\) admet un \(DL\) en \(0\) l’ordre \(n\) qui est:

\end{itemize}

\sphinxAtStartPar
\((f+g)(x)=f(x)+g(x)=(c_0+d_0)+(c_1+d_1)x+(c_0+d_0)x^2+...+(c_n+d_n) x^n +x^{n}\varepsilon(x)\)
\begin{itemize}
\item {} 
\sphinxAtStartPar
\(f\times g\) admet un \(DL\) en \(0\) l’ordre \(n\) qui est:

\end{itemize}

\sphinxAtStartPar
\((f\times g)(x)=f(x)\times g(x)=T_n(x)+x^n \varepsilon(x)\) où \(T_n(x)\) est le polynôme \((c_0+c_1 x+...+c_n x^n)\times (d_0+d_1 x+...+d_n x^n)\) tronqué à l’ordre \(n.\)
\end{sphinxadmonition}

\begin{sphinxadmonition}{note}{Exercice}

\sphinxAtStartPar
Calculer le \(DL\) de \(\cos x \times\sqrt{1+x}\) à l’ordre \(2,\)

\sphinxAtStartPar
On a \(\cos x= 1-\frac{x^2}{2}+x^2 \varepsilon_1 (x)\) et \(\sqrt{1+x}=1+\frac{1}{2}x-\frac{1}{8}x^2+x^2\varepsilon_2 (x).\)

\sphinxAtStartPar
Donc:
\begin{equation*}
\begin{split}
\begin{eqnarray*}
\cos x \times\sqrt{1+x}
&=& (1-\frac{x^2}{2}+x^2 \varepsilon_1 (x))\times (1+\frac{1}{2}x-\frac{1}{8}x^2+x^2\varepsilon_2 (x))\\
&=& 1+\frac{1}{2}x-\frac{5}{8}x^2+x^2 \varepsilon(x)
\end{eqnarray*}
\end{split}
\end{equation*}\end{sphinxadmonition}


\subsubsection{Composition}
\label{\detokenize{dl:composition}}
\sphinxAtStartPar
Soient \(f(x)=C(x)+x^n \varepsilon_1 (x)=c_0+c_1 x+...+c_n x^n +x^n \varepsilon_1 (x)\) et \(g(x)=D(x)+x^n \varepsilon_2 (x)=d_0+d_1 x+...+d_n x^n +x^n \varepsilon_2 (x)\)

\begin{sphinxadmonition}{note}{Proposition}

\sphinxAtStartPar
Si \(g(0) = 0\) (c’est\sphinxhyphen{}à\sphinxhyphen{}dire \(d_0 = 0\)) alors la fonction \(f\circ g\) admet un \(DL\) en \(0\) à l’ordre \(n\) dont la partie polynomiale est le
polynôme tronqué à l’ordre \(n\) de la composition \(C(D(x)).\)
\end{sphinxadmonition}

\begin{sphinxadmonition}{note}{Exercice}

\sphinxAtStartPar
Soit \(h(x)=\sqrt{\cos x},\) On cherche le \(DL\) de \(h\) en \(0\) à l’ordre \(4.\) On pose \(f(u)=\sqrt{1+u},\) alors on a \(f(u)=1+\frac{1}{2}u-\frac{1}{8}u^2+u^2\varepsilon (u)\) au voisinage de \(0\) à l’ordre \(2,\) et si on pose \(u(x)=\cos x-1,\) alors \(h(x)=f(u(x)),\) et \(u(0)=0,\) D’autre part le \(DL\) de \(u(x)\) en \(x = 0\) à l’ordre \(4\) est: \(u=-\frac{1}{2}x^2+\frac{1}{24}x^4+x^4 \varepsilon(x),\) en appliquant la prop .., on a \(u^2=\frac{1}{4}x^4+x^4 \varepsilon(x).\) D’où,
\begin{equation*}
\begin{split}
\begin{eqnarray*}
h(x)=f(u)=1+\frac{1}{2}u-\frac{1}{8}u^2+u^2\varepsilon (u)
&=& 1+\frac{1}{2}(-\frac{1}{2}x^2+\frac{1}{24}x^4)-\frac{1}{8}(\frac{1}{4}x^4)+x^4 \varepsilon(x)\\
&=& 1-\frac{1}{4}x^2+\frac{1}{48}x^4-\frac{1}{32}x^4+x^4\varepsilon(x)\\
&=& 1-\frac{1}{4}x^2-\frac{1}{96}x^4+x^4\varepsilon(x)
\end{eqnarray*}
\end{split}
\end{equation*}\end{sphinxadmonition}


\subsubsection{Division}
\label{\detokenize{dl:division}}
\sphinxAtStartPar
Soit à déterminer le DL de \(f/g,\) quitte à multiplier le DL de \(f\) par celui de \(1/g,\) il suffier de trouver le DL  de ce dernier,

\sphinxAtStartPar
en posant \(f(x)=c_0+c_1 x+...+c_n x^n +x^n \varepsilon_1 (x)\quad\mbox{et}\quad g(x)=d_0+d_1 x+...+d_n x^n +x^n \varepsilon_2 (x)\)

\sphinxAtStartPar
cherchons le DL de \(1/g,\) pour cela on utilise le DL de \(\frac{1}{1+u},\)
\begin{enumerate}
\sphinxsetlistlabels{\arabic}{enumi}{enumii}{}{.}%
\item {} 
\sphinxAtStartPar
Si \(d_0=1,\) \(u=d_1 x+...+d_n x^n +x^n \varepsilon(x),\)

\item {} 
\sphinxAtStartPar
Si \(d_0\) est quelconque avec \(d_0 \neq0\) alors on se ramène au cas précédent en écrivant

\end{enumerate}

\sphinxAtStartPar
\(\frac{1}{g(x)}=\frac{1}{d_0}\frac{1}{1+\frac{d_1}{d_0}x+...+\frac{d_n}{d_0}x^n+\frac{x^n \varepsilon_2 (x)}{d_0}}\)
\begin{enumerate}
\sphinxsetlistlabels{\arabic}{enumi}{enumii}{}{.}%
\item {} 
\sphinxAtStartPar
Si \(d_0 = 0\) alors on factorise par \(x^k\) (pour un certain \(k\)) afin de se ramener aux cas précédents.

\end{enumerate}

\begin{sphinxadmonition}{note}{Exercice}

\sphinxAtStartPar
DL de \(\frac{1+x}{2+x}\) en \(0\) à l’ordre \(4.\)
\begin{equation*}
\begin{split}
\begin{eqnarray*}
\frac{1+x}{2+x}
&=& (1+x)\frac{1}{2}\frac{1}{1+\frac{x}{2}}\\
&=& \frac{1}{2}(1+x)(1-\frac{x}{2}+(\frac{x}{2})^2-(\frac{x}{2})^3+(\frac{x}{2})^4+x^4 \varepsilon(x))\\
&=& \frac{1}{2}+\frac{x}{4}-\frac{x^2}{8}+\frac{x^3}{16}-\frac{x^4}{32}+x^4 \varepsilon(x)
\end{eqnarray*}
\end{split}
\end{equation*}\end{sphinxadmonition}


\subsection{Applications des développements limités, Calculs de limites}
\label{\detokenize{dl:applications-des-developpements-limites-calculs-de-limites}}
\sphinxAtStartPar
Les DL sont très efficaces pour calculer des limites ayant des formes indéterminées ! Il suffit juste de remarquer que si \(f(x)=c_0+c_1 (x-a)+...\) alors \(\lim\limits_{\substack{x \rightarrow a}}f(x)=c_0=0\)

\sphinxAtStartPar
Limite en \(0\) de \(\frac{\ln(1+x)-\tan x+\frac{1}{2}\sin^2 x}{3 x^2 \sin^2 x}\)

\sphinxAtStartPar
Notons \(\frac{f(x)}{g(x)}\) cette fraction, en \(0,\) \(f(x)=\ln(1+x)-\tan x+\frac{1}{2}\sin^2 x=(x-\frac{x^2}{2}+\frac{x^3}{3}-\frac{x^4}{4}+o(x^4))-(x+\frac{x^3}{3}+o(x^4))+\frac{1}{2}(x-\frac{x^3}{6}+o(x^3))^2=-\frac{5}{12}x^4+o(x^4)\) et \(g(x)=3x^2 \sin^2 x=3x^4+o(x^4).\) Ainsi, \(\frac{f(x)}{g(x)}=\frac{-\frac{5}{12}x^4+o(x^4)}{3x^4+o(x^4)}=\frac{-\frac{5}{12}+o(1)}{3+o(1)}.\) D’où, \(\lim\limits_{\substack{x \rightarrow a}}\frac{f(x)}{g(x)}=-\frac{5}{36}.\)


\chapter{Intégration}
\label{\detokenize{integrationd:integration}}\label{\detokenize{integrationd::doc}}
\sphinxAtStartPar
Le présent Chapitre contiendra :
\begin{itemize}
\item {} 
\sphinxAtStartPar
Définition de l’intégration

\item {} 
\sphinxAtStartPar
Premitive et intégrale des fonctions continues

\item {} 
\sphinxAtStartPar
Methodes de calcul des premitives

\item {} 
\sphinxAtStartPar
Exercices

\end{itemize}


\section{Intégrale des fonctions en escalier}
\label{\detokenize{fe:integrale-des-fonctions-en-escalier}}\label{\detokenize{fe::doc}}
\sphinxAtStartPar
\(a\) et \(b\) désignent deux réels tels que \(a < b \). Toutes    les fonctions sont supposées définies sur \([a, b]\) et a valeurs réelles.
Le but de l’intégration est de définir un nombre qui, pour une fonction \(f\) positive sur un segment \([a, b]\) , mesure l’aire délimitée par sa courbe représentative, l’axe des abscisses et les deux droites d’équations \(x = a\) et \(x = b\).
Ce nombre sera appelé intégrale de \(f\) sur \([a, b]\) et notée :
\begin{equation*}
\begin{split}
\int_a^b f(x)dx
\end{split}
\end{equation*}

\subsection{Subdivision d’un segment}
\label{\detokenize{fe:subdivision-d-un-segment}}
\begin{sphinxadmonition}{note}{Définition 1}
\begin{itemize}
\item {} 
\sphinxAtStartPar
On appelle subdivision de \([a, b]\) toute famille \(u=(x_i)_{i=0}^n\) telle que \(n \in \mathbb{N}^\star\) et

\end{itemize}
\begin{equation*}
\begin{split}
a=x_0 < x_1 < \ldots < x_n=b
\end{split}
\end{equation*}\begin{itemize}
\item {} 
\sphinxAtStartPar
On appelle pas ou module de la subdivision \(u=(x_i)_{i=0}^n\), le reel

\end{itemize}
\begin{equation*}
\begin{split}
\delta(u)=\max_{i \in [1, n]}(x_i - x_{i-1})
\end{split}
\end{equation*}\end{sphinxadmonition}

\begin{sphinxadmonition}{note}{Exemple}

\sphinxAtStartPar
Une subdivision \((x_i)_{i=0}^n\) ou \(n\) un entier naturel non nul est dite a pas constant si:
\begin{equation*}
\begin{split}
\forall i \in \{0, \ldots, n\}, x_i = a+i\dfrac{b-a}{n}
\end{split}
\end{equation*}
\sphinxAtStartPar
Son module est \(\dfrac{b-a}{n}\)
\end{sphinxadmonition}

\begin{sphinxadmonition}{note}{Définition 2}

\sphinxAtStartPar
Si \(u\) et \(v\) sont deux subdivisions de \([a, b]\) , on dit que \(u\) est plus fine que \(v\) si tout élément de \(v\) est élément de \(u\).
\end{sphinxadmonition}

\begin{sphinxadmonition}{note}{Proposition 1}

\sphinxAtStartPar
Pour toutes subdivisions \(u\) et \(v\) de \([a, b]\) il existe une subdivision plus fine que  \(u\) et \(v\).
\end{sphinxadmonition}


\subsection{Fonctions en escalier}
\label{\detokenize{fe:fonctions-en-escalier}}
\begin{sphinxadmonition}{note}{Définition 3}

\sphinxAtStartPar
Une fonction \(\varphi\) de \([a,b]\) dans \(\mathbb{R}\) est dite en escalier si l’on peut trouver une subdivision \(u=(x_i)_{i=0}^n\) de \([a, b]\) telle que \(\varphi\) soit constante sur chacun des intervalles \(]x_{i-1}, x_i[, (1\leq i \leq n)\).
La subdivision \(u\) est dite adaptée à la fonction \(\varphi\).
\end{sphinxadmonition}

\begin{sphinxadmonition}{note}{Exemples}
\begin{itemize}
\item {} 
\sphinxAtStartPar
Une fonction constante sur l’intervalle \([a,b]\) est une fonction en escalier sur \([a,b]\).

\item {} 
\sphinxAtStartPar
La fonction \sphinxstyleemphasis{partie entière} est une fonction en escalier sur segment \([a,b]\) (pensez à une subdivision adaptée!).

\end{itemize}
\end{sphinxadmonition}

\sphinxAtStartPar
\sphinxstylestrong{Remarques}:
\begin{enumerate}
\sphinxsetlistlabels{\arabic}{enumi}{enumii}{}{.}%
\item {} 
\sphinxAtStartPar
Une fonction en escalier prend un nombre fini de valeurs. En particulier, elle est bornée.

\item {} 
\sphinxAtStartPar
Si \(u\) est une subdivision adaptée à une fonction \(\varphi\) en escalier, alors toute subdivision plus fine que \(u\) est adaptée à \(\varphi\).

\item {} 
\sphinxAtStartPar
Si \(\varphi\) et \(\psi\) sont deux fonctions en escalier sur \([a. b]\), alors il existe une subdivision adaptée à \(\varphi\) et \(\psi\).

\end{enumerate}

\begin{sphinxadmonition}{note}{Proposition 2}

\sphinxAtStartPar
L’ensemble des fonctions en escalier sur \([a, b]\) est un sous\sphinxhyphen{}espace vectoriel de l’espace des fonctions définies sur \([a, b]\)
\end{sphinxadmonition}

\begin{sphinxadmonition}{note}{Démonstration}

\sphinxAtStartPar
Soient \(\varphi\) et \(\psi\) deux fonctions en escalier sur \([a, b]\).

\sphinxAtStartPar
Soient \(\lambda\) et \(\mu\) deux réels.

\sphinxAtStartPar
D’après la proposition 1 il existe une subdivision \(u=(x_i)_{i=0}^n\) adaptées à \(\varphi\) et \(\psi\).
Donc les deux fonctions sont constantes sur \(]x_{i-1}, x_{i}[\)
Il en est de même pour \(\lambda \varphi + \mu \psi\) (constante sur \(]x_{i-1}, x_{i}[\)).
Donc, \(u\) est une subdivision adaptée pour \(\lambda \varphi + \mu \psi\) qui est par la suite une fonction en escalier.

\sphinxAtStartPar
Les autres conditions sont faciles à vérifier !
\end{sphinxadmonition}


\subsection{Intégrale d’une fonction en escalier}
\label{\detokenize{fe:integrale-d-une-fonction-en-escalier}}
\begin{sphinxadmonition}{note}{Proposition 3}

\sphinxAtStartPar
Soit \(\varphi\) une fonction en escalier sur \([a, b]\) et \(u=(x_i)_{i=0}^n\) une subdivision  adaptées à \(\varphi\). Soit \(c_i\) la valeur prise par \(\varphi\) sur \(]x_{i-1}, x_i[\) pour \(i \in \{1, \ldots, n\}\) (i.e \(\varphi(x)=c_i\) pour tout \(x \in ]x_{i-1}, x_i[\)). Alors la quantité
\begin{equation*}
\begin{split}
\sum_{i=1}^nc_i(x_i-x_{i-1})
\end{split}
\end{equation*}
\sphinxAtStartPar
ne dépend pas de la subdivision choisie.

\sphinxAtStartPar
Cette quantité s’appelle \sphinxcode{\sphinxupquote{l'intégrale}} de \(\varphi\) sur  \([a, b]\) et on le note:
\begin{equation*}
\begin{split}
\int_a^b \varphi(x)dx=\int_{[a, b]}\varphi
\end{split}
\end{equation*}\end{sphinxadmonition}

\begin{sphinxadmonition}{note}{Démonstration}

\sphinxAtStartPar
Pour toute \(u\) une subdivision adaptée à la fonction \(\varphi\), on note
\begin{equation*}
\begin{split}
I(\varphi, u)= \sum_{i=1}^nc_i(x_i-x_{i-1})
\end{split}
\end{equation*}
\sphinxAtStartPar
Le but est de montrer que \(\forall u, v\) subdivision adaptée à \(\varphi\), \(I(\varphi, u)=I(\varphi, v)\)
\begin{itemize}
\item {} 
\sphinxAtStartPar
Si \(v\) est plus fine que \(u\),
Elle est obtenue en rajoutant un nombre fini d’éléments à la subdivision \(u\). Pour démontrer que \(I(\varphi, u)=I(\varphi, v)\), il suffit de le démontrer dans le cas où  \(v\) a un élément de plus que \(u\).

\end{itemize}

\sphinxAtStartPar
Soit donc \(u=(x_i)_{i=0}^n\) et \(v=(x_1, \ldots, x_p, y, x_{p+1}, \ldots, x_n)\)

\sphinxAtStartPar
Il est claire que la fonction \(\varphi\) sur \(]x_p, y[\) et \(]y, x_{p+1}[\). Donc :
\begin{equation*}
\begin{split}
\begin{aligned}
I(\varphi, v) & =  \sum_{i=1}^{p-1}c_i(x_i-x_{i-1})+c_p(y-x_p) + c_p(x_{p+1}-y) + \sum_{i=p+2}^{n}c_i(x_i-x_{i-1})  \\ \\
 & =  \sum_{i=1}^{n}c_i(x_i-x_{i-1}) = I(\varphi, u) 
\end{aligned}
\end{split}
\end{equation*}\begin{itemize}
\item {} 
\sphinxAtStartPar
Dans le cas général:

\end{itemize}

\sphinxAtStartPar
D’après la proposition 1, il existe une subdivision \(w\) plus fine que \(u\) et \(v\), cette subdivision est aussi adaptée à \(\varphi\) (remarque 2).
Donc on aura d’une part \(I(\varphi, w)=I(\varphi, u)\) et d’autre part \(I(\varphi, w)=I(\varphi, v)\).

\sphinxAtStartPar
Par conséquent,
\begin{equation*}
\begin{split}
I(\varphi, v)=I(\varphi, u)
\end{split}
\end{equation*}\end{sphinxadmonition}


\subsection{Propriétés de l’intégrale des fonctions en escalier}
\label{\detokenize{fe:proprietes-de-l-integrale-des-fonctions-en-escalier}}
\begin{sphinxadmonition}{note}{Proposition 4}

\sphinxAtStartPar
Montrer que :

\sphinxAtStartPar
1\sphinxhyphen{} pour toutes fonctions en escalier sur \([a, b]\) \(\varphi\) et \(\psi\), et pour tous réels \(\alpha\) et \(\beta\) nous avons:
\begin{equation*}
\begin{split}
\int_{[a, b]}\alpha\varphi + \beta\psi = \alpha\int_{[a, b]}\varphi + \beta\int_{[a, b]}\psi
\end{split}
\end{equation*}
\sphinxAtStartPar
2\sphinxhyphen{} une fonction en escalier positive a une intégrale positive.

\sphinxAtStartPar
3\sphinxhyphen{} si  \(\varphi\) et \(\psi\) sont deux fonctions en escalier sur \([a, b]\) alors:
\begin{equation*}
\begin{split}
\varphi \leq \psi \Rightarrow  \int_{[a, b]}\varphi \leq \int_{[a, b]}\psi
\end{split}
\end{equation*}\end{sphinxadmonition}

\begin{sphinxadmonition}{note}{Démonstration}

\sphinxAtStartPar
1\sphinxhyphen{}

\sphinxAtStartPar
Soient \(\varphi\) et \(\psi\) deux fonctions en escalier sur \([a, b]\) et \(\alpha\) et \(\beta\) deux réels.

\sphinxAtStartPar
Soit \(u=(x_i)_{i=0}^n\) une subdivision adaptée à  \(\varphi\) et \(\psi\). Si, pour \(i \in \{1, \ldots, n\}\), \(c_i\) et \(d_i\) sont respectivement les valeurs prises par \(\varphi\) et \(\psi\) sur \(]x_{i-1}, x_i[\) alors \(\alpha\varphi + \beta\psi\) est une fonction en escalier qui prend \(\alpha c_i + \beta d_i\) sur \(]x_{i-1}, x_i[\).

\sphinxAtStartPar
Nous avons donc,
\begin{equation*}
\begin{split}
\begin{aligned}
\int_{[a, b]}\alpha\varphi + \beta\psi & = \sum_{i=1}^{n}(\alpha c_i + \beta d_i)(x_i-x_{i-1})   \\ \\
 & =  \alpha\sum_{i=1}^{n} c_i(x_i-x_{i-1}) +  \alpha\sum_{i=1}^{n} d_i(x_i-x_{i-1}) \\ \\
 & = \alpha\int_{[a, b]}\varphi + \beta\int_{[a, b]}\psi
\end{aligned}
\end{split}
\end{equation*}
\sphinxAtStartPar
2\sphinxhyphen{}

\sphinxAtStartPar
Soit \(\varphi\) une fonction en escalier sur \([a, b]\) qui est positive et  \(u=(x_i)_{i=0}^n\) une subdivision adaptée à \(\varphi\). Puisqu’elle est positive, les valeurs \(c_i\) prise par \(\varphi\) sont positives. D’autre part, puisque pour tout \(i \in \{1, \ldots, n\}, x_{i-1} \leq x_i\) (par définition de la subdivision) alors \(\int_{[a, b]}\varphi = \sum_{i=1}^{n} c_i(x_i-x_{i-1}) \geq 0\).

\sphinxAtStartPar
3\sphinxhyphen{}

\sphinxAtStartPar
La fonction \(\varphi - \psi\) est une fonction en escalier positive donc son intégral est positive.
\end{sphinxadmonition}

\begin{sphinxadmonition}{note}{Proposition 5}

\sphinxAtStartPar
Une fonction \(\varphi\) est en escalier sur \([a, b]\) si et seulement si pour tout \(c \in ]a, b[\), ses restrictions sur \(]a, c[\) et \(]c, b[\) le sont. Le cas échéant,
\begin{equation*}
\begin{split}
\int_{[a, b]}\varphi = \int_{[a, c]}\varphi_{|[a, c]} + \int_{[c, b]}\varphi_{|[c, b]}
\end{split}
\end{equation*}\end{sphinxadmonition}

\begin{sphinxadmonition}{note}{Démonstration}
\begin{itemize}
\item {} 
\sphinxAtStartPar
\(\Rightarrow\)
Soit \(\varphi\) une fonction en escalier sur \([a, b]\) et \(u\) une subdivision adaptée à \(\varphi\). On ajoutant \(c\) a la subdivision \(u\) on obtient un subdivision \(v=(x_i)_{i=0}^n\) qui est encore adaptée à \(\varphi\). Soit \(p\) l’entier naturel tel que \(c=x_p\). Alors nous avons :
\begin{itemize}
\item {} 
\sphinxAtStartPar
\((x_1, \ldots, x_p)\) une subdivision de  \([a, c]\) et puisque \(\varphi\) est constante sur chaque \(]x_{i-1}, x_{i}[\) donc \(\varphi_{|[a, c]}\) est fonction en escalier sur \([a, c]\). Nous avons donc :

\end{itemize}
\begin{equation*}
\begin{split}
    \int_{[a, c]}\varphi_{|[a, c]} = \sum_{i=1}^{p} c_i(x_i-x_{i-1})
    \end{split}
\end{equation*}\begin{itemize}
\item {} 
\sphinxAtStartPar
\((x_{p+1}, \ldots, x_n)\) une subdivision de  \([c, b]\) et puisque \(\varphi\) est constante sur chaque \(]x_{i-1}, x_{i}[\) donc \(\varphi_{|[c, b]}\) est fonction en escalier sur \([c, b]\). Nous avons donc :

\end{itemize}
\begin{equation*}
\begin{split}
    \int_{[c, b]}\varphi_{|[c, b]} = \sum_{i=p+1}^{n} c_i(x_i-x_{i-1})
    \end{split}
\end{equation*}
\sphinxAtStartPar
Par suite :
\begin{equation*}
\begin{split}
    \begin{aligned}
    \int_{[a, b]}\varphi &= \sum_{i=1}^{n} c_i(x_i-x_{i-1})=\sum_{i=1}^{p} c_i(x_i-x_{i-1}) + \sum_{i=p+1}^{n} c_i(x_i-x_{i-1}) \\ \\ 
     &= \int_{[a, c]}\varphi_{|[a, c]} + \int_{[c, b]}\varphi_{|[c, b]}
    \end{aligned}
    \end{split}
\end{equation*}
\end{itemize}
\begin{itemize}
\item {} 
\sphinxAtStartPar
\(\Leftarrow\)
Supposons que \(\varphi_{|[a, c]}\) et \(\varphi_{|[c, b]}\) sont en escalier sur \([a, c]\) et \([c, b]\) respectivement.

\end{itemize}

\sphinxAtStartPar
Soit \(u=(x_i)_{i=0}^p\) (respectivement \(v= (x_i)_{i=0}^q\)) une subdivision de \([a, c]\) (respectivement de \([c, b]\)) adaptée à \(\varphi_{|[a, c]}\) (respectivement q \(\varphi_{|[c, b]}\)).

\sphinxAtStartPar
Alors, \((x_0, \ldots, x_{p-1}, x_p=c=y_1, \ldots, y_q)\) est une subdivision \([a, b]\). De plus \(\varphi\) est constante sur chaque intervalle \(]x_{i-1}, x_{i}[\) et \(]y_{i-1}, y_{i}[\). Donc \(\varphi\) est en escalier sur \([a, b]\).
\end{sphinxadmonition}


\section{Fonctions continues par morceaux}
\label{\detokenize{fcm:fonctions-continues-par-morceaux}}\label{\detokenize{fcm::doc}}

\subsection{Définition, exemples}
\label{\detokenize{fcm:definition-exemples}}
\begin{sphinxadmonition}{note}{Définition 4}

\sphinxAtStartPar
Une application \(f\) de \([a, b]\) dans \(\mathbb{R}\) est dite continue par morceaux s’il existe une subdivision \(u=(x_i)_{i=0}^n\) de \([a, b]\) telle que pour chaque \(i \in \{1,\ldots, n\}\) la restriction de \(f\) à \(]x_{i-1}, x_i[\) soit continue et admette des limites finies en \(x_{i-1}\) et \(x_{i}\).
\end{sphinxadmonition}

\sphinxAtStartPar
La subdivision \(u\) est dite adaptée à la fonction \(f\).

\sphinxAtStartPar
L’exemple suivant donne une illustration graphique d’une fonction continue par morceaux.

\sphinxAtStartPar
\sphinxstylestrong{Illustration avec un exemple graphique}:

\noindent{\hspace*{\fill}\sphinxincludegraphics[width=500\sphinxpxdimen]{{fig1}.PNG}\hspace*{\fill}}

\begin{sphinxadmonition}{note}{Exemples}
\begin{itemize}
\item {} 
\sphinxAtStartPar
Toute fonction en escalier est continue par morceaux.

\item {} 
\sphinxAtStartPar
Toute fonction continue est continue par morceaux.

\item {} 
\sphinxAtStartPar
La fonction \(f\) définie sur \([-1, 1]\) par :

\end{itemize}
\begin{equation*}
\begin{split}f(0)=0 \mbox{  et  } \forall x \in [-1, 1] \setminus \{0\}, f(x)=\dfrac{1}{x}
\end{split}
\end{equation*}
\sphinxAtStartPar
n’est pas continue par morceaux, car elle n’a pas de limite finie à droite et à gauche de \(0\).
\end{sphinxadmonition}

\begin{sphinxadmonition}{note}{Remarques}

\sphinxAtStartPar
Comme pour les fonctions en escalier, on peut vérifier que :
\begin{itemize}
\item {} 
\sphinxAtStartPar
si \(u\) est une subdivision adaptée à une fonction \(f\) continue par marceaux,
alors toute subdivision plus fine que \(u\) est adaptée à \(f\),

\item {} 
\sphinxAtStartPar
si \(f\) et \(g\) sont deux fonctions continues par morceaux sur \([a, b]\) , alors il existe
une subdivision adaptée à \(f\) et \(g\).

\end{itemize}
\end{sphinxadmonition}

\begin{sphinxadmonition}{note}{Proposition 6}

\sphinxAtStartPar
Une fonction continue par morceaux sur  \([a, b]\) est bornée sur \([a, b]\) .
\end{sphinxadmonition}

\begin{sphinxadmonition}{note}{Démonstration}

\sphinxAtStartPar
Soit \(f\) une fonction continue par morceaux sur  \([a, b]\). Soit \(u=(x_i)_{i=0}^n\) une subdivision adaptée à \(f\).

\sphinxAtStartPar
Pour chaque \(i \in \{1, \ldots, n\}\) admet des limites finies en \(x_{i-1}\) et \(x_i\). Donc la restriction de \(f\) sur \(]x_{i-1}, x_i[\) admet un prolongement par continuité sur \([x_{i-1}, x_i]\) qui donc borne. Par suite, \(f\) est bornée sur \(]x_{i-1}, x_i[\). Soit \(M_i = \sup_{]x_{i-1}, x_i[}|f|\)

\sphinxAtStartPar
En prenant \(M = max (M_1, M_2, \ldots, M_n, |f(x_0)|, \ldots, |f(x_0)|)\), nous aurons \(\forall x \in [a, b], |f(x)| \leq M\).

\sphinxAtStartPar
Par suite, \(f\) est bornée sur \([a, b]\).
\end{sphinxadmonition}

\begin{sphinxadmonition}{note}{Proposition 7}

\sphinxAtStartPar
Soient \(f\) et \(g\) deux fonctions continues par morceaux sur \([a,b]\). Les assertions suivantes sont correctes :
\begin{itemize}
\item {} 
\sphinxAtStartPar
\( \forall \lambda, \mu \in \mathbb R, \lambda f + \mu g\) est continue par morceaux.

\item {} 
\sphinxAtStartPar
\(fg\) est continue par morceaux.

\end{itemize}
\end{sphinxadmonition}

\begin{sphinxadmonition}{note}{Démonstration}

\sphinxAtStartPar
Soit \(u=(x_i)_{i=0}^n\) une subdivision adaptée à \(f\) et \(g\).

\sphinxAtStartPar
Les restrictions des fonctions \(f\) et \(g\) à chacun des intervalles \( ]x_{i-1}, x_i[\) sont continues et admettent des limites finies en  \(x_{i-1}\) et \(x_i\), donc il en est de même pour \(\lambda f + \mu g\) et \(fg\). Les fonctions
\(\lambda f + \mu g\) et \(fg\) sont donc continues par morceaux sur \([a, b]\).
\end{sphinxadmonition}


\subsection{Intégrale d’une fonction continue par morceaux}
\label{\detokenize{fcm:integrale-d-une-fonction-continue-par-morceaux}}
\begin{sphinxadmonition}{note}{Théorème (admis)}

\sphinxAtStartPar
Soit \(f\) une fonction continue par morceaux sur le segment \([a, b]\) . Pour tout
réel \(\epsilon > O\):
\begin{itemize}
\item {} 
\sphinxAtStartPar
il existe une fonction en escalier \(\theta\) telle que \(|f - \theta| \leq \epsilon\)

\item {} 
\sphinxAtStartPar
il existe des fonctions en escalier \(\varphi\) et \(\psi\) telles que :

\end{itemize}
\begin{equation*}
\begin{split}
\varphi \leq f \leq \psi \mbox{   et  } \psi - \varphi \leq \epsilon
\end{split}
\end{equation*}\end{sphinxadmonition}

\sphinxAtStartPar
\sphinxstylestrong{Notations}: Dans ce qui suit, pour une fonction continue par morceaux \(f\) nous allons adopte les notations suivantes :
\begin{itemize}
\item {} 
\sphinxAtStartPar
\(\mathcal{E}^+(f)\) l’ensemble des fonctions en escalier plus grandes que \(f\).

\item {} 
\sphinxAtStartPar
\(\mathcal{E}^-(f)\) l’ensemble des fonctions en escalier plus petites que \(f\).

\end{itemize}

\begin{sphinxadmonition}{note}{Proposition 8}

\sphinxAtStartPar
Soit \(f\) une fonction continue par morceaux sur le segment \([a, b]\). Alors :
\begin{itemize}
\item {} 
\sphinxAtStartPar
\(\left\{\int_{[a, b]} \varphi | \varphi \in \mathcal{E}^-(f)\right\}\) admet une borne supérieure,

\item {} 
\sphinxAtStartPar
\(\left\{\int_{[a, b]} \psi | \psi \in \mathcal{E}^+(f)\right\}\) admet une borne inferieure,

\end{itemize}

\sphinxAtStartPar
de plus,
\begin{equation*}
\begin{split}
sup\left\{\int_{[a, b]} \varphi | \varphi \in \mathcal{E}^-(f)\right\}= inf\left\{\int_{[a, b]} \psi | \psi \in \mathcal{E}^+(f)\right\}
\end{split}
\end{equation*}\end{sphinxadmonition}

\begin{sphinxadmonition}{note}{Démonstration}

\sphinxAtStartPar
Soit \(f\) une fonction continue par morceaux sur \([a, b]\). Donc \(\left\{f(x)|x\in [a, b]\right\}\) est une partie non vide de \(\mathbb R\) bornée donc admet une borne supérieure et une borne inferieure. Soit \(m=\inf_{[a, b]} f\) et \(M=\sup_{[a, b]} f\).
\begin{itemize}
\item {} 
\sphinxAtStartPar
Les deux fonctions constantes \(m\) et \(M\) sur \([a, b]\) sont aussi continues par morceaux sur \([a, b]\). \(\left\{\int_{[a, b]} \varphi | \varphi \in \mathcal{E}^-(f)\right\}\) est donc une partie de \(\mathbb R\) non vide majorée (par \(M(b-a)\)). Alors, elle possède une borne supérieure (\(\alpha\)). De même, \(\left\{\int_{[a, b]} \psi | \psi \in \mathcal{E}^+(f)\right\}\) une partie de \(\mathbb R\) non vide minorée donc possède une borne inferieure (\(\beta\)).

\item {} 
\sphinxAtStartPar
Toute fonction \(\varphi \in \mathcal{E}^-(f)\) est inférieure à toute fonction \(\psi \in \mathcal{E}^+(f)\). Par suite :

\end{itemize}
\begin{equation*}
\begin{split}
\int_{[a, b]} \varphi \leq \int_{[a, b]} \psi
\end{split}
\end{equation*}
\sphinxAtStartPar
Fixons \(\psi \in \mathcal{E}^+(f)\). L’ensemble \(\left\{\int_{[a, b]} \varphi | \varphi \in \mathcal{E}^-(f)\right\}\) est majoreé par \(\int_{[a, b]} \psi\). Alors nous avons forcement \(\alpha \leq \int_{[a, b]} \psi\) et ça pour tout \(\psi \in \mathcal{E}^+(f)\). Donc \(\alpha\) est un minorant de \(\left\{\int_{[a, b]} \psi | \psi \in \mathcal{E}^+(f)\right\}\). Par conséquent, \(\alpha \leq \beta\).

\sphinxAtStartPar
Donc
\begin{equation*}
\begin{split}
\alpha \leq \beta
\end{split}
\end{equation*}\begin{itemize}
\item {} 
\sphinxAtStartPar
soit \(\epsilon >0\). En utilisant le théorème précédant, il existe deux fonctions en escalier \(\varphi \in \mathcal{E}^-(f)\) et \(\psi \in \mathcal{E}^+(f)\)  tel que \(\psi - \varphi \leq \epsilon\).

\end{itemize}

\sphinxAtStartPar
Donc \(\int_{[a, b]} \psi - \int_{[a, b]} \varphi \leq \int_{[a, b]} \epsilon = \epsilon(b-a)\)

\sphinxAtStartPar
donc \(0 \leq \beta - \alpha \leq \epsilon(b-a)\)

\sphinxAtStartPar
Et ca pour tout \(\epsilon \geq 0\).

\sphinxAtStartPar
Donc \(\alpha = \beta\).
\end{sphinxadmonition}

\begin{sphinxadmonition}{note}{Définition}

\sphinxAtStartPar
Soit \(f\) une fonction continue par morceaux sur le segment \([a, b]\).

\sphinxAtStartPar
On appelle intégrale de \(f\) sur \([a, b]\) le réel
\begin{equation*}
\begin{split}
\int_{[a, b]} f = sup\left\{\int_{[a, b]} \varphi | \varphi \in \mathcal{E}^-(f)\right\}= inf\left\{\int_{[a, b]} \psi | \psi \in \mathcal{E}^+(f)\right\}
\end{split}
\end{equation*}\end{sphinxadmonition}

\sphinxAtStartPar
\sphinxstylestrong{Question} : Comparer l’intégrale d’une fonction en escalier avec son intégral en tant que fonction continue par morceaux.

\begin{sphinxadmonition}{note}{Indications pour la réponse}
\begin{itemize}
\item {} 
\sphinxAtStartPar
Une fonction en escalier est continue par morceaux ;

\item {} 
\sphinxAtStartPar
si \(f\) est une fonction en escalier, alors \(f \in \left\{\int_{[a, b]} \varphi | \varphi \in \mathcal{E}^-(f)\right\}\) et \( f \in \left\{\int_{[a, b]} \psi | \psi \in \mathcal{E}^+(f)\right\}\).

\end{itemize}
\end{sphinxadmonition}

\sphinxAtStartPar
Nous avons vu que si \(f\) une fonction continue par morceaux sur \([a, b]\), alors \(f\) est bornée. Soient \(m = inf \left\{f(x)| x\in [a, b]\right\}\) et \(M = sup \left\{f(x)| x\in [a, b]\right\}\). \(m\) et \(M\) sont donc des fonctions en escalier sur \([a, b]\). Leurs intégrales sont respectivement \(m(b-a)\) et \(M(b-a)\).

\sphinxAtStartPar
Et puisque \(m \leq f \leq M\), nous avons \(m(b-a)\leq \int_{[a,b]} f \leq M(b-a)\).

\sphinxAtStartPar
Donc, la quantité \(\dfrac{1}{b-a}\int_{[a, b]} f\) est comprise entre \(m\) et \(M\).

\begin{sphinxadmonition}{note}{Définition}

\sphinxAtStartPar
Soit \(f\) une fonction continue par morceaux sur \([a, b]\). La quantité \(\dfrac{1}{b-a}\int_{[a, b]} f\) s’appelle \sphinxstylestrong{la valeur moyenne} de \(f\).
\end{sphinxadmonition}


\section{Propriétés de l’intégrale}
\label{\detokenize{pptint:proprietes-de-l-integrale}}\label{\detokenize{pptint::doc}}
\begin{sphinxadmonition}{note}{Proposition (linéarité)}

\sphinxAtStartPar
Soient \(f_1, f_2\) deux fonctions continues par morceaux sur \([a, b]\), \(\lambda_1, \lambda_2 \in \mathbb R\). Alors :
\begin{equation*}
\begin{split}
\int_{[a, b]} (\lambda_1f_1 + \lambda_2f_2) = \lambda_1 \int_{[a, b]} f_1 + \lambda_2 \int_{[a, b]} f_2
\end{split}
\end{equation*}\end{sphinxadmonition}

\sphinxAtStartPar
On dit que l’intégrale est une forme linéaire sur l’espace vectoriel des fonctions continues par morceaux sur \([a, b]\).

\begin{sphinxadmonition}{note}{Démonstration}
\begin{itemize}
\item {} 
\sphinxAtStartPar
Soient \(\epsilon>0\) et \(f\) est une fonction continue par morceaux, alors il existe \(\theta\) est fonction en escalier telle que \(|f-\theta|\leq \epsilon\). On a \(\theta - \epsilon \leq f \leq \theta + \epsilon\). Alors

\end{itemize}
\begin{equation*}
\begin{split}
 \int_{[a, b]} (\theta - \epsilon) \leq  \int_{[a, b]} f \leq  \int_{[a, b]} (\theta + \epsilon)
 \end{split}
\end{equation*}
\sphinxAtStartPar
et donc
\begin{equation*}
\begin{split}
\left|\int_{[a, b]} f- \int_{[a, b]} \theta \right| \leq (b-a)\epsilon
\end{split}
\end{equation*}\begin{itemize}
\item {} 
\sphinxAtStartPar
Maintenant, soient \(f_1\) et \(f_2\) deux fonctions continues par morceaux sur \([a, b]\), ainsi que \(\lambda_1, \lambda_2\) deux réels.

\end{itemize}

\sphinxAtStartPar
Pour \(\epsilon>0\) quelconque, il existe \(\theta_1, \theta_2\) deux fonctions en escalier telles que :
\begin{equation*}
\begin{split}
|f_1 - \theta_1| \leq \epsilon ~~~ \mbox{ et } ~~~ |f_2 - \theta_2| \leq \epsilon
\end{split}
\end{equation*}
\sphinxAtStartPar
Ce qui entraine
\begin{equation*}
\begin{split}
\left|\int_{[a, b]} f_1- \int_{[a, b]} \theta_1 \right| \leq (b-a)\epsilon
\end{split}
\end{equation*}
\sphinxAtStartPar
et
\begin{equation*}
\begin{split}
\left|\int_{[a, b]} f_2- \int_{[a, b]} \theta_2 \right| \leq (b-a)\epsilon
\end{split}
\end{equation*}
\sphinxAtStartPar
Donc
\begin{equation*}
\begin{split}
|\lambda_1|\left|\int_{[a, b]} f_1- \int_{[a, b]} \theta_1 \right| \leq |\lambda_1|(b-a)\epsilon
\end{split}
\end{equation*}
\sphinxAtStartPar
et
\begin{equation*}
\begin{split}
|\lambda_2|\left|\int_{[a, b]} f_2- \int_{[a, b]} \theta_2 \right| \leq |\lambda_2|(b-a)\epsilon
\end{split}
\end{equation*}
\sphinxAtStartPar
Donc (puisque l’intégrale sur les fonctions en escalier est linéaire)
\begin{equation*}
\begin{split}
\left|\lambda_1\int_{[a, b]} f_1- \int_{[a, b]} \lambda_1\theta_1 \right| \leq |\lambda_1|(b-a)\epsilon
\end{split}
\end{equation*}
\sphinxAtStartPar
et
\begin{equation*}
\begin{split}
\left|\lambda_2\int_{[a, b]} f_2- \int_{[a, b]} \lambda_2\theta_2 \right| \leq |\lambda_2|(b-a)\epsilon
\end{split}
\end{equation*}
\sphinxAtStartPar
Par suite
\begin{equation*}
\begin{split}
\left|\lambda_1\int_{[a, b]} f_1 +\lambda_2\int_{[a, b]} f_2 - \left(\int_{[a, b]} \lambda_1\theta_1 + \int_{[a, b]} \lambda_2\theta_2\right)\right| \leq (|\lambda_1|+|\lambda_2|)(b-a)\epsilon
\end{split}
\end{equation*}
\sphinxAtStartPar
On pose
\begin{equation*}
\begin{split}
f= \lambda_1 f_1 + \lambda_2 f_2 ~~ \mbox{ et } \theta = \lambda_1 \theta_1 + \lambda_2 \theta_2
\end{split}
\end{equation*}
\sphinxAtStartPar
On a
\begin{equation*}
\begin{split}
|f-\theta| \leq |\lambda_1||f_1-\theta_1| + |\lambda_2||f_2 - \theta_2| \leq (|\lambda_1|+|\lambda_2|)\epsilon
\end{split}
\end{equation*}
\sphinxAtStartPar
Et par suite
\begin{equation*}
\begin{split}
\left|\int_{[a, b]} f- \int_{[a, b]} \theta \right| \leq (b-a)(|\lambda_1|+|\lambda_2|)\epsilon
\end{split}
\end{equation*}
\sphinxAtStartPar
On pose
\begin{equation*}
\begin{split}
I = \int_{[a, b]} \theta = \int_{[a, b]} \lambda_1\theta_1 + \int_{[a, b]} \lambda_2\theta_2
\end{split}
\end{equation*}
\sphinxAtStartPar
et
\begin{equation*}
\begin{split}
\Delta = \left|\int_{[a, b]} f- \left(\lambda_1\int_{[a, b]} f_1 +\lambda_2\int_{[a, b]} f_2\right)\right|
\end{split}
\end{equation*}
\sphinxAtStartPar
Alors
\begin{equation*}
\begin{split}
\begin{aligned}
\Delta &=& \left|\int_{[a, b]} f- I + I - \left(\lambda_1\int_{[a, b]} f_1 +\lambda_2\int_{[a, b]} f_2\right)\right| \\ \\
& \leq & \left|\int_{[a, b]} f- I\right| + \left| I - \left(\lambda_1\int_{[a, b]} f_1 +\lambda_2\int_{[a, b]} f_2\right)\right| \\ \\
& \leq & 2(b-a)(|\lambda_1|+|\lambda_2|)\epsilon
\end{aligned}
\end{split}
\end{equation*}
\sphinxAtStartPar
On en déduit
\begin{equation*}
\begin{split}
\forall \epsilon >0, \Delta \leq 2(b-a)(|\lambda_1|+|\lambda_2|)\epsilon
\end{split}
\end{equation*}
\sphinxAtStartPar
Ce qui prouve que \(\Delta =0\) et donc
\begin{equation*}
\begin{split}
 \left|\int_{[a, b]} f = \lambda_1\int_{[a, b]} f_1 +\lambda_2\int_{[a, b]} f_2\right|
\end{split}
\end{equation*}\end{sphinxadmonition}

\sphinxAtStartPar
Deux fonctions continues par morceaux sur l’intervalle \([a. b]\)  qui sont égales sauf en un nombre fini de points ont la même intégrale car leur différence, qui est nulle sauf
en un nombre fini de points, est une fonction en escalier dont l’intégrale est nulle.

\begin{sphinxadmonition}{note}{Proposition (relation de Chasles)}

\sphinxAtStartPar
Soit \(c \in [a, b]\) et \(f\) une fonction définie sur \([a, b]\).

\sphinxAtStartPar
La fonction \(f\) est continue par morceaux sur \([a, b]\) si, et seulement si, ses
restrictions à \([a, c]\) et à \([c, b]\) sont continues par morceaux, et l’on a alors :
\begin{equation*}
\begin{split}
\int_{[a, b]} f = \int_{[a, c]} f_{|[a, c]}  + \int_{[c, b]} f_{|[c, b]}
\end{split}
\end{equation*}\end{sphinxadmonition}

\begin{sphinxadmonition}{note}{Démonstration}

\sphinxAtStartPar
Soit \(\varphi \in \mathcal E^-(f)\) (une fonction en escalier plus petite que \(f\)).

\sphinxAtStartPar
On a \(\varphi_{|[a, c]} \in \mathcal E^-(f_{|[a, c]})\) et \(\varphi_{|[c, b]} \in \mathcal E^-(f_{|[c, b]})\).

\sphinxAtStartPar
Par suite
\begin{equation*}
\begin{split}
\begin{aligned}
\int_{[a, b]}\varphi &=& \int_{[a, c]} \varphi_{|[a, c]} + \int_{[c, b]} \varphi_{|[c, b]} \\ \\
&\leq& \int_{[a, c]} f_{|[a, c]} + \int_{[c, b]} f_{|[c, b]}
\end{aligned}
\end{split}
\end{equation*}
\sphinxAtStartPar
Le réel \(\int_{[a, c]} f_{|[a, c]} + \int_{[c, b]} f_{|[c, b]}\) est un majorant de de \(\left\{\int_{[a, b]}\varphi ~~|~~ \varphi \in \mathcal E^-(f) \right\}\).

\sphinxAtStartPar
Donc il est plus grands que sa borne supérieure (\(\int_{[a, b]} f\)).
Ce qui donne
\begin{equation*}
\begin{split}
\int_{[a, b]} f \leq \int_{[a, c]} f_{|[a, c]} + \int_{[c, b]} f_{|[c, b]}
\end{split}
\end{equation*}
\sphinxAtStartPar
En applique les mêmes étapes pour \(-f\)

\sphinxAtStartPar
Soit \(\varphi \in \mathcal E^-(-f)\) (une fonction en escalier plus petite que \(-f\)).

\sphinxAtStartPar
On a \(\varphi_{|[a, c]} \in \mathcal E^-(-f_{|[a, c]})\) et \(\varphi_{|[c, b]} \in \mathcal E^-(-f_{|[c, b]})\).

\sphinxAtStartPar
Par suite
\begin{equation*}
\begin{split}
\begin{aligned}
\int_{[a, b]}\varphi &=& \int_{[a, c]} \varphi_{|[a, c]} + \int_{[c, b]} \varphi_{|[c, b]} \\ \\
&\leq& \int_{[a, c]} -f_{|[a, c]} + \int_{[c, b]} -f_{|[c, b]}
\end{aligned}
\end{split}
\end{equation*}
\sphinxAtStartPar
Le reel \(\int_{[a, c]} -f_{|[a, c]} + \int_{[c, b]} -f_{|[c, b]}\) est un majorant de de \(\left\{\int_{[a, b]}\varphi ~~|~~ \varphi \in \mathcal E^-(-f) \right\}\).

\sphinxAtStartPar
Donc il est plus grands que sa borne supérieure (\(\int_{[a, b]} -f\)).
Ce qui donne
\begin{equation*}
\begin{split}
\int_{[a, b]} -f \leq \int_{[a, c]} -f_{|[a, c]} + \int_{[c, b]} -f_{|[c, b]}
\end{split}
\end{equation*}
\sphinxAtStartPar
Et d’après la linéarité de l’intégrale nous avons
\begin{equation*}
\begin{split}
\int_{[a, b]} f \geq \int_{[a, c]} f_{|[a, c]} + \int_{[c, b]} f_{|[c, b]}
\end{split}
\end{equation*}
\sphinxAtStartPar
En fin
\begin{equation*}
\begin{split}
\int_{[a, b]} f = \int_{[a, c]} f_{|[a, c]} + \int_{[c, b]} f_{|[c, b]}
\end{split}
\end{equation*}\end{sphinxadmonition}

\sphinxAtStartPar
\sphinxstylestrong{Remarque}:

\sphinxAtStartPar
Soient \(f\) une fonction continue par morceaux sur \([a, b]\), et \(u=(x_i)_{i\in\{1,\ldots,n\}}\) une subdivision adaptée à \(f\). Pour \(i \in \{1,\ldots,n\}\), notons \(f_i\) la fonction continue sur \([x_{i-1}, x_i]\) tel que \(\forall x \in ]x_{i-1}, x_i[, f_i(x) = f(x)\). Alors d’après la relation de Chasles :
\begin{equation*}
\begin{split}
\int_{[a, b]} f = \sum_{i=1}^n \int_{[x_{i-1}, x_i]}f = \sum_{i=1}^n \int_{[x_{i-1}, x_i]}f_i
\end{split}
\end{equation*}

\subsection{Quelques inégalités}
\label{\detokenize{pptint:quelques-inegalites}}
\begin{sphinxadmonition}{note}{Proposition}
\begin{itemize}
\item {} 
\sphinxAtStartPar
Une fonction positive et continue par morceaux à une intégrale positive.

\item {} 
\sphinxAtStartPar
Si \(f\) et \(g\) sont deux fonctions continues par morceaux sur \([a, b]\) alors:

\end{itemize}
\begin{equation*}
\begin{split}
f \leq g \Rightarrow \int_{[a, b]}f \leq \int_{[a, b]}g 
\end{split}
\end{equation*}\end{sphinxadmonition}

\begin{sphinxadmonition}{note}{Démonstration}
\begin{itemize}
\item {} 
\sphinxAtStartPar
Si \(f\) est une fonction continue par morceaux et positive alors la fonction nulle (qui constante donc en escalier) appartient à \(\mathcal E^-(f)\). Donc son intégrale (qui vaut 0) est inférieure à l’intégrale de \(f\). Par suite \(\int_{[a, b]} f \geq 0\).

\item {} 
\sphinxAtStartPar
On applique le résultat précèdent à \(g-f\) puis on utilise la linéarité de l’intégrale.

\end{itemize}
\end{sphinxadmonition}

\begin{sphinxadmonition}{note}{Théorème}

\sphinxAtStartPar
Si \(f\) est continue par morceaux sur \([a, b]\), alors \(|f|\) est continue par morceaux sur \([a, b]\) et:
\begin{equation*}
\begin{split}
\left |\int_{[a, b]}f \right | \leq \int_{[a, b]}|f|
\end{split}
\end{equation*}\end{sphinxadmonition}

\begin{sphinxadmonition}{note}{Démonstration}

\sphinxAtStartPar
Soient \(f\) une fonction continue par morceaux sur \([a, b]\) et \(u=(x_i)_{i=0}^n\) une subdivision adaptée à \(f\). Alors, pour tout \(i \in \{1, \ldots, n\}\) la restriction de \(f\) sur chacun des intervalles \(]x_{i-1}, x_i[\) est continue et admet des limites finies en \(x_{i-1}\) et \(x_i\). Il en est de même pour \(|f|\) d’après les propriétés des limites. Donc \(|f|\) est une fonction continue par morceaux sur \([a, b]\).

\sphinxAtStartPar
Nous avons \(-|f| \leq f \leq |f|\) donc
\begin{equation*}
\begin{split}
- \int_{[a, b]}|f| \leq \int_{[a, b]}f \leq \int_{[a, b]}|f|  
\end{split}
\end{equation*}
\sphinxAtStartPar
Ce qui veut dire
\begin{equation*}
\begin{split}
\left| \int_{[a, b]}f \right | \leq \int_{[a, b]}|f|  
\end{split}
\end{equation*}\end{sphinxadmonition}

\begin{sphinxadmonition}{note}{Proposition (Inégalité de la moyenne)}

\sphinxAtStartPar
Si \(f\) et \(g\) sont deux fonctions continues par morceaux sur \([a, b]\), alors:
\begin{equation*}
\begin{split}
\left |\int_{[a, b]}fg \right | \leq \sup_{[a, b]} |f| \int_{[a, b]}|g|
\end{split}
\end{equation*}\end{sphinxadmonition}

\begin{sphinxadmonition}{note}{Démonstration}

\sphinxAtStartPar
Soit \(M = sup_{[a,b]}|f|\).
Nous avons
\begin{equation*}
\begin{split}
\forall x \in [a, b], |f(x)g(x)|=|f(x)||g(x)| \leq M |g(x)|
\end{split}
\end{equation*}
\sphinxAtStartPar
Donc
\begin{equation*}
\begin{split}
\left |\int_{[a, b]}fg \right | \leq \int_{[a, b]}|fg| \leq \int_{[a, b]}M|g| =M\int_{[a, b]}|g|
\end{split}
\end{equation*}\end{sphinxadmonition}

\begin{sphinxadmonition}{note}{Corollaire}

\sphinxAtStartPar
Si \(f\) est une fonction continue par morceaux sur \([a, b]\), alors:
\begin{equation*}
\begin{split}
\left |\int_{[a, b]}f \right | \leq (b-a)  \sup_{[a, b]}|f|
\end{split}
\end{equation*}\end{sphinxadmonition}

\begin{sphinxadmonition}{note}{Démonstration}

\sphinxAtStartPar
En pose \(g=1\) est on applique l’inégalité de la moyenne.
\end{sphinxadmonition}

\begin{sphinxadmonition}{note}{Théorème (Inégalité de Cauchy\sphinxhyphen{}Schwarz)}

\sphinxAtStartPar
Si \(f\) et \(g\) sont deux fonctions continues par morceaux sur \([a, b]\), alors:
\begin{equation*}
\begin{split}
\left (\int_{[a, b]}fg \right ) ^2 \leq \int_{[a, b]} f^2 \int_{[a, b]}g^2
\end{split}
\end{equation*}\end{sphinxadmonition}

\begin{sphinxadmonition}{note}{Démonstration}

\sphinxAtStartPar
On pose
\begin{equation*}
\begin{split}
P(\lambda) = \int_{[a, b]} (f+\lambda g)^2 = \lambda^2 \int_{[a, b]} g^2 + 2\lambda \int_{[a, b]} fg + \int_{[a, b]} f^2
\end{split}
\end{equation*}
\sphinxAtStartPar
\(P\) est donc une fonction polynomiale de degré au plus égale à 2 qui est positive pour tout \(\lambda \in \mathbb R\).
\begin{itemize}
\item {} 
\sphinxAtStartPar
Si \(\int_{[a, b]} g^2 = 0\), alors la fonction \(P\) ne peut pas changer de signe (car elle est positive) donc ne peut pas être de degré 1. On en déduit \(\int_{[a, b]} fg = 0\)

\end{itemize}

\sphinxAtStartPar
Donc \(\left (\int_{[a, b]} fg\right)^2 = 0=\int_{[a, b]} f^2 \int_{[a, b]} g^2\)
\begin{itemize}
\item {} 
\sphinxAtStartPar
Sinon, \(P\) est polynôme de degré 2 qui positif pour tout \(\lambda \in \mathbb R\). Son discriminent

\end{itemize}
\begin{equation*}
\begin{split}
\Delta = 4 \left (\left (\int_{[a, b]} fg\right)^2- \int_{[a, b]} f^2 \int_{[a, b]} g^2 \right)
\end{split}
\end{equation*}
\sphinxAtStartPar
est donc negatif. Ce donne
\begin{equation*}
\begin{split}
\left (\int_{[a, b]} fg\right)^2 \leq \int_{[a, b]} f^2 \int_{[a, b]} g^2
\end{split}
\end{equation*}\end{sphinxadmonition}

\sphinxAtStartPar
On peut écrire l’inégalité de Cauchy\sphinxhyphen{}Schwarz comme suit :
\begin{equation*}
\begin{split}
\left |\int_{[a, b]}fg \right | \leq \left (\int_{[a, b]} f^2 \right)^{\frac{1}{2}}  \left (\int_{[a, b]}g^2\right)^{\frac{1}{2}} 
\end{split}
\end{equation*}

\subsection{Cas des fonctions continues}
\label{\detokenize{pptint:cas-des-fonctions-continues}}
\begin{sphinxadmonition}{note}{Théorème}

\sphinxAtStartPar
Une fonction \sphinxstylestrong{continue et positive} sur \([a, b]\) est nulle si, et seulement si, son intégrale sur \([a, b]\) est nulle.
\end{sphinxadmonition}

\begin{sphinxadmonition}{note}{Démonstration}

\sphinxAtStartPar
Soit \(f\) une fonction continue et positive sur \([a, b]\).

\sphinxAtStartPar
Si \(f\) est nulle alors son intégrale est nulle.

\sphinxAtStartPar
Montrons maintenant, que si l’intégrale de \(f\) est nulle alors \(f\) est la fonction nulle.

\sphinxAtStartPar
Par absurde, on suppose que \(f\) n’est pas nulle. Donc il existe (au moins) \(c \in [a, b]\) tel que \(f(c) \neq 0\) et puisque \(f\) est positive \(f(c) > 0\).

\sphinxAtStartPar
1\sphinxhyphen{} Si \(c \in ]a, b[\). Puisque la fonction est continue en \(c\) alors pour tout \(\epsilon >0\) il existe \(\alpha > 0\) tel que \(\forall x \in [a, b],~~ |x-c| \leq \alpha \Rightarrow |f(x) - f(x)|\leq \epsilon\).

\sphinxAtStartPar
On pose \(\epsilon = \dfrac{f(x)}{2}\)

\sphinxAtStartPar
alors il existe \(\alpha>0\) tel que pour tout \(x\in [a, b], ~~ x \in ]c-\alpha, c +\alpha[ \Rightarrow f(x)\leq f(c)-\epsilon > 0\).

\sphinxAtStartPar
On pose \(\beta = max(\dfrac{a+c}{2}, c-\alpha), \gamma = max(\dfrac{b+c}{2}, c+\alpha)\). On a
\begin{equation*}
\begin{split}
\begin{aligned}
\int_{[a, b]}f &= \int_{[a, \beta]}f + \int_{[\beta, \gamma]}f + \int_{[\gamma, b]}f \\ \\
&\geq  \int_{[\beta, \gamma]}f \\ \\
&\geq  (\gamma -\beta )\dfrac{f(c)}{2}> 0
\end{aligned}
\end{split}
\end{equation*}
\sphinxAtStartPar
2\sphinxhyphen{} Si \(c =a\) ou \(c=b\). La fonction \(f\) est continue en \(c\) et \(f(c)>0\). Donc elle est strictement positive au voisinage de \(c\). Ceci dit, il existe un réel \(d \in ]a, b[\) tel f(d)>0. On répète les mêmes étapes précédentes mais cette fois avec \(d\) et on aboutit a \(\int_{[a, b]}f >0\)

\sphinxAtStartPar
Ce qui est absurde.

\sphinxAtStartPar
Donc \(f\) est nulle.
\end{sphinxadmonition}

\begin{sphinxadmonition}{warning}{Avertissement:}
\sphinxAtStartPar
Les deux hypothèses (continuité et positivité) sont nécessaires pour que le résultat soit vrai.
\end{sphinxadmonition}

\begin{sphinxadmonition}{note}{Corollaire}

\sphinxAtStartPar
Si \(f\) et \(g\) sont deux fonctions continues sur \([a, b]\), alors:
\begin{equation*}
\begin{split}
\left (\int_{[a, b]}fg \right ) ^2 = \int_{[a, b]} f^2 \int_{[a, b]}g^2
\end{split}
\end{equation*}
\sphinxAtStartPar
Si et seulement si, \(f\) et \(g\) sont proportionnelles.
\end{sphinxadmonition}

\begin{sphinxadmonition}{note}{Démonstration}
\begin{itemize}
\item {} 
\sphinxAtStartPar
Si \(f\) et \(g\) sont proportionnelles, donc il existe \(\lambda \in \mathbb R\) tel que \(f=\lambda g\) où  \(g=\lambda f\). Supposons par exemple que \(f=\lambda g\).

\end{itemize}

\sphinxAtStartPar
Alors,
\begin{equation*}
\begin{split}
\left(\int_{[a, b]} fg \right)^2 = \lambda^2\left(\int_{[a, b]} g^2\right)^2 = \int_{[a, b]} f^2\int_{[a, b]} g^2
\end{split}
\end{equation*}
\sphinxAtStartPar
Supposons maintenant que
\begin{equation*}
\begin{split}
\left (\int_{[a, b]}fg \right ) ^2 = \int_{[a, b]} f^2 \int_{[a, b]}g^2
\end{split}
\end{equation*}
\sphinxAtStartPar
On pose
\begin{equation*}
\begin{split}
P(\lambda) = \int_{[a, b]}(f+\lambda g)^2 = \lambda^2\int_{[a, b]}g^2 + 2\lambda \int_{[a, b]}fg + \int_{[a, b]}f^2
\end{split}
\end{equation*}\begin{itemize}
\item {} 
\sphinxAtStartPar
\(si \int_{[a, b]}g^2 = 0\) alors \(g^2\) est nulle (puisqu’elle est continue est positive avec intégrale nulle) donc \(g\) est nulle. Donc \(g= 0\times f\). Par suite \(f\) et \(g\) sont proportionnelles.

\item {} 
\sphinxAtStartPar
sinon, le polynôme \(P\) a un discriminent nul. Donc il existe \(\lambda_0\) tel que \(P(\lambda_0)=0\).

\end{itemize}

\sphinxAtStartPar
Donc \(P(\lambda_0) = \int_{[a, b]}(f+\lambda_0 g)^2 = 0\). La fonction \((f+\lambda_0 g)^2\) est continue et positive avec intégrale nulle donc elle est nulle. Par suite \(f=\lambda_0 g\). Les deux fonctions sont proportionnelles.
\end{sphinxadmonition}


\subsection{Invariance par translation}
\label{\detokenize{pptint:invariance-par-translation}}
\begin{sphinxadmonition}{note}{Proposition}

\sphinxAtStartPar
Soient \(f\) une fonction continue par morceaux sur \([a, b]\) et \(\alpha\) un réel.
La fonction \(f_\alpha\) définie sur \([a+\alpha, b+\alpha]\) par \(f_\alpha(x)f(x-\alpha)\) est continue par morceaux sur \([a+\alpha, b+\alpha]\). De plus :
\begin{equation*}
\begin{split}
\int_{[a, b]}f= \int_{[a+\alpha, b+\alpha]} f_{\alpha}
\end{split}
\end{equation*}\end{sphinxadmonition}

\begin{sphinxadmonition}{note}{Démonstration}

\sphinxAtStartPar
On va dabord montrer le resultat pour une fonction en escalier pouis, pour une fonction continue par morceaux.
\begin{itemize}
\item {} 
\sphinxAtStartPar
Si \(f\) est une fonction en escalier sur \([a, b]\):

\end{itemize}

\sphinxAtStartPar
soit \(u=(x_i)_{i=0}^n\) une subdivision adaptee a \(f\). Alors on peut facilement montrer que la famille \((y_i)_{i=0}^n\) avec \(y_i = x_i +\alpha\) est une subdivision de \([a+\alpha, b+\alpha]\).

\sphinxAtStartPar
Si \(f\) prend la valeur \(c_i\) sur \(]x_{i-1}, x_i[\) alors \(f_\alpha\) vaut aussi \(c_i\) sur \(]x_{i-1}+\alpha,  x_i+\alpha[\). Donc \(f_\alpha\) est une fonction en escalier sur \([a+\alpha, b+\alpha]\).

\sphinxAtStartPar
De plus,
\begin{equation*}
\begin{split}
\int_{[a+\alpha, b+\alpha]} f_\alpha = \sum_{i=1}^n (y_i - y_{i-1})c_i = \sum_{i=1}^n (x_i - x_{i-1})c_i = \int_{[a, b]} f
\end{split}
\end{equation*}
\sphinxAtStartPar
Maintenant si \(f\) est continue par morceaux,

\sphinxAtStartPar
soit \(u=(x_i)_{i=0}^n\) une subdivision adaptee a \(f\). La famille \((y_i)_{i=0}^n\) avec \(y_i = x_i +\alpha\) est une subdivision de \([a+\alpha, b+\alpha]\).

\sphinxAtStartPar
\(f\) est continue sur \(]x_{i-1}, x_i[\) est admet des limites finies en \(x_{i-1}\) et \(x_{i}\). Donc, \(f_\alpha\) est continue sur \(]y_{i-1}, y_i[=]x_{i-1}+\alpha, x_i+\alpha[\) est admet des limites finies en \(y_{i-1}\) et \(y_{i}\). Donc  \(f_\alpha\) est continue par morceaux sur \([a+\alpha, b+\alpha]\).

\sphinxAtStartPar
si \(\varphi \in \mathcal E^-(f)\) donc \(\varphi \leq f\) donc \(\varphi_\alpha  \leq f_\alpha\)

\sphinxAtStartPar
donc \(\left\{\varphi_\alpha ~~ | ~~ \varphi \in \mathcal E^-(f) \right\} \subset \mathcal E^-(f_\alpha)\)

\sphinxAtStartPar
D’autre part, si \(\psi \in  \mathcal E^-(f_\alpha)\) donc il existe \(\phi \in  \mathcal E^-(f)\) telle que \(\psi = \varphi_\alpha\).

\sphinxAtStartPar
Donc, \(\mathcal E^-(f_\alpha)= \left\{\varphi_\alpha ~~ | ~~ \varphi \in \mathcal E^-(f) \right\}\)

\sphinxAtStartPar
Par suite
\begin{equation*}
\begin{split}
\begin{aligned}
\int_{[a+\alpha, b+\alpha]} f_{\alpha} &= sup \left\{\int_{[a+\alpha, b+\alpha]} \psi ~~| ~~ \psi \in \mathcal E^-(f_\alpha) \right\} \\ \\
&= sup \left\{\int_{[a+\alpha, b+\alpha]} \varphi_\alpha ~~| ~~ \varphi \in \mathcal E^-(f) \right\} \\ \\
&= sup \left\{\int_{[a, b]} \varphi ~~| ~~ \varphi \in \mathcal E^-(f) \right\} = \int_{[a, b]} f
\end{aligned}
\end{split}
\end{equation*}\end{sphinxadmonition}

\sphinxAtStartPar
Soient \(T>0\) et \(f\) une fonction \(T\)\sphinxhyphen{}périodique et continue par morceaux sur une période et donc sur tout segment de \(\mathbb R\). Nous avons :
\begin{equation*}
\begin{split}
\forall a \in \mathbb R, \int_{[a, a+ T]} f = \int_{[0, T]}f
\end{split}
\end{equation*}

\section{Fonctions continue par morceaux sur un intervalle}
\label{\detokenize{fcmint:fonctions-continue-par-morceaux-sur-un-intervalle}}\label{\detokenize{fcmint::doc}}
\sphinxAtStartPar
Dans cette partie, \(I\) désigne un intervalle de \(\mathbb R\).

\begin{sphinxadmonition}{note}{Définition}

\sphinxAtStartPar
Soit \(f\) une fonction définie sur \(I\). On dit que \(f\) est continue par morceaux sur \(I\) si elle est continue par morceaux sur tout segment de \(I\) (\([a, b]\) avec \(a, b \in I\) et \(a<b\)).
\end{sphinxadmonition}

\begin{sphinxadmonition}{note}{Exemples}

\sphinxAtStartPar
1\sphinxhyphen{} Une fonction continue sur \(I\) est continue par morceaux sur \(I\).
2\sphinxhyphen{} La fonction \( x \to x-E(x) \) est continue par morceaux sur \(\mathbb R\).
3\sphinxhyphen{} La fonction \(f\) définie sur \(\mathbb R\) par :
\begin{equation*}
\begin{split}f(0)=0 \mbox{  et  } \forall x \in [-1, 1] \setminus \{0\}, f(x)=\dfrac{1}{x}
\end{split}
\end{equation*}
\sphinxAtStartPar
n’est pas continue morceaux sur \(\mathbb R\) puisque elle n’est pas continue par morceaux sur \([-1, 1]\). Cependant, elle est continue par morceaux sur \(\mathbb R^*_+\) et \(\mathbb R^*_-\) car elle continue sur chacun de ces intervalles.
\end{sphinxadmonition}

\sphinxAtStartPar
\sphinxstylestrong{Notations}

\sphinxAtStartPar
Soient \(f\) une fonction continue par morceaux sur un intervalle \(I\), ainsi que \(a\) et \(b\) deux éléments de \(I\) ( a partir de maintenant, on a pas nécessairement \(a<b\)) On adopte les notations suivantes:
\begin{itemize}
\item {} 
\sphinxAtStartPar
si \(a<b\), \(\int_a^b f(x)dx = \int_{[a,b]} f\)

\item {} 
\sphinxAtStartPar
si \(a>b\), \(\int_a^b f(x)dx = -\int_{[b,a]} f\)

\item {} 
\sphinxAtStartPar
si \(a = b\), \(\int_a^b f(x)dx =0\)

\end{itemize}

\begin{sphinxadmonition}{warning}{Avertissement:}
\sphinxAtStartPar
Le résultat :
\begin{equation*}
\begin{split}
f \leq g \Rightarrow \int_a^b f(x)dx \leq \int_a^b g(x)dx
\end{split}
\end{equation*}
\sphinxAtStartPar
n’est pas valide que lorsque \(a\leq b\).
\end{sphinxadmonition}

\begin{sphinxadmonition}{note}{Proposition (Relation de Chasles)}

\sphinxAtStartPar
Si \(f\) est continue par morceaux sur un intervalle I, alors :
\begin{equation*}
\begin{split}
\forall a, b, c \in I, ~~ \int_a^b f(x)dx = \int_a^c f(x)dx + \int_c^b f(x)dx
\end{split}
\end{equation*}\end{sphinxadmonition}

\begin{sphinxadmonition}{note}{Démonstration}

\sphinxAtStartPar
Nous allons traiter le cas ou \(a=b=c\), \(a<c<b\) et \(a<b<c\). Les autres casd (\(b<a<c\), \(b<c<a\), \(c<a<b\) et \(c<b<a\)) sont similaire aux deux premiers cas.
\begin{itemize}
\item {} 
\sphinxAtStartPar
si \(a=b=c\) chaque intégrale vaut 0, le résultat est donx trivial.

\item {} 
\sphinxAtStartPar
si \(a<c<b\), c’est le cas qu’on a vu dans la proposition de la Relation de Chasles pour le cas d’une fonction continue par morceaux sur \([a, b]\).

\item {} 
\sphinxAtStartPar
si \(a<b<c\), on applique la même proposition (Relation de Chasles) pour \(f\) qui est continue par morceaux sur  \([a, c]\).

\end{itemize}

\sphinxAtStartPar
Donc
\begin{equation*}
\begin{split}
\int_a^c f(x)dx = \int_a^b f(x)dx + \int_b^c f(x)dx
\end{split}
\end{equation*}
\sphinxAtStartPar
Or \(\int_b^c f(x)dx = - \int_c^b f(x)dx\). Donc
\begin{equation*}
\begin{split}
\int_a^c f(x)dx = \int_a^b f(x)dx - \int_b^c f(x)dx
\end{split}
\end{equation*}
\sphinxAtStartPar
Et par suite
\begin{equation*}
\begin{split}
\int_a^b f(x)dx = \int_a^c f(x)dx + \int_c^b f(x)dx
\end{split}
\end{equation*}\end{sphinxadmonition}

\begin{sphinxadmonition}{note}{Proposition}

\sphinxAtStartPar
Si \(f\) est continue par morceaux et bornée sur \(I\), on :
\begin{equation*}
\begin{split}
\forall a, b \in I, ~~ \left|\int_a^b f(x)dx\right| \leq |b-a| sup_{I} |f|
\end{split}
\end{equation*}\end{sphinxadmonition}

\begin{sphinxadmonition}{note}{Démonstration}

\sphinxAtStartPar
Nous avons 3 cas : \(a=b\), \(a<b\) et \(a>b\).

\sphinxAtStartPar
1\sphinxhyphen{} si \(a=b\) alors \$\textbackslash{}left|\textbackslash{}int\_a\textasciicircum{}b f(x)dx\textbackslash{}right| = 0|, le résultat est immédiat.

\sphinxAtStartPar
2\sphinxhyphen{} si \(a<b\), la fonction \(f\) est continue par morceaux sur \([a, b]\). Donc, nous avons \(\left|\int_{[a, b]}f \right| \leq (b-a)sup_{[a, b]}|f|\)

\sphinxAtStartPar
Donc \(\left|\int_{a}^bf(x)dx \right| \leq (b-a)sup_{[a, b]}|f|\)

\sphinxAtStartPar
Or \(sup_{[a, b]}|f| \leq sup_{I}|f|\)

\sphinxAtStartPar
Donc, \(\left|\int_{a}^bf(x)dx \right| \leq (b-a)sup_{I}|f|\)

\sphinxAtStartPar
3\sphinxhyphen{} si \(a>b\), en applique les mêmes étapes précédentes pour la fonction \(f\) est continue par morceaux sur \([a, b]\) et on reçoit \(\left|\int_{b}^af(x)dx \right| \leq (b-a)sup_{I}|f|\)

\sphinxAtStartPar
Et puisque \(\int_a^b f(x)dx = - \int_b^a f(x)dx\) donc \(\left|\int_a^b f(x)dx\right| = \left|- \int_b^a f(x)dx\right| = \left| \int_b^a f(x)dx\right|\).

\sphinxAtStartPar
En fin, \(\left|\int_{a}^bf(x)dx \right| \leq (b-a)sup_{I}|f|\)
\end{sphinxadmonition}


\section{Primitive et intégrale des fonctions continues}
\label{\detokenize{pintfc:primitive-et-integrale-des-fonctions-continues}}\label{\detokenize{pintfc::doc}}
\sphinxAtStartPar
Dans tout ce qui suit, \(I\) désigne un intervalle de \(\mathbb R\) contenant au moins deux points distincts.


\subsection{Primitive d’une fonction continue sur un intervalle}
\label{\detokenize{pintfc:primitive-d-une-fonction-continue-sur-un-intervalle}}
\begin{sphinxadmonition}{note}{Définition}

\sphinxAtStartPar
Soit \(f\) est une fonction de \(I\) dans \(\mathbb R\) continue sur \(I\). On appelle primitive de \(f\) sur \(I\) toute fonction de \(I\) dans \(\mathbb R\), dérivable sur \(I\) et dont la dérivée est \(f\).
\end{sphinxadmonition}

\begin{sphinxadmonition}{note}{Proposition}

\sphinxAtStartPar
Soit \(f\) est une fonction de \(I\) dans \(\mathbb R\) continue sur \(I\).

\sphinxAtStartPar
Si \(F\) est une primitive de \(f\) sur \(I\), alors les primitives de \(f\) sur \(I\) sont les fonctions \(F+\lambda\) avec \(\lambda \in \mathbb R\).
\end{sphinxadmonition}

\begin{sphinxadmonition}{note}{Démonstration}

\sphinxAtStartPar
Les fonctions \(F+\lambda\) sont dérivables sur \(I\) est leurs dérivées valent \(f\). Donc \(F+\lambda\) sont des primitives de \(f\).

\sphinxAtStartPar
Soit \(G\) une primitive de \(f\) donc \(G-F\) a une dérivée nulle sur \(I\). Donc \(G-F\) est une constante sur \(I\).
\end{sphinxadmonition}


\subsection{Primitives usuelles}
\label{\detokenize{pintfc:primitives-usuelles}}
\sphinxAtStartPar
Les deux tableaux suivants contiennent les primitives des fonctions usuelles :

\noindent{\hspace*{\fill}\sphinxincludegraphics[width=500\sphinxpxdimen]{{prim11}.png}\hspace*{\fill}}

\noindent{\hspace*{\fill}\sphinxincludegraphics[width=500\sphinxpxdimen]{{prim21}.png}\hspace*{\fill}}

\begin{sphinxadmonition}{note}{Exemple}

\sphinxAtStartPar
Les primitives d’une fonction polynomiale de la forme
\begin{equation*}
\begin{split}
f(x) = \sum_{k=0}^n a_k x^k
\end{split}
\end{equation*}
\sphinxAtStartPar
sont de la forme \(F + \lambda\) avec:
\begin{equation*}
\begin{split}
F(x) = \sum_{k=0}^n \dfrac{a_k}{k+1} x^{k+1} ~~~ \mbox{ et } \lambda \in \mathbb{R}
\end{split}
\end{equation*}\end{sphinxadmonition}


\subsection{Théorème fondamental}
\label{\detokenize{pintfc:theoreme-fondamental}}
\begin{sphinxadmonition}{note}{Proposition}

\sphinxAtStartPar
Soient \(f\) une fonction continue par morceaux sur \(I\) et \(a\) un point de \(I\). La fonction \(F_a\) définie par :
\begin{equation*}
\begin{split}
F_a(x) = \int_a^x f(t) dt
\end{split}
\end{equation*}
\sphinxAtStartPar
est continue sur \(I\)
\end{sphinxadmonition}

\begin{sphinxadmonition}{note}{Théorème}

\sphinxAtStartPar
Soient \(F\) une fonction continue de \(I\) dans \(\mathbb{R}\) et \(a\) un point de \(I\). La fonction \(F_a\) définie par :
\begin{equation*}
\begin{split}
F_a(x) = \int_a^x f(t) dt
\end{split}
\end{equation*}
\sphinxAtStartPar
est une primitive de \(f\) sur \(I\). C’est l’unique primitive qui s’annule en \(a\).
\end{sphinxadmonition}

\begin{sphinxadmonition}{note}{Corollaire}

\sphinxAtStartPar
Soient f une fonction continue sur \(I\), ainsi que \(\alpha\) et \(\beta\) deux fonctions dérivables sur un intervalle \(J\) et a valeurs dans \(I\). La fonction définie sur \(J\) par :
\begin{equation*}
\begin{split}
\varphi(x) = \int_{\alpha(x)}^{\beta(x)} f(t) dt
\end{split}
\end{equation*}
\sphinxAtStartPar
est dérivable sur \(J\) et sa dérivée est:
\begin{equation*}
\begin{split}
\varphi^{'}(x) = \beta^{'}(x)f(\beta(x))- \alpha^{'}(x)f(\alpha(x))
\end{split}
\end{equation*}\end{sphinxadmonition}

\begin{sphinxadmonition}{note}{Exemple}
\begin{itemize}
\item {} 
\sphinxAtStartPar
Si \(f\) est une fonction continue par morceaux sur \(\mathbb{R}\) et périodique de période \(T\), alors:

\end{itemize}
\begin{equation*}
\begin{split}
g(x) = \int_x^{x+T}f(t)dt
\end{split}
\end{equation*}
\sphinxAtStartPar
est indépendante de \(x\) (constante), car:
\begin{equation*}
\begin{split}
g'(x) = f(x+T) - f(x) = 0
\end{split}
\end{equation*}\end{sphinxadmonition}

\begin{sphinxadmonition}{note}{Proposition}

\sphinxAtStartPar
Soient \(f\) une fonction continue par morceaux sur \(I\) et \(a\) un point de \(I\). Si \(F\) est une primitive de \(f\) sur \(I\), on a :
\begin{equation*}
\begin{split}
\int_a^b f(t) dt = F(b) - F(a)
\end{split}
\end{equation*}\end{sphinxadmonition}

\begin{sphinxadmonition}{note}{Exemple}
\begin{itemize}
\item {} 
\end{itemize}
\begin{equation*}
\begin{split}
\int_a^b e^{2x} dx = \dfrac{1}{2} (e^{2b} - e^{2a})
\end{split}
\end{equation*}\begin{itemize}
\item {} 
\end{itemize}
\begin{equation*}
\begin{split}
\int_0^\pi \sin x dx = - \cos \pi + \cos 0 = 2
\end{split}
\end{equation*}\begin{itemize}
\item {} 
\sphinxAtStartPar
Soit \(f(x)=\alpha x + \beta\). Une primitive de \(f\) est \(x \mapsto \dfrac{\alpha}{2}x^2 + \beta x\), donx:

\end{itemize}
\begin{equation*}
\begin{split}
\int_a^bf(x)dx = \dfrac{\alpha}{2}(b^2 - a^2) + \beta (b-a) = (b-a)\dfrac{f(a)+ f(b)}{2}
\end{split}
\end{equation*}\end{sphinxadmonition}

\begin{sphinxadmonition}{note}{Corollaire}

\sphinxAtStartPar
Si \(f \in \mathcal C (I)\)(dérivable et sa dérivée est continue), alors pour \(a, x \in I\) on a:
\begin{equation*}
\begin{split}
f(x) - f(a) = \int_a^x f^{'}(t)dt
\end{split}
\end{equation*}\end{sphinxadmonition}

\sphinxAtStartPar
\sphinxstylestrong{Notations}:
\begin{itemize}
\item {} 
\sphinxAtStartPar
Dans ce qui suit, on va noter la différence de la fonction \(F\) entre \(a\) et \(b\): \([F(x)]_a^b\). Ceci dit,

\end{itemize}
\begin{equation*}
\begin{split}
\int_a^b f(x)dx = F(b) - F(a) =[F(x)]_a^b
\end{split}
\end{equation*}\begin{itemize}
\item {} 
\sphinxAtStartPar
Lorsque \(f\) est une fonction continue, la notation \(\int f(x)dx\) représente une primitive quelconque de la fonction \(f\) (\(\int f(x)dx = F(x) = Cst\)).

\end{itemize}


\section{Méthodes de calcul des primitives}
\label{\detokenize{methodcalp:methodes-de-calcul-des-primitives}}\label{\detokenize{methodcalp::doc}}

\subsection{Intégration par parties}
\label{\detokenize{methodcalp:integration-par-parties}}
\begin{sphinxadmonition}{note}{Proposition}

\sphinxAtStartPar
Si \(u\) et \(v\) sont deux fonctions de classe \(\mathcal C^1\) sur le segment \([a, b]\), on a:
\begin{equation*}
\begin{split}
\int_a^b u(t)v^{'}(t)dt = u(b)v(b)- u(a)v(a) - \int_a^b u^{'}(t)v(t)dt
\end{split}
\end{equation*}\end{sphinxadmonition}

\begin{sphinxadmonition}{note}{Démonstration}

\sphinxAtStartPar
Si \(u\) et \(v\) sont deux fonctions de classe \(\mathcal C^1\), alors \(uv\) est aussi de classe \(\mathcal C^1\).

\sphinxAtStartPar
Donc
\begin{equation*}
\begin{split}
u(b)v(b)- u(a)v(a) = \int_a^b (uv)^{'}(t)dt = \int_a^b u(t)v^{'}(t)dt + \int_a^b u^{'}(t)v(t)dt
\end{split}
\end{equation*}\end{sphinxadmonition}

\sphinxAtStartPar
\sphinxstylestrong{Remarque}: Dans un calcul de primitive, la formule d’intégration par parties s’écrit:
\begin{equation*}
\begin{split}
\int u(x)v^{'}(x)dx = u(x)v(x) - \int v(x)u^{'}(x)dx
\end{split}
\end{equation*}
\sphinxAtStartPar
La formule d’intégration par parties est en général utilisée pour:
\begin{itemize}
\item {} 
\sphinxAtStartPar
éliminer une fonction transcendantes dont la dérivée est plus simple comme par exemple les fonctions \(\ln, \arcsin, \arctan,\ldots\)

\item {} 
\sphinxAtStartPar
calculer une intégrale par récurrence.

\end{itemize}

\begin{sphinxadmonition}{note}{Exemples}

\sphinxAtStartPar
1\sphinxhyphen{} sur \(\mathbb R_+^*\) on a :
\begin{equation*}
\begin{split}
\int \ln x dx = \int (x)^{'}\ln x dx = x\ln x - \int x\dfrac{1}{x} = x\ln x = x + Cst 
\end{split}
\end{equation*}
\sphinxAtStartPar
2\sphinxhyphen{} Sur \(\mathbb R\) on a:
\begin{equation*}
\begin{split}
\int \arctan x dx = x \arctan x - \int \dfrac{x}{1+x^2} dx = x \arctan x - \dfrac{1}{2}\ln (1+x^2) + Cst
\end{split}
\end{equation*}
\sphinxAtStartPar
3 \sphinxhyphen{} Pour calculer \(\int x^2 e^x dx\), on peut intégrer l’exponentielle et dériver le polynôme:
\begin{equation*}
\begin{split}
\int x^2 e^x dx = x^2e^x - 2\int xe^x dx
\end{split}
\end{equation*}
\sphinxAtStartPar
puis recommencer:
\begin{equation*}
\begin{split}
\int xe^x dx = xe^x - \int e^x dx
\end{split}
\end{equation*}
\sphinxAtStartPar
ce qui donne:
\begin{equation*}
\begin{split}
\int x^2 e^x dx = x^2e^x - 2(xe^x - \int e^x dx) = x^2e^x - 2xe^x + 2e^x + Cst
\end{split}
\end{equation*}\end{sphinxadmonition}


\subsection{Changement de variable}
\label{\detokenize{methodcalp:changement-de-variable}}
\begin{sphinxadmonition}{note}{Proposition}

\sphinxAtStartPar
Soient \(I\) et \(J\) deux intervalle de \(\mathbb R\), ainsi que \(f\) une fonction continue de \(I\) dans \(\mathbb R\) et \(\varphi\) une fonction de classe \(\mathcal C^1\) de \(J\) dans \(I\). Si \(\alpha\) et \(\beta\) sont deux éléments de \(J\), on a :
\begin{equation*}
\begin{split}
\int_{\varphi(\alpha)}^{\varphi(\beta)} f(t)dt = \int_\alpha^\beta f(\varphi(u)) \varphi^{'}(u)du
\end{split}
\end{equation*}\end{sphinxadmonition}

\begin{sphinxadmonition}{note}{Démonstration}

\sphinxAtStartPar
Comme \(f\) est continue sur \(I\), elle possède une primitive \(F\) et l’on a:
\begin{equation*}
\begin{split}
\int_{\varphi(\alpha)}^{\varphi(\beta)} f(t)dt = F(\varphi(\beta)) - F(\varphi(\alpha)) = F \circ\varphi(\beta) - F \circ\varphi(\alpha)
\end{split}
\end{equation*}
\sphinxAtStartPar
D’autre part, puisque les deux fonctions \(F\) et \(varphi\) sont de classe \(\mathcal C^1\) donc \(F \circ\varphi\) est aussi de classe \(\mathcal C^1\)

\sphinxAtStartPar
Donc on peut écrire : \( F \circ\varphi(\beta) - F \circ\varphi(\alpha) = F(\varphi(\beta)) - F(\varphi(\alpha)) = \int_\alpha^\beta  (F \circ\varphi)^{'}(u) du\).

\sphinxAtStartPar
Par suite :
\begin{equation*}
\begin{split}
\begin{aligned}
F(\varphi(\beta)) - F(\varphi(\alpha)) &= \int_\alpha^\beta  (F \circ\varphi)^{'}(u) du \\ \\
& = \int_\alpha^\beta  F^{'}(\varphi(u)) \varphi^{'}(u) du \\ \\
& = \int_\alpha^\beta f(\varphi(u)) \varphi^{'}(u)du
\end{aligned}
\end{split}
\end{equation*}\end{sphinxadmonition}

\sphinxAtStartPar
\sphinxstylestrong{Remarques}:
\begin{itemize}
\item {} 
\sphinxAtStartPar
la formule de changement de variable n’est que la formule de dérivation d’une fonction composée lue à l’envers.

\item {} 
\sphinxAtStartPar
Quand on utilise la formule de changement de variable avec les notations vues dans la proposition, on dit que l’on effectue le changement de variable \(t=\varphi(u)\) (d’où l’appellation changement de variable). On remplace alors \(t\) par \(\varphi(u)\) et \(dt\) par la différentielle \(\varphi^{'}(u)du\), ce qui rend le calcul assez naturel.

\item {} 
\sphinxAtStartPar
il faut faire attention lors de l’application de cette méthode, les bornes de l’intégral doivent être changées.

\end{itemize}

\begin{sphinxadmonition}{note}{Exemples}

\sphinxAtStartPar
1\sphinxhyphen{} Pour calculer l’intégrale :
\begin{equation*}
\begin{split}
\int_0^{\frac{\pi}{2}} \sin^2 u \cos u du
\end{split}
\end{equation*}
\sphinxAtStartPar
On pose \(t = \sin u\), donc
\begin{equation*}
\begin{split}
\int_0^{\frac{\pi}{2}} sin^2 u \cos u du = \int_0^{1} t^2 dt = \dfrac{1}{3}
\end{split}
\end{equation*}
\sphinxAtStartPar
2\sphinxhyphen{} Pour calculer l’intégrale
\begin{equation*}
\begin{split}
\int_{-1}^2 \sqrt{4-u^2} u du
\end{split}
\end{equation*}
\sphinxAtStartPar
On pose \(t = u^2\), donc :
\begin{equation*}
\begin{split}
\int_{-1}^2 \sqrt{4-u^2} u du = \dfrac{1}{2}\int_1^4 \sqrt{4-t}dt = [-\dfrac{1}{3}(4-t^2)^{\frac{3}{2}}]_1^4 = \sqrt{3}
\end{split}
\end{equation*}\end{sphinxadmonition}


\section{Exercices}
\label{\detokenize{exoint:exercices}}\label{\detokenize{exoint::doc}}

\subsection{Exercice 1}
\label{\detokenize{exoint:exercice-1}}
\sphinxAtStartPar
Trouver les primitives suivantes :
\begin{itemize}
\item {} 
\sphinxAtStartPar
a) \( \int (2x^2 + 3x - 5)dx\)

\item {} 
\sphinxAtStartPar
b) \(\int (x-1) dx\)

\item {} 
\sphinxAtStartPar
c) \(\int \dfrac{(1-x)^2}{\sqrt{x}}dx\)

\item {} 
\sphinxAtStartPar
d) \(\int \dfrac{x+3}{x+1}dx\)

\end{itemize}


\subsection{Exercice 2}
\label{\detokenize{exoint:exercice-2}}
\sphinxAtStartPar
Calculer :
\begin{itemize}
\item {} 
\sphinxAtStartPar
a) \(\int x\sqrt{1+x}dx\)

\item {} 
\sphinxAtStartPar
b) \(\int x^3e^{2x}dx\)

\item {} 
\sphinxAtStartPar
c) \(\int x^2ln(x)dx\)

\end{itemize}


\subsection{Exercice 3}
\label{\detokenize{exoint:exercice-3}}
\sphinxAtStartPar
Soit \(f\) une fonction continue de \([a, b]\) dans \(\mathbb R\) telle que:
\begin{equation*}
\begin{split}
\forall x \in [a, b], f(a+b-x)=f(x)
\end{split}
\end{equation*}
\sphinxAtStartPar
Montrer que :
\begin{equation*}
\begin{split}
\int_a^b xf(x)dx = \dfrac{a+b}{2}\int_a^b f(x)dx
\end{split}
\end{equation*}

\subsection{Exercice 4}
\label{\detokenize{exoint:exercice-4}}
\sphinxAtStartPar
En utilisant la reconnaissance de forme déterminer toutes les primitives des fonctions suivantes :
\begin{itemize}
\item {} 
\sphinxAtStartPar
\(f(x)=\dfrac{x}{1+x^2}\)

\item {} 
\sphinxAtStartPar
\( g(x) = \dfrac{e^{3x}}{1+e^{3x}}\)

\item {} 
\sphinxAtStartPar
\(h(x) = \dfrac{ln(x)}{x}\)

\item {} 
\sphinxAtStartPar
\(k(x) = cos(x)sin^2(x)\)

\item {} 
\sphinxAtStartPar
\(l(x) = \dfrac{1}{xln(x)}\)

\item {} 
\sphinxAtStartPar
\( m(x) = 3x\sqrt{1+x^2}\)

\end{itemize}


\subsection{Exercice 5}
\label{\detokenize{exoint:exercice-5}}\begin{itemize}
\item {} 
\sphinxAtStartPar
Calculer \(I_n = \int ln^n (x)dx\) pour \(n = 0; 1; 2\).

\item {} 
\sphinxAtStartPar
Calculer \(I_n\) en fonction de \(I_{n-1}\).

\end{itemize}


\subsection{Exercice 6}
\label{\detokenize{exoint:exercice-6}}
\sphinxAtStartPar
Calculer avec deux méthodes (reconnaissance de la forme et changement de variable) les primitives de la fonction suivantes :
\begin{itemize}
\item {} 
\sphinxAtStartPar
\(f(x) = cos^{1234}(x)sin(x)\)

\item {} 
\sphinxAtStartPar
\( g(x) = \dfrac{1}{xln(x)}\)

\end{itemize}


\subsection{Exercice 7}
\label{\detokenize{exoint:exercice-7}}
\sphinxAtStartPar
Calculer les intégrales suivantes :
\begin{itemize}
\item {} 
\sphinxAtStartPar
\(\int_0^{\frac{\pi}{2}} xsin(x)dx\) (par parties)

\item {} 
\sphinxAtStartPar
\(\int_0^1 \dfrac{e^x}{\sqrt{e^x}+1}\) (changement de variable)

\end{itemize}







\renewcommand{\indexname}{Index}
\printindex
\end{document}